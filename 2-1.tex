\chapter{Indledning}
\renewcommand{\labelprefix}{ch:intro}
\llabel{}
\vspace*{-4.5cm}
\mbox{}\hspace{\fill}\includegraphics[width=5cm]{img/stonehenge2.eps} 
\vspace{1cm}


\aufmacher{\noindent Hvis man vil blive billedhugger,\footnote{%
Billedet af stenkredsen ved Stonehenge er taget fra \cite{Pin1808}.}
skal man lære en masse grundlæggende teknikker:
Hvor finder man passende sten?
Hvordan flytter man dem, hvordan virker mejslen, hvordan bygger man et stillads, osv.
Når de grundlæggende teknikker er på plads, er man langtfra nogen berømt kunstner, men selv en sjældent talent kan ikke blive berømt uden at beherske grundlaget.
 Wer die Grundtechniken beherrscht, 
Man behøver ikke kende det hele, inden man går i gang med sin første skulptur
Men man skal være parat til at gå tilbage til grundteknikkerne for at blive bedre og bedre.
}

Dette indledende kapitel spiller en lignende rolle i bogen.
Vi præsenterer de grundlæggende begreber og metoder for at bedre kunne beskrive og analysere algoritmer i de senere kapitler.
Man behøver ikke at arbejde sig igennem dette kapitel fra A til Z inden man giver sig i kast med de følgende kapitler.
Vi anbefaler ved første læsning at studere materialet til og med afsnit~\lref{s:pseudocode} grundigt, og skimme de følgende afsnit.
Afsnit~\lref{s:o} beskriver notation og terminologi for at beskrive algoritmers kompleksitet kort og præcist.
I afsnit~\lref{s:model} præsenteres en enkel beregningsmodel, som gør det muligt at abstrahere bort fra mange af de komplikationer, der ville opstå ved at tage hensyn til egenskaberne ved moderne maskinarkitektur.
Modellen er tilstrækkelig konkret til at levere nyttige forudsigelser, men tilstrækkelig abstrakt til at tillade elegante overvejelser.
I afsnit~\lref{s:pseudocode} introduceres et notation for pseudokode, som minder om et højniveausprogrammeringssprog og tillader en bekvem beskrivelse af algoritmer end maskinmodellens kode.
Desuden giver pseudokoden mulighed for at benytte notation fra matematikken, uden at vi behøver at bekymre os om, hvordan denne ville skulle oversættes til en ægte maskine. 
Vi vil gøre hyppig brug af kommentarer i programmerne, både for at øge deres læsbarhed og for at gøre det letter at føre formelle korrekthedsbeviser.
Teknikker for den slags beviser er genstant for afsnit~\lref{s:correct}.
%
Afsnit~\lref{s:binary search} indeholder det første omfattende eksempel:
Binærsøgning i en sorteret række.
I afsnit~\lref{s:analysis} beskrives matematiske teknikker for programmers kompleksitetsanalyse med vægt på indlejrede løkker og rekursive procedurekald. 
For analysen af gennemsnitligt tidsforbrug
\index{algoritmenanalyse!gennemsnit}
har vi brug for yderligere teknikker; disse beskrives i afsnit~\lref{s:average case analysis}.
Randomiserede algoritmer, præsenteret i afsnit ~\lref{s:random}, gør brug af tilfældighed ved under udførelsen at kunne slå plat og krone.
Afsnit~\lref{s:graphnot} handler om grafer, et begreb som spiller en stor rolle i resten af bogen.
I afsnit~\lref{s:P and NP} diskuteres spørgsmålet, hvornår man skal betegne en algoritme som effektiv, og kompleksitetsklasserne $\classP$ og $\NP$ samt den vigtige klasse af $\NP$-fuldstændige problemer. 
Som alle andre kapitler slutter kapitlet med implementationsaspekter
(afsnit~\lref{s:implementation}) og historiske anmærkninger og videre resultater 
(afsnit~\lref{s:further}).

%%%%%%%%%%%%%%%%%%%%%%%%%%%%%%%%%%%%%%%%%%%%%%%%%%%%%%%%%%%%%%%%%%%%%%
\section{Asymptotisk notation}\llabel{s:o}

Algoritmeanalysens formål er primært at skabe tilforladelige udsagn om algoritmers opførsel, bl.a. deres kørselstid, som er både præcise, kortfattede, almentgyldige og begribelige.
Det er selvfølgeligt vanskeligt at opfylde alle disse krav samtidigt.
For fx at beskrive algoritmens tidsforbrug $T$
\index{tid|siehe{kørselstid}}
\index{regnetid|siehe{kørselstid}}
\index{beregningstid|siehe{kørselstid}}
\index{udførelsestid|siehe{kørselstid}}
\index{kørselstid}
kan man opfatte $T$ som en funktion, der afbilder mængden
 $\Inputs$
 af alle mulige probleminstanser (eller \emph{input})
\index{instans} 
til mængden $\RR_+$ af positive reelle tal.
For hver instans 
\index{probleminstans}
\index{instans}
$\Input$ til problemet er da $T(\Input)$ kørselstiden på $\Input$. 
Denne detaljeringsgrad fører dog til så overvældende meget information, at det ville være håbløst at udvikle en brugbar teori.
I stedet skal vi betragte  algoritmens opførsel i et mere mere almengyldigt perspektiv.

Vi vil opdele mængden af instanser i klasser af »lignende« instanser og så sammenfatte algoritmens opførsel på instanser fra samme klasse som et eneste tal.
Det mest gængse kriterium for klassedelingen er 
\index{instansstørrelse}
\index{størrelse!instans-}
instansens \emph{størrelse}.
Sædvanligvis er der en naturlig måde for at bestemme instansens størrelse.
Størrelsen på et heltal er antallet af cifre i dens binær\-repræsentation;
størrelsen af en mængde er dens kardinalitet, dvs. antallet af elementer.
Instansstørrelsen er altid et naturligt tal.
Sommetider bruger man mere end én parameter for at angive størrelsen på en instans; fx er det gængs at karakteriser størrelsen på en graf i termer af både antal knuder og antal kanter.
Vi vil i første omgang se bort fra de komplikationer, der optræder herved.
% TODO size notation introduced in original, but never used.
%Størrelsen af instansen $\Input$ skrives som $\Size(\Input)$,
%\index{Size@\ensuremath{\Size}|siehe{instansstørrelse}}
%og
Mængden af alle instanser af størrelse $n$ skrives som  $\Inputs_n$ for $n \in \NN$. 
For instanser af størrelse $n$ kan vi interessere os for maksimale, minimale og gennemsnitlige kørselstider, defineret på følgende måde:
\footnote{Vi vil altid sikre, at mængden 
  $\setGilt{T(\Input)}{\Input\in\Inputs_n}$ har både maksimum og minimum, og at mængden 
  $\Inputs_n$ er endelig, når vi beregner gennemsnit.}
%
\[ 
T(n) = \begin{cases}
  \max\setGilt{T(\Input)}{\Input\in\Inputs_n} &
  \text{»i værste fald«}\,, \\
  \min\setGilt{T(\Input)}{\Input\in\Inputs_n} &
\text{»i bedste fald«}\,, \\
      \displaystyle\frac{1}{|\Inputs_n|}\sum_{\Input\in \Inputs_n}T(\Input) &
  \text{»i gennemsnit«}\,.
\end{cases}
\]
\index{værste fald|sieheunter{kørselstid}}
\index{bedste fald|sieheunter{kørselstid}}
\index{gennemsnit|sieheunter{kørselstid}}
\index{kørselstid!værste fald|textbf}
\index{kørselstid!bedste fald|textbf}
\index{kørselstid!gennemsnit|textbf}%
Den mest interessante af disse størrelse er kørselstiden i værste fald, fordi den udgør den mest omfattende garanti for algoritmens opførelse.
Sammenligningen af opførelsen i bedste og værste fald fortæller os, hvor stor variation i kørselstid der kan forekomme mellem instanser i samme størrelsesklasse.
Når afvigelsen mellem bedste og værste fald er meget stor, kan sommetider en analyse af den gennemsnitlige kørselstid give nærmere indsigt i algoritmens faktiske tidsforbrug.
Vi skal se nærmere på en eksempel i afsnit~\lref{s:average case analysis}.

Vi går endnu et skridt videre i vores forsøg på at informationen mere overskuelig ved at gøre analysen grovere:
Vi koncentrerer os på kørselstidens \emph{vækstrate} ved at bruge \emph{asymptotisk analyse}.
\index{asymptotisk analyse}  
To funktioner $f(n)$ og $g(n)$ har samme 
\index{vækstrate}
\emph{vækstrate}, hvis der eksisterer positive konstanter $c$ og $d$, så uligheden $c\le f(n)/g(n)\le d$ gælder der for alle tilstrækkeligt store $n$.
Funkionen $f(n)$ \emph{vokser hurtigere} end $g(n)$, hvis der gælder for hver positive konstant $c$, at uligheden $f(n)\ge c\cdot g(n)$ for alle tilstrækkeligt store $n$.
For eksempel har funktionerne $n^2$, $n^2 + 7n$, $5n^2 - 7n$ og
$\frac{1}{10}n^2 + 10^6 n$ alle samme vækstrate.
Desuden vokser disse funktioner alle hurtigere end funktionen $n^{3/2}$, som selv vokser hurtigere end funktionen  $n \log n$.
Læg mærke til, at vækstraten fokuserer på opførslen for store $n$, hvilket også er meningen med begrebet »asymptotisk« i »asymptotisk analyse«.\footnote{Ovs. anm.:
Ordet »asymptotisk« er kendt fra matematisk analyse, hvor det betegner opførslen af en funktion, som nærmer sig en grænseværdi uden at antage den.
Funktionen $x\mapsto 1/x$ har fx $y$-aksen som asymptote for $x\rightarrow 0$.
Ordet er græsk og betyder »ikke-sammenfaldende« og betoner altså, at funktionen nærmer sig, men aldrig helt når, en bestemt linje.
I modsætning hertil vil man lede forgæves efter en god forklaring for, hvorfor man i algoritmeanalysen bruger »asymptotisk« for at betegne noget i retning af »opførsel for store $n$«.}

Hvad er grunden til, at vi kun interesserer os for vækstrate og opførslen for store $n$?
Hovedårsagen for at udvikle effektive algoritmer er netop ønsket om at kunne håndtere store instanser.
Når algoritmen $A$ har en lavere vækstrate end algoritmen $B$ for samme problem, vil $A$ typisk være $B$ overlegen for store $n$.
Desuden er vores maskinmodel 
\index{maskinmodel}
i forvejen en abstraktion af de faktiske kørselstider og nøjes med at bestemme en konkret maskines opførelse inden for en maskinafhængig konstant faktor.
\index{konstant faktor}
Derfor vil vi ikke skelne mellem algoritmer, hvis kørselstider har samme vækstrate.
Vores indskrænkning til vækstrater har desuden den glædelige sideeffekt, at algoritmers kørselstider kan karakteriseres af meget enkle funktioner.
Vi vil dog i bogens implementationsafsnit regelmæssigt se lidt nærmere på de maskinnære detaljer, som den asymptotiske analyse er blind for. 
Generelt bør læseren ved studiet og anvendelsen af algoritmer i denne bog altid spørge sig selv, om det asymptotiske perspektiv er relevant.

Vi skal nu indføre den gængse notation for funktioners \emph{asymptotiske opførsel}.
\index{asymptotisk|textbf}%
\index{omega@$\Omega(\cdot)$|textbf}% 
\index{omega@$\omega(\cdot)$|textbf}% 
\index{o@$o(\cdot)$|textbf}% 
\index{theta@$\Theta(\cdot)$|textbf}% 
\index{o@$O(\cdot)$|textbf}
Her betegner $f(n)$ og $g(n)$ funktioner, som afbilder naturlige tal til ikke-negative reelle tal. 
Vi definerer
\begin{align*}
  O(f(n)) & = \{\,g(n)\colon\exists c>0\colon\exists n_0\in\NN_+\colon\forall n\geq n_0\colon g(n)\leq c\cdot f(n)\}\,,\\
\Omega(f(n)) & = \{\,g(n)\colon\exists c>0\colon\exists n_0\in\NN_+\colon\forall n\geq n_0\colon g(n)\geq c\cdot f(n)\,\}\,,\\
  \Theta(f(n)) & = O(f(n))\cap{}\Omega(f(n))\,,\\
o(f(n)) & = \{\,g(n)\colon\forall c>0\colon\exists n_0\in\NN_+\colon\forall n\geq n_0\colon g(n)\leq c\cdot f(n)\,\}\,,\\
\omega(f(n)) & = \{\,g(n)\colon\forall c>0\colon\exists n_0\in\NN_+\colon\forall n\geq n_0\colon g(n)\geq c\cdot f(n)\,\}\,.
\end{align*}
Venstresiderne læses som »store-o af $f(n)$« og tilsvarende for  »store-omega«, »theta«, »lille-o« og »lille-omega«.
Læg mærke til, at »$f(n)$« i udtrykket  »$O(f(n))$« og »$g(n)$« i udtrykket »$\{\,g(n)\colon \ldots\}$« betegner funktionerne $f$ og $g$ -- notationen forsøger blot at tydeliggøre, at funktionen afhænger af variablen $n$.
Derimod menes i betingelsen »$\forall n\geq n_0\colon g(n) \le c\cdot f(n)$« funktions\emph{værdien} for $n$.

Lad os betragte nogle eksempler.
Mængden $O(n^2)$ indeholder de funktioner, som vokser højst kvadratisk.
Mængden $o(n^2)$ indeholder de funktioner ,som vokser langsommere end kvadratisk.
Mængden $o(1)$ indeholder de funktioner, som går mod $0$ for voksende $n$, hvor strengt taget symbolet »$1$« betegner den konstante funktion $n \mapsto 1$, som altid har funktionsværdien $1$.
Dermed tilhører funktionen $f(n)$ mængden $o(1)$, hvis $f(n) \le c\cdot 1$ for hvert positive $c$ og tilstrækkeligt stort $n$, dvs. når $f(n)$ går mod $0$ for voksende $n$.
Generelt kan man tænke på $O(f(n))$ som mængden af funktioner, som »ikke vokser hurtigere end« $f(n)$; og på $\Omega(f(n))$ som mængden af funktioner, som »vokser mindst lige så hurtigt som« $f(n)$.
For eksempel ligger den asymptotiske værstefaldstid for Karatsubas algoritme for heltalsmultiplikation i $O(n^{1,585})$, mens den asymptotiske kørselstid for skolemetoden ligger i $\Omega(n^2)$. 
Derfor kan vi sige, at Karatsubas algoritme er hurtigere end skolemetoden.
Notationen $o(f(n))$ angiver mængden af funktioner, som »vokser skarpt langsommere end« $f(n)$.
Dens modsætning, notationen $\omega(f(n))$, forekommer ganske sjældent i den grundlæggende algoritmeanalyse og er her kun medtaget for fuldstændighedens skyld.

De fleste algoritmer i denne bog har kørselstider, som kan skrives som polynomium eller som logaritmisk funktion, eller som produkt af den sådanne funktioner.
Det næste resultat ser nærmere på polynomier i det asymptotiske perspektiv; beviset giver nogle eksempler på omgangen med notationen.

\begin{lemma}\llabel{lem:polynomial}
  Lad $p(n)=\sum_{i=0}^ka_in^i$ være et polynomium med reelle koefficienter, hvor $a_k>0$.
  Da gælder $p(n)\in \Theta(n^k)$.
\end{lemma}
\begin{proof}
  Vi skal vise $p(n)\in O(n^k)$ og $p(n)\in \Omega(n^k)$.
  Vi bemærker først, at der for $n>0$ gælder
  \[ p(n)\leq\sum_{i=0}^k\abs{a_i}n^i\leq n^k\sum_{i=0}^k\abs{a_i}\,, \]
  hvilket medfører $p(n)\leq (\sum_{i=0}^k\abs{a_i})n^k$ for alle positive $n$. 
  Derfor gælder $p(n)\in O(n^k)$.

  Sæt $A=\sum_{i=0}^{k-1}\abs{a_i}$.
  For alle $n>0$ har vi nu
\[ p(n)\geq
  a_kn^k-An^{k-1}=\frac{a_k}{2}n^k+n^{k-1}\left(\tfrac{a_k}{2}n-A\right)\,, \]
  og derfor $p(n) \geq (\frac{1}{2}a_k)n^k$ for $n > 2A/a_k$.
  Ved at vælge  $c=\frac{1}{2}a_k$ og $n_0=2A/a_k$ i definitionen af
  $\Omega(n^k)$ ses nu, at $p(n)$ tilhører $\Omega(n^k)$.
\qed\end{proof}


\begin{exerc} 
  Sandt eller falskt? 
  (a)~$n^2 + 10^6 n \in O(n^2)$; 
  (b)~$n \log n
\in O(n)$; (c)~$n \log n \in \Omega(n)$; (d)~$\log n \in o(n)$. 
\end{exerc}

Asymptotisk notation er så udbredt i algoritmeanalysen, at man af bekvemmelighedsgrunde ofte anvender den præcise notation på en mere fleksibel måde.
Ikke mindst benytter man ofte betegnelser for funktionsmængder (fx $O(n^2)$) som om de selv var en enkelt funktion. 
Især plejer man at skrive $h(n)= O(f(n))$ i stedet for $h(n)\in O(f(n))$ og $O(h(n))= O(f(n))$ is stedet for $O(h(n)) \subseteq O(f(n))$, som fx:
\[   3n^2 + 7n = O(n^2) = O(n^3) \,. \]
Følger af »ligninger« med $O$-notation skal strengt taget opfattes som udsagn om tilhørsforhold og mængdeinklusioner, og de giver kun mening læst fra venstre til højre.

For en funktion  $h(n)$, funktionsmængder $F$ og $G$ og en operator $\diamond$ (fx $+$, $\cdot$ eller $/$) lad $F\diamond G$ være en forkortelse for $\{\,f(n) \diamond g(n)\colon f(n)\in F, g(n)\in G\,\}$, og $h(n)\diamond F$ være en forkortelse for $\{h(n)\} \diamond F$. 
Med denne konvention betegner $f(n)+o(f(n))$ altså mængden af funktioner $f(n) + g(n)$ med den egenskab,  at $g(n)$ vokster stærkt langsommere end als $f(n)$, dvs. at kvotienten $(f(n) + g(n))/f(n)$ går mod $1$ for $n\to\infty$.
Ækvivalent skrives $(1+o(1))f(n)$.
Vi bruger denne notation, når vi vil understrege $f(n)$s rolle som
»\emph{førende term}«,
\index{førende term|textbf}
i forhold til hvilken »\emph{termer af lavere orden}«
\index{term af lavere orden|textbf}
kan ignoreres.

\begin{lemma}[regneregler]
\llabel{lem:ocalculus} 
Der gælder: 
\begin{align*}
cf(n)&=\Theta(f(n))\text{, for hver positive konstant $c$,}\\
f(n)+g(n)&=\Omega(f(n))\,,\\
f(n)+g(n)&=O(f(n))\text{, når }g(n)=O(f(n))\,,\\
O(f(n)) \cdot O(g(n)) &= O(f(n) \cdot g(n))\,.
\end{align*}
\end{lemma} 


\begin{exerc}
Bevis lemma~\lref{lem:ocalculus}.
\end{exerc}



\begin{exerc}
Skærp lemma~\lref{lem:polynomial} ved at vise $p(n)=a_kn^k+o(n^k)$.
\end{exerc}

\begin{exerc} 
Bevis, at der gælder $n^k = o(c^n)$ for heltal $k$ og vilkårligt $c > 1$. 
Hvor står $n^{\log\log n}$ i forhold til $n^k$ og $c^n$?
\end{exerc}

