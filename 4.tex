\chapter{Hakning og associative rækker}
\index{hakning|textbf}
\index{række!associativ}

\renewcommand{\labelprefix}{ch:hash}
\llabel{}
\vspace*{-3.5cm}
\begin{flushright}
\includegraphics[width=7cm]{img/FleischwolfKlein.pdf}
\end{flushright}

I dette kapitel skal vi betragte en robust og effektiv implementation af en datastruktur til implementation af såkaldte ordbøger eller associative rækker.
Denne struktur kaldes en \emph{hakketabel}.
\footnote{Foroven: En kødkværn.}
\index{ordbog@ordog|textbf}
\index{associativ række|textbf}
\index{række associativ|textbf}
Den anden grundlæggende mulighed er en datastruktur baseret på træer, som vi skal se nærmere på i kap.~\ref{ch:search:}.
Man skal forestille sig associativ række som en række med potentielt uendelig -- eller i det mindste meget stor -- indeksmængde, hvoraf dog kun nogle få indeks faktisk er i brug.
For eksempel udgøres de mulige indeks af mængden af strenge, og de faktisk benyttede indeks er alle navne i et konkret C-program.
Eller de mulige indeks kunne være alle muligheder for at placere skakfigurer på et skakbræt,
\index{skak}
og de faktisk benyttede index er de stillinger, som optræder i et konkret parti.
Associative rækker har mange anvendelser.
Oversættere
\index{oversætter}
benytter dem til deres symboltabeller
\index{oversætter!symboltabel}
som knytter navne i det program, der skal oversættes, til deres tilhørende information.
Programmer, som løser problemer inden for kombinatorisk søging,
\index{kombinatorisk søgning}
benytter dem ofte for at afgøre, om en bestem konfiguration allere er blevet undersøgt.
I skakeksemplet, hvor samme stilling jo kan nås ad forskellige træk, bør programmet jo kun vurdere hver enkelt stilling én gang.
Løsningen består i at gemme vurderingen i en associativ række, indekseret af stillingerne.
Inden for relationelle databaser gemmer en af de mest benyttede implementationer af \emph{join}-operationen
\index{database!join@\emph{join}} 
 de indgående relationer i en associativ række.
Skriptsprog som 
\index{AWK}
\index{Perl} 
AWK \cite{AWK88} og Perl \cite{WCO00} bruger associative rækker som deres primære datastruktur.
I alle de nævnte eksempler er den associative række normalt implementeret som hakketabel.
I opgaverne til dette afsnit betragtes yderligere anvendelser af associative rækker.

\newcommand{\Key}{\Id{Nøgle}}
\newcommand{\key}{\Id{nøgle}}
\newcommand{\find}{\Id{find}}
\newcommand{\remove}{\Id{fjern}}

Formelt består en associativ række $S$ af en mængde af \emph{indgange}.
Til hver indgang $e$ knyttes en nøgle
\index{nzgle@nøgle} 
$\key(e) \in \Key$, hvor $\Key$ er mængden af mulige nøgler. 
Vi skal sikre, at nøglerne i $S$ er entydige, dvs. at forskellige indgange i den associative rækker har forskellige nøgler.
Associative rækker stiller følgende operationer til rådighed:%
\index{række!associativ!indsæt@\Id{indsæt}|textbf}
\index{række!associativ!fjern@\Id{fjern}|textbf}
\index{række!associativ!find@\find|textbf}
\index{række!associativ!opbyg@\Id{opbyg}|textbf}

\medskip
\noindent
\begin{tabular}{lp{8cm}}
  \toprule
  $S.\Id{opbyg}(\set{e_1,\ldots,e_n})$ & $S\Is\set{e_1,\ldots,e_n}$.\\
$S.\Id{indsæt}(e : \Id{Element})$ &
Hvis der findes $e'\in S$ med $\key(e')=\key(e)$, så $S\Is (S\setminus\set{e'})\cup\set{e}$.
 Ellers  $S\Is S\cup\set{e}$.\\
$S.\remove(x : \Key)$ &
Hvis der findes $e\in S$ med $\key(e)=x$, så $S\Is S\setminus\set{e}$.\\
$S.\find(x : \Key)$ &
  Hvis der findes $e\in S$ med $key(e) = x$ så returner $e$. Ellers returner $\bot$.\\\bottomrule
\end{tabular}
\medskip

Når kun operationerne \Id{opbyg} og $\find$ er i brug, taler man om en \emph{statisk}, ellers en \emph{dynamisk} datastruktur.
Ved udførelsen af operationen $\Id{opbyg}(\set{e_1,\ldots,e_n})$ skal man sikre, at indgangene $e_1,\ldots,e_n$  har parvist forskellige nøgler. 
Operationen $\Id{indsæt}(e)$ opfylder to formål:
Når $x=\key(e)$ ikke forekommer i  $S$, så tilføjes $e$.
Ellers erstattes indgangen med nøgle $x$ af $e$.
Det svarer til at \emph{opdatere} den indgang i den associative række, som er givet af indeks $x$.
Læg mærke til, at operationerne $\find$ hhv. $\Id{indsæt}$ og $\remove$ i alt væsentligt realiserer læsning og skrivning med »vilkårlig adgang«
\index{adgang!vilkårlig}
\index{vilkårlig adgang}
til en række -- heraf navnet associativ række.

Vi går ud fra, at der ved siden af de nævnte operationer findes en måde at gennemgå samtlige indgange i $S$.
\index{række!associativ!forAlle@\emph{forAlle}|textbf} 
Såden en \Id{forAlle}-operation er normalt ganske enkelt at realisere; vi nøjes derfor med at behandle den i øvelsesopgaverne.

Mængden $\Key$ indeholder samtlige potentielle rækkeindeks, hvorimod mængden
$\{\,\key(e)\colon e\in S\}$ kun indeholder de indeks, som er i brug på et vist tidspunkt.
I hele kaptitlet skal $n$ betegne størrelsen af $S$ og $N$ betegne størrelsen af $\Key$.
I typiske anvendelser af associative rækker er $N$ så kæmpestor, at brugen af rækker af størrelse $N$ er udelukket.
I stedet vil vi kigge på løsninger, hvis pladsbehov er $O(n)$.

\bigskip
Den grundlæggende idé for at implementere associative rækker som hakketabeller er ganske enkel.
En \emph{hakkefunktion}\index{hakkefunktion} $h$ afbilder mængden $\Key$ af mulige rækkeindeks til et lille område $\{0,\ldots,m-1\}$ af heltallene.
Desuden findes en række $t$ indekseret af $\{0,\ldots,m-1\}$, kaldt \emph{hakketabellen}.
\index{hakketabel|siehe{hakning}}
For at holde pladsforbruget nede, bør $m$ omtrent svare til antallet af elementer i $S$.
Um den Platzbedarf niedrig zu halten, sollte $m$ ungefähr der Anzahl der Elemente von $S$ entsprechen.

Hakkefunktionen afbilder hver indgang $e$ til en \emph{hakkeværdi} $h(\key(e))$.
For at forenkle notationen skriver vi $h(e)$ i stedet for $h(\Id{nøgle}(e))$ for hakkeværdien af $e$.
I det indledende bibliotekseksempel afbilder $h$ hvert studienummer til sine sidste to cifre.
Ideelt skulle vi kunne gemme indgangen $e$ på position $t[h(e)]$ i tabllen $t$.
Når det virker, kan hver af de  tre operationer \Id{indsæt}, \remove{} og \find{} gennemføres i konstant tid.
\footnote{%
For at være helt nøjagtige burde vi også tage hensyn til omkostningerne ved at evaluere hakkefunktionen og flytte indgange. 
For at holde notationen enkel antager vi i dette kapitel, at disse operationer samlet tager konstant tid.} 

Desværre vil det dog ikke altid være muligt, at gemme $e$ i $t[h(e)]$, fordi forskellige indgange kan \emph{kollidere},
\index{kollision}
dvs. have samme hakkeværdi.
Bibliotekseksemplet indeholder allerede en af de mulige udveje:
der tillades at stille de bestilte bøger til forskellige brugere i samme hylde.
Det fører selvfølgeligt til, at man skal gennemse hele hylden, for at finde en bestemt bestilling. 
En mere generel form for denne idé fører til
\emph{separat kobling}.
\index{hakning!separat kobling}
På hver plads i  tabellen gemmes en \emph{mængde} af indgange, hvor en mængde implementeres som enkelthægtet liste.
I afsnit~\lref{s:chaining} analyseres separat kobling, hvor vi gør nogle ret optimistiske (og derfor urealistiske) antagelser om hakkefunktionens opførsel.
I denne model opnår vi konstant forventet tid for alle tre ordbogsoperationer.

Vi kan holde hakketabellens struktur (men ikke dens analyse!) enkel ved at gå tilbage til den oprindelige idé om at gemme alle indgange i selve tabellen.
Når en ny indgang $e$ skal tilføjes, men tabelpladsen $t[h(e)]$ allerede er optaget, gennemgår man de følgende tabelpladser, indtil man finder en ledig plads.
I bibliotekseksemplet skulle det svare til, at der højst kan stå én bog på hver hylde.
Bibliotekaren ville så benytte de tilstødende hyldepladser for at placere de bøger, som har fået tildelt den samme plads af hakkefunktionen.
Denne tilgang, som kaldes 
\emph{hakning med åben adressering og lineær probering},
\index{hakning!lineær probering}
er nærmere beskrevet i afsnit~\lref{s:probing}.

Begrebet »hakning« skal minde om at skære, bikse, sønderdele, eller kværne.
\footnote{O. a.: Det engelske verbum »\emph{to hash}« går ligesom det danske »at hakke« tilbage til det franske  »\emph{hacher}«.
  Den engelske madret »\emph{hash}« -- biskemad gjort på hakkede kød, gerne gårdagens levninger -- findes i det danske køkken som »hachis«, og serveres her med brun sovs.
  Læg mærke til at vi på dansk benytter ordet »hash«  for at betegne et udbredt rusmiddel, navngivet efter hampplantens arabiske navn »\emph{hashish}«.
  Det danske »hash« har altså hverken en etymologisk eller kulinarisk sammenhæng med den primære betydning af det engelske »\emph{hash}«.
  }
Det er netop hvad en hakkefunktion gør med nøglerne.
Når nøglerne fx er strenge, kan hakkefunktionen skære dem i dele af fast længde, fortolke hver del som heltal, og beregne én værdi, \emph{hakkeværdien}, ud fra den resulterende talfølge.
En god hakkefunktion skaber uorden og undgår derved kollisioner.

\begin{exerc}
  Givet en mængde $M$ af heltalspar, definieret ved et binær relation $R_M$.
  Anvend en associativ række for at afgøre, om $R_M$ er symmetrisk.
\end{exerc}

\begin{exerc}\llabel{ex:count}
  Skriv et program, som læser et tekstarkiv og udgiver de 100 mest hyppige ord ved brug af den associativ  række.
\end{exerc}

\begin{exerc}[Et regnskabssystem]\llabel{ex:akamai} 
  Udvikl en algorithme baseret på en associativ række før følgende problem:
  Givet et stor database af 3-tupler (transaktion, pris, kundenummer).
  For hver kunde skal den totale regningssumme udregnes.
  Algoritmen skal bruge et lineært antal operationer på den associative række.
\end{exerc}

\begin{exerc}[Gennemløb af hakketabel]
  Beskriv en implementation af \Id{foralle}-operationen, både for separat kobling og for hakning med åben adressering og lineær probering.
  Hvad er kørselstiden for din løsning?
\end{exerc}

%%%%%%%%%%%%%%%%%%%%%%%%%%%%%%%%%%%%%%%%%%%%%%%%%%%%%%%%%%%%%%%%%%%%%%
\section{Separat kobling}
\llabel{s:chaining}
\index{hakning!separat kobling|textbf}
\index{separat kobling|siehe{hakning}}

Hægtet hakning gør brug af en række (kaldt »tabellen«) $t[0..m-1]$ af følger,  som kan organiseres som enkelthægtede lister \index{liste} (se figur.~\lref{fig:chain}).
Ideen er, at indgang $e$ befinder sig i følgen $t[h(e)]$.
Operationerne for den associative række lader sig implementere let:
For at \emph{finde} en indgang med nøgle $x$
\index{hakning!separat kobling!find@\Id{find}|textbf}
gennemgår vi følgen $t[h(x)]$; 
hvis vi herved støder på en indgang $e$ med $\Id{nøgle}(e)=x$, returnerer vi den, i modsat fald returnerer vi værdien $\bot$.
For at \emph{fjerne} en indgang med nøgle $x$
\index{hakning!separat kobling!fjern@\Id{fjern}|textbf},
gennemgår vi 
% TODO gennemse, gennemløbe
ligeledes følgen $t[h(x)]$;
hvis vi herved støder på en indgang $e$ med $\Id{nøgle}(e)=x$, fjerner vi den fra følgen.
For at \emph{indsætte} en indgang $e$
\index{hakning!separat kobling!indsæt@\Id{indsæt}|textbf}
gennemgår vi ligeledes følgen;
hvis vi herved støder på en indgang $e'$ med $\Id{nøgle}(e')=\Id{nøgle}(e)=x$,
erstattes denne med $e$; ellers tilføjes $e$ på en vilkårlig plads i listen.
Operationen $\Id{opbyg}(\set{e_1,\ldots,e_n})$ realiseres som $n$ indsætninger, hvor vi kan bortse fra at gennemgå den pågældende følge.
Derfor har hver operation kørselstid $O(n)$. 
   
\begin{figure}
  \begin{tikzpicture}[scale = .35, yscale = -1]
    \tikzstyle{list} = [anchor = west];
    \tikzstyle{index} = [anchor =  east];
    \node (a) [index] at (1, 0) {a  0};
    \node (b) [index] at (1, 1) {b  1};
    \node (c) [index] at (1, 2) {c  2};
    \node (d) [index] at (1, 3) {d  3};
    \node (e) [index] at (1, 4) {e  4};
    \node (f) [index] at (1, 5) {f  5};
    \node (a) [index] at (1, 6) {g  6};
    \node (a) [index] at (1, 7) {h  7};
    \node (a) [index] at (1, 8) {i  8};
    \node (a) [index] at (1, 9) {j  9};
    \node (a) [index] at (1,10) {k 10};
    \node (b) [index] at (1,11) {l 11};
    \node (c) [index] at (1,12) {m 12};
    \node (d) [index] at (1,13) {n 13};
    \node (e) [index] at (1,14) {o 14};
    \node (f) [index] at (1,15) {p 15};
    \node (a) [index] at (1,16) {q 16};
    \node (a) [index] at (1,17) {r 17};
    \node (a) [index] at (1,18) {s 18};
    \node (a) [index] at (1,19) {t 19};
    \node (a) [index] at (1,20) {u 20};
    \node (a) [index] at (1,21) {v 21};
    \node (a) [index] at (1,22) {w 22};
    \node (a) [index] at (1,23) {x 23};
    \node (a) [index] at (1,24) {y 24};
    \node (a) [index] at (1,25) {z 25};
    \draw [callout] (2, -.5) -- (2,25.5);
    \foreach \y in {0,...,26} \draw[callout] (2,\y) ++(0,-.5) -- +(1,0);
    \draw [callout] (12,-.5) -- (12,25.5);
    \foreach \y in {0,...,26} \draw[callout] (12,\y) ++(0,-.5) -- +(1,0);
    \draw [callout] (22,-.5) -- (22,25.5);
    \foreach \y in {0,...,26} \draw[callout] (22,\y) ++(0,-.5) -- +(1,0);
    \node at (2,  4) [list] {$\langle \text{axe}, \text{dice}, \text{cube}\rangle$};
    \node at (2, 15) [list] {$\langle \text{chop}, \text{clip}, \text{lop}\rangle$};
    \node at (2,  7) [list] {$\langle \text{hash}\rangle$};
    \node at (2, 10) [list] {$\langle \text{hack}\rangle$};
    \node at (2, 11) [list] {$\langle \text{fell}\rangle$};
    \node at (12,  4) [list] {$\langle \text{axe}, \text{dice}, \text{cube}\rangle$};
    \node at (12, 15) [list] {$\langle \text{chop}, \text{clip}, \text{lop}\rangle$};
    \node at (12,  7) [list] {$\langle \text{slash}, \text{hash}\rangle$};
    \node at (12, 10) [list] {$\langle \text{hack}\rangle$};
    \node at (12, 11) [list] {$\langle \text{fell}\rangle$};
    \node at (22,  4) [list] {$\langle \text{axe}, \text{dice}, \text{cube}\rangle$};
    \node at (22, 15) [list] {$\langle \text{chop}, \text{lop}\rangle$};
    \node at (22,  7) [list] {$\langle \text{hash}\rangle$};
    \node at (22, 10) [list] {$\langle \text{hack}\rangle$};
    \node at (22, 11) [list] {$\langle \text{fell}\rangle$};
    \node at ( 2.5, -1) {$t$};
    \node at (12.5, -1) {$t$};
    \node at (22.5, -1) {$t$};
    \node at ( 7.5,  -1) {\Id{indsæt}$($slash$)$};
    \draw [->] (5,0) -- (10,0);
    \node at (17.5,  -1) {\Id{fjern}$($clip$)$};
    \draw [->] (15,0) -- (20,0);
  \end{tikzpicture}
\caption{
  \llabel{fig:chain} 
  Separat kobling.
  Vi benytter en række $t$ af følger.
  I eksemplet indeholder datastrukturen en mængde af ord (korte engelske synonymer af »\emph{hash}«), som er gemt med en hakkefunktion, som afbilder et ord til dets sidste bogstavs position i alfabetet, dvs. til en værdi i  $\{0,\ldots, 25\}$.
  Åbenbart er dette ikke nogen særlig god hakkefunktion.}    
\end{figure}

Pladsbehovet for hele datastrukturen er $O(n+m)$.
For at søge efter, indsætte eller fjerne en indang med nøgle $x$ , skal vi gennemsøge følgen $t[h(x)]$.
I værste fald, fx når \Id{find} søger efter en ikke-eksisterende nøgle, skal hele følgen gennemgås.
Når det går rigtig galt og hakkefunktionen afbilder samtlige indgange til den samme tabelindgang, er kørselstiden $\Theta(n)$. 
I værste fald opfører separat kobling sig altså ikke bedre end lineære lister.

Findes der hakkefunktioner, som leder til, at alle følger er korte?
Svaret er klart »nej«.
En hakkefunktion afbilder mængden af nøgler til værdimængden $\{0,\ldots, m-1\}$, hvorfor der for hver hakkefunktion må findes $N/m$ nøgler, der alle havner på samme plads i tabellen.
I de fleste anvendelser gælder $n < N/m$, og i så fald kan separat kobling degenerere til lineær søgning.
Vi betragter her tre muligheder for at håndtere denne opførsel.
Den første tilgang er gennemsnitsanalyse,
\index{algoritmeanalyse!gennemsnit}, 
som vi skal beskæftige os med i dette afnit.
Den anden tilgang består i at benytte randomisering og vælge hakkefunktionen tilfældigt fra en mængde af hakkefunktioner.
Herom nærmere i dette og næste afsnit.
Den tredje tilgang består i at ændre algoritmen.
For eksempel kan vi prøve at tilpasse hakkefunktionen til mængden af nøgler, som skal gemmes i øjeblikket.
Dette skal vi undersøge i afsnit~\lref{s:perfect} og se, at det leder til god opførsel i værste fald.

Lad $H$ være mængden af alle funktioner fra $\Key$ til $\{0,\ldots, m-1\}$.
Vi antager, at hakkefuktionen $h$ er valgt tilfældigt fra $H$.
\footnote{%
  Denne antagelse er helt urealistisk.
   Der findes $m^N$ mange funktioner in $H$, derfor skal der bruges
   $N \log m$ bits alene for at beskrive en funktion i $H$.
   Det strider mod målet at reducere datastrukturens pladsbehov fra $N$ til $n$.
   }
Vi vil nu vise, at der for hver fast mængde $S$ af $n$ indgange gælder, at operationerne \Id{indsæt}, \Id{fjern} og \Id{find} tager forventet tid $O(1+n/m)$.
\index{algoritmenanalyse!gennemsnit}

\index{hakning!urealistisk analyse}
\begin{thm}\llabel{thm:chaining}
Hvis separat kobling bruges til at gemme $n$ indgange i en hakketabel med $m$ tabelpladser, og hakkefunktionen er valgt tilfældigt fra $H$,
så har operationerne \Id{indsæt}, \Id{fjern} og \Id{find} forventet kørselstid $O(1+n/m)$. 
\end{thm}
% TODO made clear what the expectation is over
\begin{proof} 
  I beviset har vi brug for begreberne »stokastisk variabel« og »forventningsværdi« fra sandsynlighedsregningen, samt at forventningsværdien er lineær, 
\index{linearitet af forventningsværdi}
som beskrevet i afsnit~\ref{app:notation:s:prob}. 
Lad $h$ være en tilfældigt valgt funktion fra $H$.
  Vi betragter kørselstiden for en operation \Id{indsæt}, \Id{fjern} eller \Id{find} for en fast nøgle $x$. 
  Hver af disse operationer behøver konstant tid samt tiden for at gennemsøge følgen $t[h(x)]$.
  De forventede kørselstid er derfor $O(1+E[X])$, hvor den stokastiske variabel $X$ angiver længden af følgen $t[h(x)]$.
  Lad $S$ med $|S|=n$ være mængden af indgange, som er gemt i tabellen. 
  Betragt for hver indgang $e\in S$ den stokastiske \emph{indikatorvariabel} $X_e$, som angiver, om $h$ afbilder $e$ til samme tabelplads som $x$:
  \[ X_e=\begin{cases}
    1\,,&\text{hvis } h(e)= h(x)\,;\\
    0\,,& \text{ellers}\,.
  \end{cases}\]
  Der er to tilfælde.
  Hvis $S$ ikke indeholder nogen indgang med nøglen $x$, så er 
  $X=\sum_{ e \in S} X_e$.
 Hvis derimod $\Id{nøgle}(e_0)=x$ for $e_0\in S$, så er $X=1+\sum_{ e \in S\setminus\{e_0\}} X_e$. 
 I det første tilfælde følger ved linearitet af forventningsværdien, at
  \[ E[X]=
  E\left[\sum_{e \in S} X_e\right]=
       \sum_{e \in S} E[X_e]=
       \sum_{e \in S} \Pr(X_e=1)\,. \]
   En tilfældig valgt funktion afbilder $e$ med samme sandsynlighed til hver af de $m$ tabelpladser, uafhængigt af $h(x)$. 
   Derfor har vi $\Pr(X_e=1)=1/m$, hvilket medfører $E[X] = n/m$.
  I det andet tilfælde, hvor nøgle $x$ findes ikke i $S$, får vi på samme måde $E[X] = 1 + (n-1)/m < 1 + n/m$.  
  I begge tilfælde er sætningen bevist.
\end{proof}  

Vi kan opnå lineært pladsbehov og forventet konstant kørselstid for alle tre operationer ved at sikre, at der gælder $m=\Theta(n)$. 
Til dette formål kan vi lade tabellens større tilpasse sig $n$, som beskrevet i afsnit~\ref{ch:sequence:s:array} om ubegrænsede rækker.

\begin{exerc}[Ubegrænsede hakketabeller]\llabel{ex:dynamic}
  Forklar, hvordan man med separat kobling kan sikre, at der altid gælder $m =  \Theta(n)$.
  Du kan gå ud fra en hakkefunktion $h'\colon\Id{Nøgle}\rightarrow\NN$.
  Sæt $h(x)=h'(x)\bmod m$ og tilpas tabelstørrelsen til $n$. 
\end{exerc}


\begin{exerc}[Ubenyttet lagerplads]\llabel{ex:waste}
  Separat kobling spilder plads i den forstand, at nogle tabelpladser forbliver ubenyttede.
  Beregn det forventede antal tomme tabelpladser under antagelse af, at hakkefunktionen er tilfældig.
  \emph{Vink}: 
  Definer de stokatiske indikatorvariabler $Y_0$, $\ldots$, $Y_{m-1}$,
med  $Y_i=1$ hvis $t[i]$ er tom.  
\end{exerc}


\begin{exerc}[Gennemsnitstilfælde]
  \llabel{ex:average}
  \index{hakning!separat kobling!gennemsnitstilfælde} 
  Antag, at hakkefunktionen afbilder mængden af mulige nøgler uniformt over tabellen, dvs. $\left|\{\, x\in\Key\colon h(x)=i\}\right|\leq \ceil{N/m}$ for alle $i\in\{0,\ldots,m-1\}$.
  Antag desuden, at at vi har gemt en tilfældig mængde af nøgler i tabellen, dvs. at mængden af nøgler i $S$ er en tilfældig $n$-delmængde af $\Key$.
  Vis, at der gælder for hver tabelplads $i$, at det forventede antal indgange, der afbildes til $i$, er højst $\ceil{N/m} \cdot n/N\approx n/m$.
\end{exerc}

\section{Universalhakning}
\llabel{s:universal}
\index{hakning!universal-}

\emph{Udeladt}

\section{Hakning med lineær probering}
\llabel{s:probing}
\index{hakning!lineær probering}

Separat kobling kaldes også \emph{åben hakning},
\index{hakning!åben|textbf}
\index{hakning!lukket|textbf}
fordi der til kollisionshåndtering bruges lagerplads uden for den egentlige tabel, fx i form af lister.
Derimod taler man om \emph{lukket} hakning, hvis samtlige ingange gemmes på selve tabelpladserne.
Ved at slække kravet om at gemme indgang $e$ på plads $t[h(e)]$, opstår fleksibilitet -- også kaldt »åben adressering« --, der gør det muligt helt at undgå at bruge en sekundær datastruktur.
Der findes mange undersøgelser af metoder for \emph{åben adressering} \cite{Pet57}\index{Peterson, W. W.}, jf.~\cite[kap.~3.3]{GonBY1991}.
Vi vil nøjes med at se på den enkleste metode, som kaldes \emph{lineær probering}.
Ubenyttede tabelpladser frå en særlig indgang $\bot$.
En indgang $e$ findes enten i $t[h(e)]$ eller længere til højre i tabellen, som vi tænker os vandret.
Vi vil dog kun holde os fra den foretrukne plads $t[h(e)]$, når det er absolut nødvendigt:
når $e$ er gemt i $t[i]$ med $i>h(e)$, så skal alle plaserne fra  $h(e)$ til og med $i-1$ være optaget af andre indgange.
(Denne \emph{invariant}
\index{invariant!datastruktur-} 
% TOTO: orig: schleifen-, should be datenstruktur-
opretholdes af operationerne.)

\begin{figure}[t]
  \begin{tikzpicture}
    \node at (0,+3.8) {\Id{indsæt} axe, chop, clip, cube, dice, fell, hack, hash, lop, slash};
    \tikzstyle{e} = [draw = none, font =\upshape]
    \matrix (1) [matrix of nodes,
    column sep=-\pgflinewidth,
    row sep= 3 pt,
    nodes = { rectangle, draw, font = \footnotesize\itshape,
    minimum width = .9cm, text height = 1.5ex, text depth=.25ex },
    ]
    {
      \node [e]{an}; & \node [e]{bo}; & \node [e]{cp}; & \node [e]{dq}; & \node [e]{er}; & \node [e]{fs}; & \node [e]{gt}; & \node [e]{hu}; & \node [e]{iv}; & \node [e]{jw}; &\node [e]{kx}; & \node [e]{ly}; & \node [e]{mx}; \\
      \node [e] {0}; & \node [e] {1}; & \node [e] {2}; & \node [e] {3}; & \node [e] {4}; & \node [e] {5}; & \node [e] {6}; & \node [e] {7}; & \node [e] {8}; & \node [e] {9}; & \node [e] {10}; & \node [e] {11}; & \node [e] {12}; \\
     $\bot$ & $\bot$ & $\bot$ & $\bot$ &   axe & $\bot$ & $\bot$ & $\bot$ & $\bot$ & $\bot$ & $\bot$ & $\bot$ & $\bot$ \\
     $\bot$ & $\bot$ & chop   & $\bot$ &   axe & $\bot$ & $\bot$ & $\bot$ & $\bot$ & $\bot$ & $\bot$ & $\bot$ & $\bot$ \\
     $\bot$ & $\bot$ & chop   & clip   &   axe & $\bot$ & $\bot$ & $\bot$ & $\bot$ & $\bot$ & $\bot$ & $\bot$ & $\bot$ \\
     $\bot$ & $\bot$ & chop   & clip   &   axe & cube   & $\bot$ & $\bot$ & $\bot$ & $\bot$ & $\bot$ & $\bot$ & $\bot$ \\
     $\bot$ & $\bot$ & chop   & clip   &   axe & cube   & dice   & $\bot$ & $\bot$ & $\bot$ & $\bot$ & $\bot$ & $\bot$ \\
     $\bot$ & $\bot$ & chop   & clip   &   axe & cube   & dice   & $\bot$ & $\bot$ & $\bot$ & $\bot$ & fell   & $\bot$ \\
     $\bot$ & $\bot$ & chop   & clip   &   axe & cube   & dice   & $\bot$ & $\bot$ & $\bot$ & hack   & fell   & $\bot$ \\
     $\bot$ & $\bot$ & chop   & clip   &   axe & cube   & dice   & hash   & $\bot$ & $\bot$ & hack   & fell   & $\bot$ \\
     $\bot$ & $\bot$ & chop   & clip   &   axe & cube   & dice   & hash   & lop    & $\bot$ & hack   & fell   & $\bot$ \\
     $\bot$ & $\bot$ & chop   & clip   &   axe & cube   & dice   & hash   & lop    & slash  & hack   & fell   & $\bot$ \\
   };
 \scoped[on background layer]
    {
    \node[fill=myblue, fit=(1-3-5)(1-3-5), inner sep = 0pt ]   {};
    \node[fill=myblue, fit=(1-4-3)(1-4-3), inner sep = 0pt ]   {};
    \node[fill=myblue, fit=(1-5-3)(1-5-4), inner sep = 0pt ]   {};
    \node[fill=myblue, fit=(1-6-5)(1-6-6), inner sep = 0pt ]   {};
    \node[fill=myblue, fit=(1-7-5)(1-7-7), inner sep = 0pt ]   {};
    \node[fill=myblue, fit=(1-8-12)(1-8-12), inner sep = 0pt ]   {};
    \node[fill=myblue, fit=(1-9-11)(1-9-11), inner sep = 0pt ]   {};
    \node[fill=myblue, fit=(1-10-8)(1-10-8)   , inner sep = 0pt ]   {};
    \node[fill=myblue, fit=(1-11-3)(1-11-9)   , inner sep = 0pt ]   {};
    \node[fill=myblue, fit=(1-12-8)(1-12-10), inner sep = 0pt ]   {};
    }
    \node at (0,-4.4) {\Id{fjern}(clip)};
    \matrix (2) at (0,-6) [matrix of nodes,
    column sep=-\pgflinewidth,
    row sep= 3 pt,
    nodes = { rectangle, draw, font = \footnotesize\itshape,
    minimum width = .9cm, text height = 1.5ex, text depth=.25ex }]
    {
     $\bot$ & $\bot$ & chop   & clip   &   axe & cube   & dice   & hash   & lop    & slash  & hack   & fell   & $\bot$ \\
     $\bot$ & $\bot$ & chop   & lop   &   axe & cube   & dice   & hash   & lop    & slash  & hack   & fell   & $\bot$ \\
     $\bot$ & $\bot$ & chop   & lop   &   axe & cube   & dice   & hash   & slash    & slash  & hack   & fell   & $\bot$ \\
     $\bot$ & $\bot$ & chop   & lop   &   axe & cube   & dice   & hash   & slash   & $\bot$  & hack   & fell   & $\bot$ \\
   };
 \scoped[on background layer]
    {
    \node[inner sep = 0pt, fill=myblue , fit=(2-1-3)(2-1-4) ]   {};
    \node[inner sep = 0pt, fill=myblue, fit=(2-2-5)(2-2-9) ]   {};
    \node[inner sep = 0pt, fill=myblue, fit=(2-3-10)(2-3-10) ]   {};
    \node[inner sep = 0pt, fill=myblue, fit=(2-4-11)(2-4-13) ]   {};
  }
    \node [fit = (2-1-4), draw, callout, thick, inner sep = 0pt, cross out] {};
    \node [fit = (2-2-9), draw,  callout,thick, inner sep = 0pt, cross out] {};
    \node [fit = (2-3-10), draw,  callout,thick, inner sep = 0pt, cross out] {};
  \end{tikzpicture} 
  % TODO orig: hack in upper part 
  \caption{% 
  \llabel{fig:probing}
  Hakning med lineær probering.
  Tabellen med 13 pladser er tegnet vandret og indeholder engelske synomymer for »(at) hakke«. 
  Hakkefunktion afbilder hvert ords sidste bogstav til tallene $\{0,\ldots,12\}$ som antydet i figuren: 
  »a« og  »n« sendes til $0$, 
  »b« og »o« $1$ osv.
  Ordene indsættes et at gangen i alfabetisk rækkefølge.
  Til sidst fjernes ordet »\emph{clip}«. 
  I figuren er vist, hvordan tilstanden i tabellen forandrer sig efter hver operation.
  De pladser, som inspiceres under udførelsen af den aktuelle operation, er fremhævede.
  }
\end{figure}

Implementationerne af \Id{indsæt}
\index{hakning!lineær probering!indsæt@\Id{indsæt}|textbf}
og \Id{find} er ligetil.
For at udføre $\Id{indsæt}(e)$ gennemsøger vi tabellen sekventielt, begyndende i tabelplads $[h(e)]$, intil vi enten møder en indgang $e'$ med  $\Id{nøgle}(e')=\Id{nøgle}(e)=x$ eller en ledig plads i tabellen.
I det første tilfælde erstattes $e'$ med $e$, i det andet bliver $e$ gemt på den ledige plads, se fx figur~\lref{fig:probing}.
Fremgangsmåden for at søge efter nøgle $x$ med operationen $\Id{find}$)
\index{hakning!lineær probering!find@\Id{find}|textbf}
er lignende:
vi gennemsøger tabellen sekventielt fra plads $t[h(x)]$, indtil vi finder en indgang med nøgle $x$.
Søgningen afbrydes som mislykket, så snart vi når en tabelplads indeholdende $\bot$.
Det lyder jo meget enkelt, men der optræder en lille komplikation.
Hvad sker der, hvis vi under en indsættelse eller søgning når til slutningen af tabel $t$?
En enkel reparation består i at altid afsætte yderligere $m'$ tabelpladser uden for  de $m$ pladser, der omfattes af hakkefunktionen $h$.
For »velartede« hakkefunktioner burde det række med en værdi  for $m'$, som er langt mindre end $m$, for at tage hånd om overløb.
En anden mulighed, som betragtes i opgave~\lref{ex:cyclic} 
(jf. afsnit~\ref{ch:sequence:s:stack}),
består  i at behandle tabllen som en cyklisk række.
Denne variant er mere robust, men også langsommere.

Implementationen af sletteoperationen
\Id{fjern}\index{hakning!lineær probering!fjern@\Id{fjern}|textbf} 
er ikke lig så enkel.
Det er ikke nok at overskive den slettede ingang med $\bot$, fordi dette ville gøre invarianten ugyldig.
(Antag fx, at $h(x) = h(z)$ og $h(y) = h(x) + 1$, og at nøglerne $x$, $y$ og $z$ insættes i den rækkefølge.
Så står $z$ på plads $h(x) + 2$. 
Når ny $y$ bliver fjernet ved at overskrive plads $h(x)+1$ med $\bot$, kan $z$ ikke længere nås i en søgning.)
Der er tre løsninger på dette problem.
Den første mulighed er at forbyde fjernelse.
Den anden er at markere fjernede indgange, men ikke overskrive dem med $\bot$.
Søgninger afbrydes som før, når vi møder $\bot$, men ikke når vi møder en indgang, som er markeret som slettet.
Problemet med denne fremgangsmåde er, at antallet af ikke-tomme tabelpladser (optaget eller markeret) vokser, så søgninger bliver efterhånden langsommere end nødvendigt.
Dette problem kan afhjælpes ved regelmæssigt at genopbygge tabellen og faktisk fjerne de markerede indgange.
Den tredje mulighed er at opretholde invarianten. 
Hvordan skal det ske?
Antag, at vi vil fjerne indgangen på tabelplads $i$.
Vi overskriver pladsen med $\bot$; 
dette kan skabe et »hul«, som forstyrrer søgningen efter indgange med mindre hakkeværdier.
Nu gennemsøger vi tabelpladserne til højre for $i$ sekventielt og leder efter indgagne, som ødelægger invarianten:
Sæt $j$ til $i+1$. Hvis $t[j]=\bot$, er vi færdige.
Ellers lad $f$ være den indgang, som står på plads $t[j]$.
Hvis $h(f)>i$, kan vi øge $j$ med $1$ og gentage.
Hvis $h(f) \le i$, så ville »hullet« på plads $i$ ødeløgge invarianten for indgang $f$, og søgningen efter 
$\Id{nøgle}(f)$ skulle slå fejl.
Vi flytter derfor $f$ til $t[i]$, flytter $\bot$ til $t[j]$, og sætter $i\Is j$.
(Mao. lader vi $f$ og »hullet« bytte plads og gemmer »hullets« nye position på plads $i$.)
Ny begynder vi en ny gennemgang fra $j\Is j+1$.
Figur~\lref{fig:probing} viser et eksempel.


\begin{exerc}[Cyklisk lineær probering]
  \llabel{ex:cyclic}
  \index{hakning!lineær probering!cyklisk}
  Implementer varianten af lineær probering, hvor tabelstørrelsen ikke er $m+m'$, men $m$.
  For forhindre overløb i tabellens højre ende, organiseres gennemsøgningen cyklisk.
(1) Tilpas operationerne \Id{indsæt} og \Id{fjern} ved at erstatte inkrementationen $i \Is i+1$ med  $i \Is (i+1)\bmod m$.
  % TODO inkrementere -> øge? øgningen?
  (2) Specificer udsagnet $\Id{mellem}(i,j,k)$ om tabelpositioner i $\{0,\ldots, m-1\}$, som er  \emph{sand} hvis og kun hvis $i$ befinder sig skarpt mellem $j$ og $k$ ifølge den cykliske opfattelse.
% i<k and i<j<k or i>k and (j>i or j<k)
(3) Omformuler invarianten, så den bruger \Id{mellem}.
(4) Tilpas operationen \Id{remove}.
\end{exerc}


\begin{exerc}[Hakketabel med lineær probering]
  \index{hakning!lineær probering!ubegrænset}
  Implementer en \emph{ubegrænset} hakketabel, altså en hakketabel, hvis størrelse ikke er kendt i forvejen, med lineær probering og universelle hakkefunktioner.
  Vælg en ny hakkefunktion, hver gang tabellen bliver mindre eller større.
  Lad $\alpha$, $\beta$ og $\gamma$ være konstanter med $1<\gamma<\beta<\alpha$, som vi kan vælge.
  Opret en tæller til antallet $n$ af gemte indgange.
  Udvid tabellen til en ny størrelse $m=\beta n$, så snart $n>m/\gamma$.
  Formindsk tabellen til en ny størrelse $m=\beta n$, så snart $n<m/\alpha$.
  Medmindre du vil bruge cyklisk lineær probering som beskrevet i opgave~\lref{ex:cyclic}, sæt tabelstørrelsen til $m'=m/\delta$ for noget $\delta<1$ og vælg en ny hakkefunktion (uden at ændre $m$ eller $m'$), når tabellen er ved at løbe over i højre side.
  (Bemærkning: Der kan opstå problemer, når man forsøger at kombinere enkle universelle hakkeklasser med lineær probering, jf. afsnit~\lref{s:further}.)
\end{exerc}

\section{Sammenligning af separat kobling og lineær probering}
\llabel{s:versus}
\index{ingeniørssaspekter}%
% TODO algorihtm engineering?
Vi har nu set to forskellige fremgangsmåder for realisering af hakketabeller.
Hvilken er bedre?
Det spørgsmål kan vi ikke besvare på baggrund af en teoretisk analyse, fordi svaret afhænger af anvendelsessituationen og af mange tekniske parametre.
Vi skal her nævne nogle kvalitative askpekter og skitsere nogle eksperimentelle resultater.

En af fordelene ved separat kobling er »referenceintegritet«,
\index{referenceintegritet}
\index{integritet!reference-|siehe{referenceintegritet}}
hvilket betyder, at en indangs lagerposition ikke ændrer sig, efter den er blevet indrettet.
Senere søgninger kommer altid til at finde indgangen på præcis samme plads i lageret.
Derfor kan man arbejde med pegere på listeknuder.
I modsætning hertil vil fjernelser ved lineær probering flytte indgange; absolutte referencer eller pegere på flyttede indgange bliver herved ugyldige.

Fordelen ved lineær probering er, at hver adgang til tabellen arbejder med et sammenhængende lagersegment.
Moderne processorers lagerarkitektur er optimeret for dette adgangsmønster, mens de er langsomme til at følge pegere ved gennemløb af en lineær liste, som ikke får plads i gemmelageret.
Et problem ved lineær probering er, at søgetiderne vokser hurtigt, når antal indgange nærmer sig tabelstørrelsen, og at tabellen kan flyde over.
Ved separat kobling forekommer ingen sådanne abrupte overgange -- de forventede søgetider forbliver små, selv når nøgleantallet overstiger tabelstørrelsen.
På den anden side skal separat kobling bruge lagerplads til pegerne, som man ved lineær probering ville kunne have brugt til en større tabel.
En retfærdig sammenligning må derfor tage hensyn til hele lagerforbruget og ikke bare tabelstørrelsen.

Vi har implementeret begge metoder og gennemført udførlige eksperimenter.
Resultatet var, at begge metoder opfører sig nogenlunde lige godt, hvis de har samme lagerplads til rådighed.
Forskellene er så små, at konklusionerne afhænger af implementationsdetaljer, specielle egenskaber ved oversætteren, operativsystemet og den benyttede regner.
Derfor nævner vi her ingen eksakte resultater.

Vi fandt dog implementationen af separat kobling vanskeligere.
For at kunne konkurrere med lineær probering, skal alle forbedringer i afsnit~\lref{s:implementation} gennemføres.
Separat kobling bliver meget langsommere, hvis man sløser med implementationen eller organiserer lagerstyringen dårligt.

Med hensyn til matematiske argumenter skal man nævne, at udsagn i stil med sætning~\lref{thm:universal} \emph{kun} gælder for kombinationen af separat kobling og $c$-universelle klasser og ikke for lineær probering med $c$-universelle klasser.
For at opnå lineær søgetid har lineær probering brug for hakkefunktioner med stærkere sandsynlighedsegenskaber, eller antage fuld tilfældighed.
(Se hertil afsnit~\lref{s:further}.) 

\section{Perfekt hakning}
\llabel{s:perfect}
\index{hakning!perfekt}  

\emph{Udeladt}

\section{Implementationsaspekter}
\llabel{s:implementation}

\emph{Udeladt}

\section{Historiske anmærkninger og yderligere resultater}
\llabel{s:further}

\emph{Udeladt}
