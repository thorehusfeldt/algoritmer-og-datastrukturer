\chapter{Korteste veje}
\index{korteste veje|textbf}

\renewcommand{\labelprefix}{ch:spath}
\llabel{}
\vspace*{-3.7cm}
%\mbox{}\hspace{\fill}\includegraphics[width=4cm]{\masterdir/ShortPath/figs/labyrinth.ps} 
%\mbox{}\hspace{\fill}\includegraphics[width=4cm]{\masterdir/ShortPath/figs/Faeden.eps} 
%\vspace*{-1.5cm}
\begin{center}
\includegraphics[width=0.6\textwidth]{img/campus-plan3.eps}%
\includegraphics[width=0.4\textwidth]{img/faeden-neu.eps}
% für Fadenknaeuel:
%{\tiny Copyright Universit"at Karlsruhe (TH), Institut f"ur Photogrammetrie und Fernerkundung}
\end{center}
%
%\begin{flushleft}
%\includegraphics[width=\textwidth]{\masterdir/ShortPath/figs/campus-plan.eps}
%\end{flushleft}

\aufmacher{\noindent 
Problemet at finde en korteste, hurtigste eller billigste vej i en netværk er allestedsnærværende.
Hver af os løser det dagligt.
Er man på plads $s$ og ønsker at være på plads $t$, spørger man om vej for at komme fra $s$ til $t$ på den bedste måde.
Brandvæsnet ønsker at kende den hurtigste vej fra brandstationen $s$ til samtlige pladser  $t$ i hele byen -- varianten af korteste-veje-problemet »med samme udgangspunkt« eller »med en enkelt kilde«.
\index{korteste veje!samme udgangspunkt}
Sommetider ønsker man til og med en fuldstændig tabel over alle parvise afstande -- denne variant hedder »korteste veje for alle par«.
\index{korteste veje!alle par}
Et eksempel på den slags tabeller med afstande mellem større byer kan man finde i de fleste vejatlas.

Der findes en algoritme til ruteplanlægning, som kun bruger et vejkort, en rulle sytråd og en masse fingerfærdighed, men ingen computer.
Man lægger tynde tråde langs gaderne på vejkortet.
Ved vejkryds knytter man trådene sammen til knuder; udgangspunktet (»kilden«) får ligeledes en knude.
Nu lyfter man forsigtigt kildeknuden lodret opad, til hele netværket dingler under den. 
Lykkes man med dette uden at forkludre trådene, og knuderne er så små, at kun stramme tråde kan hindre dem i at falde ned, så definerer de spændte trådstykker netop de korteste veje fra udgangsknuden.
Billedet foroven\footnote{{\copyright} KIT, Institut f\"ur Photogrammetrie und Fernerkundung.} 
viser Instituttet for Teknologi i Karlsruhe (KIT) og den trådbaserede ruteplanlægningsalgoritme med $M$ som kilde.
}


Ruteplanlægning
\index{korteste veje!anvendelse}
\index{graf!korteste veje|siehe{korteste veje}} 
i vejnet er en af de mange anvendelser for korteste-veje-beregninger.
Men der er mange andre problemer, som løses bedste af korteste-veje-beregninger, når man har fundet en passende modellering i termer af grafer.
Ahuja \emph{et al.}~\cite{AhuMagOrl93} nævner anvendelser som netværksstrømme, boligbyggeri, forvaltning af lagerbeholdning i produktion og handel, 
dna-sekvensering, rygsæksproblemet 
\index{rygkæksproblemet} 
(se også kapitel~\ref{ch:optimization:}), produktionsplanlægning, skemalægning af arbejdstider fx for callcentre,
\index{skemalægning}
parkering, approksimation af stykvist linjære funktioner og fordeling af inspektionstid af et samlebånd.
\index{korteste veje!anvendelse}

I den mest almene formulering af korteste-veje-problemet betrager man en rettet graf $G=(V,E)$ og en omkostningsfunktion $c$, som afbilder hver kant til en reelt tal, som modellerer omkostningen ved at passere denne kant. 
Det viser sig, at problemet i denne formulering er overordentligt tidskrævende at løse.
Defor interesserer vi os for forskellige begrænsninger, som tillader enkle og effektive algoritmer: 
ikkenegative kantomkostninger,
\index{korteste veje!ikkenegative kantomkostninger}, 
heltallige kantomkostninger
\index{korteste veje!heltallige kantomkostninger}
og acykliske grafer.
\index{korteste veje!acyklisk graf}
Læg mærke til at det meget særlige tilfælde, hvor alle kanter har omkostning~1
\index{graf!bfs}
\index{korteste veje!identiske kantomkostninger}
allerede blev løst i kapitel~\ref{ch:gtraverse:s:bfs} -- bredde-først-træet med rod~$s$  er en kompakt repræsentation af de korteste veje fra $s$ til samtlige andre knuder. 
I afsnit~\lref{s:concepts} betragter vi indledningsvist nogle grundlæggende begreber, som leder til et fælles grundlag for alle korteste-veje-algoritmer. 
En systematisk fremgangsmåde skal hjælpe os at holde styr på de mange forskellige korteste-veje-algoritmer.
Som første eksempel på en hurtig og enkel algoritme for en begrænset problemklasse betragter vi i afsnit~\lref{s:acyclic} en fremgangsmåde for acykliske grafer.
I afsnit~\lref{s:Dijkstra} kommer vi til den mestbenyttede algoritme for kortest veje, nemlig Dijkstras algoritme for rettede grafer med ikkenegative kantomkostninger.
Hvor effektiv Dijkstras algoritme i virkeligheden er, afhænger af, hvor effektive de benyttede prioritetskøer er.
I et indledende kursus eller ved første gennemlæsning kan man sagtens afslutte temaet »korteste veje« med Dijkstras algoritme.
Men resten af kapitlet indeholder mange andre interessante aspekter af temaet.
Vi begynder men en analyse af gennemstnitsopførelsen af Dijkstra i afsnit~\lref{s:average}, som antyder, at operationerne på prioritetskøen i normaltilfældet er mindre dominerende end hvad værstefaldsanalysen fortæller.
I afsnit~\lref{s:monotone} diskuterer vi \emph{monotone prioritetskøen for heltallige nøgler}, som udnytter en særlig egenskab i Dijkstras algoritme.
Når man kombinerer disse køer med gennemsnitsanalysen, opnår man lineær forventet udførelsetid!
Afsnit~\lref{s:arbitrary edge costs} behandler situationen med vilkårlige kantomkostninger, og i afsnit~\lref{s:potential} behandler vi problemet at finde korteste veje mellem alle par af knuder.
Vi viser, at problemet kan reducere til først at løse en instans med én kildeknude og vilkårlige kantomkostninger, og derefter $n$ instanser med én kildeknude og ikkenegative kantomkostninger.
I løbet af reduktionen møder læseren også det generelt nyttige koncept af knudepotentialer.
Kapitlet afrundes med afsnit~\lref{s:query}, fokusseres på korteste-veje-\emph{forespørgsler}.

%----------------------------------------------------------------------
\section{Fra grundbegreberne til en almen metode}
\llabel{s:concepts}
%%%%%%%%%%%%%%%%%%%%%%%%%%%%%%%%%%%%%%%%%%%%%%%%%%%%%%%%%%%%%%%%%%%%%%%%%%%%%%%%%%%%%%%%%%%%%%

Omkostningsfunktionen for kanterne udvides til veje på den oplagte måde.
Omkostningen af en vej er summen af omkostningerne af dens kanter, dvs.
at omkostningen for vejen $p = \seq{e_1,\ldots,e_k}$ er givet af $c(p) = \sum_{1 \le i \le k} c(e_i)$. 
Den tomme vej har omkostning 0.
Vi bruger begreberne »omkostning«, »længde« og »vægt« i flæng, så »kortest« og »billigst« betyder det samme.

For knuder $s$ og $v$ interesserer vi os for en \emph{korteste vej} fra $s$ til $v$, dvs. en vej fra $s$ til $v$, som koster mindst muligt. 
Vi undgår her bevidst den bestemte artikel som i »den korteste vej«, idet der kan findes flere forskellige veje, som alle i denne forstand er kortest mulige.
Findes der overhovedet altid en korteste vej?
Læg mærke til, at der kan findes uendeligt mange veje fra $s$ til $v$.
Hvis fx vejen $r = pCq$ fra $s$ til $v$ indeholder en kreds $C$, kan vi gennemløbe kredsen vilkårligt ofte og danner hver gang en ny vej fra $s$ til $v$, se figur~\lref{path with cycle}.  
Lad os beskrive denne sitation mere nøjagtigt:
Lad $p$ være en vej fra $s$ til $u$, lad $C$ være en kreds fra $u$ til $u$, og lad $q$ være en vej fra $u$ til $v$. 
Betragt nu vejen $r^{(i)}=pC^iq$, som begynder med at gennemløbe $p$, derefter gennemløber $C$ præcis $i$ gange, og endeligt gennemløber $q$.
Omkostningen for $r^{(i)}$ er $c(p) +i \cdot c(C) +c(q) $. 
Hvis $C$ er en  \emph{negativ kreds},
\index{korteste veje!negativ kreds}
\index{graf!negativ kreds|sieheunter{korteste veje}}
dvs. hvis $c(C)<0$, så findes der ingen kortest vej fra  $s$ til $v$.
(Det er fordi mængden $\{\,r^{(i)} \colon i\ge0\,\} = \{\, c(p) + c(q) - i\cdot|c(C)|  \colon i\ge 0\,\}$ indeholder negative tal af vilkårlig stor absolutværdi.) 
Tilsvarende skal vi nu vise, at der findes kortest veje, hvis bare grafen $G$ ikke indeholder negative kredse.
\begin{figure}[h]
\[\begin{tikzpicture}[>=stealth']
\tikzstyle{vertex} = [inner sep = 1.5pt, fill, circle];
\node (s) at (0,0) [vertex, label = below:$s$] {};
\node (u) at (1,0) [vertex, label = below:$u$] {};
\node (v) at (2,0) [vertex, label = below:$v$] {};
\draw [->] (s) -- node [above] {$p$} (u);
\draw [->] (u) -- node [above] {$q$} (v);
\path [->] (u) edge [loop above, out = 45, in = 135, looseness = 30] node {$C$} (u);
\begin{scope}[xshift = 3cm]
\node (s) at (0,0) [vertex, label = below:$s$] {};
\node (u) at (1,0) [vertex, label = below:$u$] {};
\node (v) at (2,0) [vertex, label = below:$v$] {};
\draw [->] (s) -- node [above] {$p$} (u);
\draw [->] (u) -- node [above] {$q$} (v);
\path [->] (u) edge [loop above, out = 45, in = 135, looseness = 30] node {$C^2$} (u);
\path [->] (u) edge [loop above, out = 50, in = 140, looseness = 30] (u);
\end{scope}
\end{tikzpicture}
\]
\caption{En ikke-simpel vej $pCq$ fra $s$ til $v$}
\llabel{path with cycle}
\end{figure}

\begin{lemma}
\llabel{l:simple:path} 
Hvis $G$ ikke indeholder negative kreds og $v$ kan nås fra $s$, så findes der en korteste vej $p_0$ fra $s$ til $v$. 
Desuden kan $p_0$ antages af være simpel.
\end{lemma}
\begin{proof} 
Lad $p_0$ være en  \emph{simpel} vej fra $s$ til $v$ med minimale omkostninger blandt de simple veje fra $s$ til $v$.
Læg mærke til, at denne mængde af simple veje er endelig, og $p_0$ derfor eksisterer.
Antag modsætningsvist, at $p_0$ ikke er en korteste vej fra $s$ til $v$.
Da findes en kortere vej $r$ fra $s$ til $t$, som ikke er simpel.
Vi kan skrive $r$ som $pCq$ som i figur~\lref{path with cycle}, hvor $C$ mer en kreds og $pq$ er en simpel vej.
Vi har $c(p_0) \le c(pq)$, og derfor $c(pq) + c(C) = c(r) < c(p_0) \leq c(pq)$.
Heraf følger $c(C)<0$, så $C$ er en negativ kreds, i modstrid med antagelsen.
Derfor er $p_0$ en korteste vej.
\end{proof}


\begin{exerc}\llabel{ex:negative:cycles} 
Giv en stærkere udgave af ovenstående lemma ved at vise følgende:
For hver knude $v$, som kan nås fra $s$, findes en korteste vej fra $s$ til $v$ hvis og kun hvis der ikke findes en negativ kreds, som kan nås fra $s$, og fra hvilken man kan nå $v$.
\end{exerc}

For hvert par af knuder $s$ og $v$ definerer vi »korteste-veje-afstanden« eller bare \emph{afstanden $\mu(s,v)$ fra $s$ til $v$} som
\[
\mu(s,v) \Is
\begin{cases}
+\infty\,, & \text{hvis der ikke findes en vej fra $s$ til $v$}\,;\\
-\infty\,, & \parbox[t]{8cm}{hvis der findes en vej fra $s$ til $v$, men ingen korteste vej fra~$s$ til~$v$\,;}\\
c(p)\,,&\text{hvis $p$ er en korteste vej fra $s$ til $v$}\,.
\end{cases}
% TODO: second case completely wrong in original (error)
\]
Når startknuden $s$ er underforstået, skriver vi kort $\mu(v)\Is\mu(s,v)$. 
Læg mærke til, at når $v$ kan nås fra $s$, uden at der  findes en korteste vej, så må der findes veje fra $s$ til $v$ med vilkårlig store negative omkostninger.
Derfor giver det mening at sætte $\mu(v) = -\infty$.
Korteste veje har mange kønne egenskaber, som vi formulerer som øvelsesopgaver.
 
\begin{exerc}[Delveje af korteste veje]
\llabel{ex:dynamic}
\index{korteste veje!delvej}
Vis, at en delvej af en korteste vej selv er en korteste vej,
  \index{algoritmekonstruktion!dynamisk programmering!korteste veje}
dvs. at der hvis $p$, $q$ or $r$ er veje som sammen danner en korteste vej $pqr$, så er $q$ også en korteste vej.
\end{exerc}

\begin{exerc}[Korteste-veje-træer]
\index{korteste veje!træ}
Antag, at alle $n$ knuder i en graf kan nås fra $s$ og at der ikke forekommer negative kredse.
Vis, at der findes et træ $T$ med $n$ knuder, som er rodfæstet i $s$ og i hvilket alle veje er korteste veje i grafen. 
\emph{Vink}:
Begynd med at antage, at vejen fra $s$ til $v$ er entydigt bestemt.
\index{korteste veje!entydighed}
Betragt delgrafen $T$, som består af alle korteste veje fra $s$.
Anvend opgave~\ref{ex:dynamic} for at vise, at $T$ er et træ.
Udvid argumentet til det tilfælde, hvor de korteste veje ikke er entydigt bestemte.
\end{exerc}

Vores strategi for at finde korteste veje fra startknuden $s$ er en generalisering af bredde-først-søgning fra figur~\ref{ch:gtraverse:fig:bfs}.
Vi arbejder med to rækker af typen \Id{Knuderække}, som hedder $d$ og $\Id{forælder}$.
\index{korteste veje!forælderpeger}
Her indeholder indgangen $d[v]$ vores nuværende bedste bud på afstanden fra $s$ til $v$, og $\Id{forælder}[v]$ gemmer $v$s forgænger på en vej fra $s$ til $v$, som er den korteste, vi har fundet hidtil.
Værdien $d[v]$ kan man opfatte som \emph{skønnet afstand} til $v$, et »pessimistisk« skøn som aldrig undervurderer den sande afstand.
\index{korteste veje!afstandsskøn}
I begyndelsen har vi  $d[s]=0$ og $\Id{forælder}[s]=s$.
Alle andre knuders afstand skønner vi til at ligge »uendelig langt borte«, og de har ingen forgænger.

% TODO skal d være a?
% TODO brug startknude pga s, ikke udgangspunkt u eller kilde k

Den oplagte måde for at forbedre afstandsskønnene er at føre afstansinformation videre gennem enkelte kanter.
Når der findes en vej fra $s$ til $u$ med omkostning højst $d[u]$, og $e = (u,v)$ er en udgående kant fra $u$, så findes der også en vej fra $s$ til $v$ med omkostning højst $d[u] + c(e)$.
Hvis dette omkostning er mindre end den bedste hidtidlige afstandsværdi $d[v]$, opdaterer vi $d[v]$ og $\Id{forælder}[v]$ tilsvarende.
Denne proces kaldes at \emph{slække} kanten $e$.
\index{korteste veje!slækning af kanter|textbf}
\index{korteste veje!kantslækning|textbf}

\begin{indentedcode}
\Procedure \Id{slæk}$(\Declare{e=(u,v)}{\Id{Edge}})$\+\\
  \If  $d[u] + c(e) < d[v]$ \Then $d[v]\Is d[u] + c(e)$;$\;$ \Id{forælder}$[v]\Is u$
\end{indentedcode}
(Regnereglerne for værdien $\infty$ er de oplagte:
Der gælder $a<\infty$  og  $\infty+a=\infty$ for alle tal $a$, samt $\infty \not< \infty$.)

\begin{lemma}
\llabel{l:basic lemma} 
Antag at der efter en vilkårlig følge af kantslækninger gælder $d[v] < \infty$.
Da findes en vej fra $s$ til $v$ af længde $d[v]$.
\end{lemma} 
\begin{proof} 
Induktion  efter antal slækninger.
Påstanden gælder klart inden første slækning:
Den tomme vej fører fra $s$ til $s$ og har længde $0=d[s]$;
ingen andre knuder er berørte af udsagnet, da de skønnes at være uendeligt langt borte.
Betragt nu slækningen af kanten $e = (u,v)$. 
Ifølge induktionsantagelsen findes der en vej $p$ af længde $d[u]$ fra $s$ til $u$ og en vej af længde $d[v]$ fra $s$ til $v$.
Hvis $d[u] + c(e) \ge d[v]$, er der intet at vise. 
Ellers er $pe$ en vej fra $s$ til $v$ og har længde $d[u] + c(e)$.         
\end{proof}


\begin{figure}[tb]
  \[
  \begin{tikzpicture}[scale = 1.5, >=stealth']
    \tikzstyle{vertex} = [draw, text depth=.25ex, circle, inner sep = 1pt, minimum height = 13pt];
    \node (k) at (-2,0) [vertex, label = below:$+\infty$] {$k$};
    \node (j) at (-1,0) [vertex, fill = myblue, label = below:$-\infty$] {$j$};
    \node (s) at ( 0,0) [vertex, label = below:$0$] {$s$};
    \node (i) at ( 1,0) [vertex, label = below:$-1$] {$i$};
    \node (h) at ( 2,0) [vertex, label = below:$-3$] {$h$};
    \node (a) at (-2,1) [vertex, fill = myblue, label = above:$-\infty$] {$a$};
    \node (b) at (-1,1) [vertex, fill = myblue, label = above:$-\infty$] {$b$};
    \node (d) at ( 0,1) [vertex, fill = myblue, label = above:$-\infty$] {$d$};
    \node (f) at ( 1,1) [vertex, label = above:$2$] {$f$};
    \node (g) at ( 2,1) [vertex, label = above:$-3$] {$g$};
    \draw [->, dashed] (b) -- node [above] {42} (a);
    \draw [->, dashed] (b) -- node [left] {0} (j);
    \draw [->, dashed] (d) -- node [above] {0} (b);
    \draw [->]         (f) -- node [left] {$2$} (i);
    \draw [->, very thick] (g) -- node [above] {$5$} (f);
    \draw [->] (k) -- node [below] {$-2$} (j);
    \draw [->, dashed]     (j) -- node [right] {$-1$} (d);
    \draw [->, dashed]     (s) -- node [right] {$0$} (d);
    \draw [->, very thick] (s) -- node [below] {$-1$} (i);
    \draw [->, very thick] (i) -- node [right] {$-2$} (g);
    \draw [->, very thick] (i) -- node [below] {$-2$} (h);
  \end{tikzpicture}
  \]
\caption{
  \llabel{example}
  En graf med startknude $s$ og korteste-veje-afstande $\mu(v)$ angivet over eller under hver knude.
  Kantomkostningerne er angivet ved siden af hver kant.
  De fede kanter hører til korteste veje fra $s$.
  Stiplede linjer og schatterede knuder hører til en uendelig familje af veje med startknude $s$ og vilkårligt stærkt negative omkostninger.
  Knude $k$ kan ikke nås fra $s$.  }
\end{figure}

Algoritmerne i dette kapitel følger alle strategien at gentagne gange slække kanter, indtil enten der for hver knude $v$ er fundet en korteste vej fra $s$ til $v$, eller en negativ kreds er blevet opdaget.
For eksempel gengiver (omvendingen af) de fede kanter i figur~\lref{example} informationen i  \Id{forælder}-rækken, som er indsamlet efter tilstrækkelig mange kantslækninger:
knuderne $f$, $g$, $i$ og $h$ kan nås fra $s$ via disse kanter og har de korrekte $\mu(\cdot)$-værdier  $2$, $-3$, $-1$ og $-3$.
Knuderne $b$, $j$ og $d$ danner en negativ kreds, som kan nås fra $s$, hvilket betyder, at deres korteste-veje-omkostning er $-\infty$.
% TODO længde, vægt, omkostning, afstand
Knude $a$ kan nås fra denne kreds og har derfor ligeledes afstand $\mu(a)=-\infty$. 
Derimod kan knude $k$ ikke nås fra $s$; dens afstand er $+\infty$. 

Men hvad udgør en god følge af kantslækninger?
Lad $p = \seq{e_1,\ldots,e_k}$ være en vej fra $s$ til $v$. 
Hvis vi slækker vejens kanter i rækkefølge fra $e_1$ til $e_k$, gælder derefter $d[v] \le c(p)$.
Dersom $p$ er en korteste vej fra $s$ til $v$, kan ifølge lemmaet foroven værdien $d[v]$ aldrig blive mindre end $c(p)$, og derfor gælder der efter denne følge af kantslækninger, at $d[v] = c(p)$.


\begin{lemma}[Korrekthedskriterium]
  \index{korteste veje!korrekthedskriterium}
  \llabel{lem:criterion} 
  Betragt udførelsen af en følge $R$ af slækninger, som indeholder en korteste vej $p = \seq{e_1,e_2,\ldots,e_k}$ fra $s$ til $v$ som delfølge, dvs. at der findes indeks $t_1<t_2<\cdots<t_k$ med $R[t_1]=e_1$, $R[t_2]=e_2$, $\ldots$, $R[t_k]=e_k$.
  Da gælder $d[v] = \mu(v)$.
  Desuden beskriver $\Id{forælder}$-informationen for knuderne en vej fra $s$ til $v$ af længde $\mu$.
\end{lemma}


\begin{proof}
  Vi fremstiller $R$ og $p$ skematisk.
  Første linje angiver tidspunktet.
  Til tidspunkt $t_1$ slækkes $e_1$, til tidspunkt $t_2$ slækkes $e_2$, osv.:
  \[
    \begin{array}{lccccccc}
      &1,2,\ldots   &,t_1,         &\ldots\, , &t_2,&\ldots\ldots&,t_k,&\ldots\\
      R\Is\langle &\ldots       &,e_1,         &\ldots\, , &e_2,&\ldots\ldots&,e_k,&\ldots\rangle\\
      p\Is        &             &\langle e_1,  &           &e_2,&\ldots&,e_k\rangle&\\
    \end{array} 
  \]
  Idet $p$ er en korteste vej, har vi $\mu(v) = c(e_1)+\cdots + c(e_k)$. 
  For $i\in\{1,\ldots, k\}$ lad $v_i$ betegne målknuden af $e_i$; desuden definerer vi $t_0=0$ og $v_0=s$. 
  Lad os først vise, at efter tidspunkt $t_i$ gælder uligheden 
  \[
    d[v_i] \le c(e_1)+\cdots+ c(e_i)\,.
  \]
  Dette indses ved induktion efter $i$.
  Efter tidspunkt $t_0$, altså helt i starten, gælder uligheden, fordi vi initielt sætter $d[s]=0$.
  Betragt nu $i>0$. 
  Ifølge induktionsantagelsen gælder uligheden 
  \[ d[v_{i-1}] \le c(e_1)+\cdots + c(e_{i-1})\]
  efter slækningen til tidspunkt $t_{i-1}$.
  Yderligere slæknigner i tidsrummet mellem $t_{i-1}$ og $t_i$ ændrer ikke uligheden, idet $d[v_{i-1}]$ ikke kan vokse, når der slækkes en kant.
  Slækningen til tidspunkt $t_i$ sikrer, at
  $d[v_i] \le d[v_{i-1}] +c(v_{i-1},v_i)$.
  Ved addition med den sidste ulighed etablerer man induktionsantatelsen for tidspunktet $t_i$, hvilket beviser uligheden for alle $i\geq 0$.
  Uligheden for $i=k$ medfører da $d[v] \le \mu(p)$.
  Yderligere slækninger efter tidspunkt $t_k$ kan kun gørd $d[v]$ endnu mindre.
  På den anden side medfører~\lref{l:basic lemma}, at der gælder $d[v]\geq \mu(v)$ uanset tidspunkt.
  Vi konkluderer $d[v] = \mu(v)$.

  Vi mangler at vise, at informationen i \Id{forælder}-rækken beskriver de korteste veje.
  Vi viser dette under den ekstra antagelse, at korteste veje er entydige; generelle tilfælde er en  øvelse, jf.opgave~\lref{ex: relax general case}.
  Overvejelserne i første del af beviset medfører, at der efter slækningerne i $R$ gælder $d[v_i] = \mu(v_i)$ for alle $i \in \{1,\ldots, k\}$.
  På noget tidspunkt udfører altså en operation på formen $\Id{slæk}(u,v_i)$ i $R$ tildelingerne $d[v_i]=\mu(v_i)$ og  $\Id{forælder}[v_i]=u$.
  (Læg mærke til, at dette sagtens kan ske inden tidspunkt $t_i$, men ikke bagefter.)
  Ifølge beviset for lemma~\lref{l:basic lemma} findes der en vej af længde $\mu(v_i)$ fra $s$ til $v_i$, dvs. en korteste vej, der ender med kanten $(u,v_i)$.  
  Ifølge antagelsen er den korteste vej unik, derfor må der gælde $u=v_{i-1}$ og dermed $\Id{forælder}[v_i]=v_{i-1}$.
  %TODO this is not clear
  Efter at $d[v_i]$ er blevet sat til den korrekte værdi $\mu(v_i)$, beholder $\Id{forælder}[v_i]$ værdien $v_{i-1}$ i resten af forløbet.
\qed\end{proof}


\begin{exerc}
  \llabel{ex:relax general case}  
  Gentag ovenstående bevis, men uden antagelsen om entydighed af korteste veje.
\end{exerc}

\begin{exerc}
  Lad $S$ være en følge af kanterne i $G$ i vilkårlig rækkefølge, og betragt følgen $S^{(n-1)}$ af $(n-1)$ gentagelser af $S$.
  Vis at der efter slækningerne $S^{(n-1)}$ gælder $\mu(v) = d[v]$ for hver knude $v$ med $\mu(v) \not= -\infty$.
\end{exerc}

Som vi skal se i de følgende afsnit, kan man finde langt mere effektive slækningsrækkefølger, når grafen en acyklisk eller kantvægtene ikke-negative.
Vi vender tilbage til generelle grafer i afsnit~\lref{s:arbitrary edge costs}.


%----------------------------------------------------------------------
\section{Rettede acykliske grafer}
\llabel{s:acyclic}
\index{korteste veje!rettet acyklisk graf}

%%%%%%%%%%%%%%%%%%%%%%%%%%%%%%%%%%%%%%%%%%%%%%%%%%%%%%%%%%%%%%%%%%%%%%%%%%%%%%%%%%%%%%%%%%%%%%

I en rettet acyklisk graf forekommer ingen rettede kredse og derfor heller ingen negative kredse.
Desuden har vi konstateret i afsnit~\ref{ch:gtraverse:ss:numbering}k, at knuderne i en sådan graf tillader en topologisk ordning,
\index{graf!topologisk ordning}
dvs. at de kan bringes i rækkefølge $\seq{v_1, \ldots, v_n}$, sommetider kaldet en \emph{topologisk sortering}, sådan at $(v_i,v_j) \in E$ altid medfører $i < j$.
En topologisk sortering kan findes i tid $O(m + n)$ med dybde-først-søgning.
Læg mærke til, at knuderne langs hver vej i en rettet acyklisk graf står i voksende rækkefølge i den topologiske ordning.
Ifølge lemma~\lref{lem:criterion} kan vi altså beregne korteste-veje-afstande ved at slække først alle udgående kanter fra $v_1$, derefter alle udgående kanter fra $v_2$, og så videre.
I figur~\lref{dag} er vist et eksempel.
På denne måde bliver hver kant slækket højst enkelt gang.
% TODO original has exactly once, but that's wrong as observed in picture caption
Fordi slækning tager konstant tid, er udførelsestiden totalt $O(m+n)$.

\begin{figure}[t]
  \sidecaption
\leavevmode\parbox{5cm}{
  \begin{tikzpicture}[>=stealth', scale =.6]
    \tikzstyle{every node} = [inner sep = 1.5, circle];
    \tikzstyle{vertex} = [inner sep = 1.5, fill, draw, circle];
    \foreach \x in {-1,...,7}
      \draw [dashed, myblue] (\x,-1.5) -- (\x,2.5) ;
    \node (s) at (0,0) [vertex, label = below left:$s$] {}; 
    \node (m1) at (-1,2) [vertex] {}; 
    \node (1) at (+1,.5) [vertex] {}; 
    \node (2) at (+2,-1) [vertex] {}; 
    \node (3) at (+3,+2) [vertex] {}; 
    \node (4) at (+4,0) [vertex] {}; 
    \node (5) at (+5,1.5) [vertex] {}; 
    \node (6) at (+6,-1) [vertex] {}; 
    \node (7) at (+7,.5) [vertex] {}; 
    \draw [->] (m1) -- (s);
    \draw [->] (m1) to [bend left = 10] (3);
    \draw [->] (s) -- node [above left] {$1$} (1);
    \draw [->] (s) -- node [below] {$3$} (2);
    \draw [->] (s) -- node [below] {$2$} (4);
    \draw [->] (1) -- node [above left] {$4$} (3);
    \draw [->] (1) -- node [above] {$5$} (5);
    \draw [->] (2) to [bend right = 10] node [below] {$6$} (6);
    \draw [->] (4) -- node [above left] {$7$} (7);
    \draw [->] (4) -- node [above right] {$8$} (6);
    \draw [->] (5) -- node [above] {$9$} (7);
  \end{tikzpicture}
}
    \caption{\llabel{dag}
 En mulig rækkefølge for kantslækningerne til beregning af korteste veje i en rettet acyklisk graf fra knude $s$. 
 Den topologiske sortering af knuderne er vist ved deres placering efter $x$-koordinat; knuderne er sorterede fra venstre til højre.
  Udgående kanter fra knuder »til venstre« for $s$ behøver slet ikke slækkes.}
\end{figure}

\begin{thm} 
  I acykliske grafer kan korteste veje fra en enkelt knude beregnes i tid $O(m + n)$. 
\end{thm}


{\begin{exerc}[Ruteplanlægning ved offentlig transport]
  \index{korteste veje!offentlig transport}
  Problemet at komme hurtigt frem med offentlig transport kan modelleleres som korteste-veje-problem i en acyklist graf.
  Betragt en bus eller et tog, som forlader station $p$ ved tidspunkt $t$ og ankommer til næste station $p'$ ved tidspunkt $t'$.
  Denne forbindelse opfattes som kant fra knude $(p,t)$ til knude $(p',t')$.
  Desuden tilføjer vi såkaldte \emph{ventekanter} mellem på hinanden følgende afgangs- og ankomsttider ved hver station $p$; fx for tidspunkter $t$ og $t'$ med $t<t'$ som kant fra $(p,t)$ til $(p,t')$.
  \begin{enumerate}[(a)]
    \item Vis, at den resulterende graf er rettet og acyklisk.
    \item En ekstra knude skal bruge til at fastlægge den rejsendes udgangspunkt som sted og tid.
      Der skal også bruges en kant, som forbinder denne startknude med transportknetværket.
      Hvad er målknuden for denne kant?
    \item Antag, at vi har fundet et træ af korteste veje fra startknuden til alle knuder i transportnetværket.
      Hvilken er nu den rute, som den rejsende er interesseret i?
  \end{enumerate}
\end{exerc}


%%%%%%%%%%%%%%%%%%%%%%%%%%%%%%%%%%%%%%%%%%%%%%%%%%%%%%%%%%%%%%%%%%%%%%%%%%%%%%%%%%%%%%%%%
\section{Ikke-negative kantomkostninger (Dijkstras algoritme)}
\llabel{s:Dijkstra}%
\index{Dijkstra, E. W.}%
\index{Dijkstras algoritme|sieheunter{korteste veje}}%
\index{korteste veje!Dijkstras algoritme|textbf}%
\index{algoritmekonstruktion!grådig!Dijkstras algoritme}

I dette afsnit går vi ud fra, at der ikke forekommer kanter med negativ omkostning.
Det medfører direkte, at der ikke forekommer negative kredse, og at der derfor findes en korteste vej til hver knude, som kan nås fra $s$.
Vi skal vise, at man kan vælge slækningsrækkefølgen på en snedig måde, så hver kant bare slækkes en gang.

Men hvad er den rigtige rækkefølge?
Langs en korteste vej vokser korteste-veje-afstandene -- eller rettere, de bliver ikke mindre.
Det leder til idéen at bearbejde knuderne i rækkefølgen af voksende afstande.
\index{knude!bearbejde}
(Vi siger, at vi \emph{bearbejder} en knude, når vi slækker alle dens udgående kanter.) 
Lemma~\lref{lem:criterion} siger os, at denne slækningsrækkefølge medfører, at de korteste veje bliver beregnede, i det mindste, hvis alle kanter har positiv længde.
I algoritmen kender vi selvfølgeligt ikke korteste-veje-afstandene (endnu), men kun de \emph{skønnede} afstande $d[v]$.
Heldigvis stemmer for en hidtil ubearbejdede knuder med minimalt afstandsskøn den sande afstand og den skønnede afstand overens.
Det skal vi vise i sætning~\lref{thm:Dijkstra}.
Disse overvejelser munder ud i algoritmen i figur.~\lref{a:Dijkstra}, som er kendt som  »Dijkstras algoritme for korteste veje«.
I figur~\lref{fig:dijkstraex} vises et eksempel på en kørsel.

\begin{buchalgorithmpos}{b}{a:Dijkstra}{Dijkstras algoritme:
  Korteste veje fra samme udgangspunkt i grafer med ikke-negative kantvægte}%
{\bf Dijkstras algoritme}\\
 Alle knuder er  »ubearbejdede«; initialiser $d$ und \Id{forælder}\\
\While der findes en ubearbejdet knude med afstandsskøn $< +\infty$ \Do\+\\[0.5em]
  \begin{minipage}[c]{0.6\textwidth}
  $u \Is \text{ en ubearbejdet knude minimalt afstandsskøn}$ \\
  slæk alle $u$s udgående kanter $(u,v)$;\\
  hermed er $u$ »bearbejdet«%
  \end{minipage}\qquad\qquad\qquad\begin{minipage}[c]{0.2\textwidth}
       \includegraphics[width=\textwidth]{img/dijkstra2-new.eps}
                 \end{minipage}%
\end{buchalgorithmpos}

\newcommand{\myrelax}[3]{\Id{slæk} $#1\stackrel{#3}{\rightarrow}#2$}
\begin{figure}
\newcommand{\stentMD}{\rule{0em}{2.6ex}}
\begin{minipage}{0.5\textwidth}
\begin{tabular}{ll}
Operation & Kø\\\midrule
\Id{tilføj}$(s)$ & $\seq{(s,0)}$\\
\Id{fjernMindste}$\leadsto (s,0)$ & $\seq{\,}$ \stentMD\\
\hspace*{0.5em}\myrelax{s}{a}{2}  & $\seq{(a,2)}$ \stentMD\\
\hspace*{0.5em}\myrelax{s}{d}{10} & $\seq{(a,2),(d,10)}$ \stentMD\\
\Id{fjernMindste}$\leadsto (a,2)$ & $\seq{(d,10)}$ \stentMD\\
\hspace*{0.5em}\myrelax{a}{b}{3} & $\seq{(b,5),(d,10)}$ \stentMD\\
\Id{fjernMindste}$\leadsto (b,5)$ & $\seq{(d,10)}$ \stentMD\\
\hspace*{0.5em}\myrelax{b}{c}{2} & $\seq{(c,7),(d,10)}$ \stentMD\\
\hspace*{0.5em}\myrelax{b}{e}{1} & $\seq{(e,6),(c,7),(d,10)}$ \stentMD\\
\Id{fjernMindste}$\leadsto (e,6)$ & $\seq{(c,7),(d,10)}$ \stentMD\\
\hspace*{0.5em}\myrelax{e}{b}{9} & $\seq{(c,7),(d,10)}$ \stentMD\\
\hspace*{0.5em}\myrelax{e}{c}{8} & $\seq{(c,7),(d,10)}$ \stentMD\\
\hspace*{0.5em}\myrelax{e}{d}{0} & $\seq{(d,6),(c,7)}$ \stentMD\\
\Id{fjernMindste}$\leadsto (d,6)$ & $\seq{(c,7)}$ \stentMD\\
\hspace*{0.5em}\myrelax{d}{s}{4} & $\seq{(c,7)}$ \stentMD\\
\hspace*{0.5em}\myrelax{d}{b}{5} & $\seq{(c,7)}$ \stentMD\\
\Id{fjernMindste}$\leadsto (c,7)$ & $\seq{\,}$ \stentMD 
\end{tabular}%
\end{minipage}
%\caption{\llabel{fig:dijkstraex}Example run of Dijkstra's algorithm.
%Fat edges give the shortes path tree. Fat numbers give the shortest path
%distance.}
\begin{minipage}{0.5\textwidth}
  \[
    \begin{tikzpicture}[xscale = 1.25, >=stealth']
    \tikzstyle{vertex} = [draw, text depth=.25ex, circle, inner sep = 1pt, minimum height = 13pt];
      \node (s) at (0,0) [vertex, label =left:$0$] {$s$};
      \node (a) at (1,1) [vertex, label =above:$2$] {$a$};
      \node (b) at (2,1) [vertex, label =above:$5$] {$b$};
      \node (c) at (3,1) [vertex, label =above:$7$] {$c$};
      \node (d) at (1,-1) [vertex, label = below:$6$] {$d$};
      \node (e) at (2,-1) [vertex, label = below:$6$] {$e$};
      \node (f) at (3,-1) [vertex, label = below:$\infty$] {$f$};
      \draw [->, very thick] (s) -- node [above left] {$2$} (a);
      \draw [->, very thick] (a) -- node [above] {$3$} (b);
      \draw [->, very thick] (b) -- node [above] {$2$} (c);
      \draw [->, very thick] (b) to [bend left = 10] node [right] {$1$} (e);
      \draw [->, very thick] (e) -- node [below] {$0$} (d);
      \draw [->] (s) to [bend left = 10] node [above right ] {$10$} (d);
      \draw [->] (d) to [bend left = 10] node [below left] {$4$} (s);
      \draw [->] (d) -- node [above left] {$5$} (b);
      \draw [->] (e) to [bend left = 10] node [left] {$9$} (b);
      \draw [->] (e) -- node [below right] {$8$} (c);
      \draw [->] (f) -- node [below] {$7$} (e);
    \end{tikzpicture}
  \]
\caption{\llabel{fig:dijkstraex}
  Kørsel af Dijkstras algoritme på grafen \emph{til højre}.
  De fede kanter danner et korteste-veje-træ, tallene ved hver knude angiver deres korteste-veje-afstand.
  Tabellen \emph{til venstre} viser de udførte skridt.
  Køen indeholder en tupel $(v,d[v])$ for hver opdagede knude $v$,
  \index{knude!opdaget}
  som endnu ikke er bearbejdet.
  Vi siger at en knude er  \emph{opdaget}, hvis dens afstandsskøn are mindre end $+\infty$.
  I begyndelsen er $s$ den eneste opdagede knude; den er ikke bearbejdet.
  Algoritmens skridt står i første spalte, køens indhold efter det pågældende skridt står i  anden spalte.}
\end{minipage}%
\end{figure}

I grunden er Dijkstras algoritme ikke andet end tråd-og-knude-algoritmen fra kapitlets indledning, i det mindste for urettede grafer.
Hvis vi lægger alle tråde på et bord og langsom lyfter startknuden i vejret, slipper de andre knuder bordpladen i samme rækkefølge som deres korteste-veje-afstand.

\begin{thm}\llabel{thm:Dijkstra}
  Dijkstras algoritme løser korteste-veje-problemet med samme udgangspunkt for grafer med ikke-negative kantomkostninger.
\end{thm}%
\begin{proof} 
  Beviset består af to dele.
  I første del skal vi vise, at alle knuder, der kan nås fra $s$, bearbejdes før eller senere.
  I anden del skal vi vise, at en knudes afstandsskøn er lig med dens korteste-veje-avstand på det tidspunkt, hvor den bearbejdes.

  En knude $v$ kaldes \emph{opdaget}, når dens afstandsskøn tildeles en endelig værdi, dvs. $d[v]< +\infty$.
  Det ses direkte fra den ydre løkke, at algoritmen før eller siden behandler hver knude $v$, der er blevet opdaget.

  Første del:
  Vi har netop indset, at det er nok at vise, at hver knude $v$, der kan nås fra $s$ bliver opdaget før eller siden.
  Betragt hertil en vej $p = \seq{v_1,\ldots,v_k}$ med $s=v_1$ og $v=v_k$.
  Vi skal vise ved induktion efter $i$, at $v_i$ bliver opdaget.
  Knude $v_1$ bliver opdaget, idet intialiseringen sætter $d[s]=0<+\infty$.
  Betragt nu $i>1$.
  Ifølge induktionshypotesen bliver $v_{i-1}$ opdatet, hvilket betyder $d[v_{i-1}]<+\infty$.
  Før eller siden bliver $v_{i-1}$ bearbejdet, hvilket fører til kaldet $\Id{slæk} (v_{i-1},v_i)$.
  Herefter (hvis ikke før) gælder $d[v_i] \le d[v_{i-1}]+c(v_{i-1},v_i) < +\infty$, og $v_i$ er opdaget.

  Anden del:
  Vi viser følgende påstand ved induktion efter antal løkkegennemløb (»runder«):
  Hvis knude $u$ behandles i runde $t$, så gælder $d[u] \le \mu[u]$.
  (Læg mærke til, at den modsatte ulighed $d[u]\ge \mu[u]$ altid gælder ifølge lemma~\lref{l:basic lemma}.)
  Påstanden gælder for $t=1$, fordi runde 1 bearbejder startknuden~$s$, og vi har  $d[s]=0=\mu(s)$.
  Betragt nu runde $t>1$ og lad $u$ med $u\neq s$ være den knude, som behandles i denne runde. 
  Vælg en vilkårlig vej $p =\seq{v_1,\ldots,v_k}$  med $s=v_1$ og $u=v_k$.
  Lad $v_i$ for $i \in\{2,\ldots, k\}$ være den første knuden på vejen, som \emph{ikke} bearbejdes før runde~$t$.
  Dette er veldefinieret, idet $v_k$ har denne egenskab. %(Mindestens $v_k$ hat diese Eigenschaft.)
  Da bliver $v_{i-1}$ bearbejdet før runde $t$.
  I den forbindelse udføres kaldet  $\Id{slæk}(v_{i-1},v_i)$.
  Derfor gælder, at
  \[
d[v_i] \le d[v_{i-1}] + c(v_{i-1},v_{i}) \le \mu(v_{i-1}) + c(v_{i-1},v_{i})\,.
\]
  (Første ulighed etableres af operationen $\Id{slæk}(v_{i-1},v_i)$; anden ulighed er netop induktionsantagelsen.)
  Vejen $\seq{v_1,\ldots,v_{i-1}}$ fra $s$ til $v_{i-1}$ har omkostning mindst $\mu(v_{i-1})$, så der gælder 
  \[
\mu(v_{i-1}) + c(v_{i-1},v_{i}) \le 
           c(\seq{v_1,\ldots,v_{i-1}}) + c(v_{i-1},v_{i}) = c(\seq{v_1,\ldots,v_i})\,.
	 \]
  Idet $\seq{v_1,\ldots,v_i}$ udgør en delvej af $p$, og kantomkostningerne ikke er negative, gælder desuden $c(\seq{v_1,\ldots,v_i}) \le c(p)$. 
  Ved at sammenholde de sidste tre uligheder får vi, at der efter operationen $\Id{slæk}(v_{i-1},v_i)$ gælder $d[v_i] \le c(p)$.
  Selv hvis $d[v_i]$ herefter skulle aftage, gælder uligheden også i begyndelsen af runde $t$.
  Nu vælger algoritmen i runde $t$ knuden $u$ som knuden med minimalt afstandsskøn til bearbejdning, så der må gælde $d[u]\le d[v_i]$, hvilket medfører $d[u]\le c(p)$.

  Da $p$ var valgt vilkårlig, gælder $d[u] \le \mu(u)$, og vi har etableret induktionshypotesen for $t$.
\end{proof}


\begin{exerc}
  \llabel{ex:monotone} 
  Lad $v_1, v_2, \ldots$ være rækkefølgen, som knuderne bearbejdes i.
  Vis, at der gælder $\mu(v_1) \le \mu(v_2) \le \ldots$.
  Med andre ord bearbejdes knuderne i rækkefølge af (svagt) voksende korteste-veje-afstand. 
\end{exerc}

\begin{exerc}[Kontrol af korteste-veje-afstand]
  \llabel{e: verification}
  \index{algoritmekonstruktion!resultatkontrol}%
  Antag, at alle kantomkostninger er positive, og at alle knuder kan nås fra $s$.
  Lad $d$ være en knuderække med ikke-negative reelle indgange, som opfylder $d[s] = 0$ og $d[v] = \min_{(u,v) \in E} d[u] + c(u,v)$ for  alle $v \neq s$.
  Vis, at der for alle $v$ gælder $d[v] = \mu(v)$.
  Er denne påstand stadig sand, når der der er kanter med omkostning 0?
\end{exerc}

Vi skal nu se nærmere på, hvordan Dijkstras algoritme kan implementeres.
Alle ubearbejdede knuder stilles i en addresserbar prioritetskø
\index{prioritetskø!addresserbar}
\index{prioritetskø!monoton}
\index{prioritetskø!anvendelse}
(se afsnit~\ref{ch:pq:s:addressable}), med afstandsskønnet som nøgle. 
Operationen \Id{fjernMindste} leverer så altid den knude, som skal bearbejdes hernest.
Til adressering af indgange for \Id{sænkNøgle}-operation bruger vi her bare knudenummeret i stedet for greb.
Med udgangspunkt i den sædvanlig prioritetskø kan man implementere sådan en datastruktur, so vi kalder \Id{KnudePK}, med en yderligere knuderække, som oversætter knudenumre til greb.
Hvis prioritetskøen er implementeret ved hjælp af en træsstruktur, som fx ved fibonaccihobe eller parhobe, kan man gemme indgange af \Id{KnudePK}, dvs. træknuderne, direkte i en knuderække.
Kaldet  \Id{sænkNøgle}$(v)$ opdaterer \Id{KnudePK}-køen under antagelse af, at nøglen $d[v]$ blev mindsket umiddelbart inden.

\begin{buchalgorithmpos}{t}{alg:dijkstra}{Pseudokode for Dijkstras algoritme}
\Funct{Dijkstra}{\Declare{s}{\Id{KnudeId}}}{\Id{Knuderække}$\times$\Id{Knuderække}}\+\comment{returnerer $(d,\Id{forælder})$}\\
  \DeclareInit{$d$}{\Id{Knuderække} \Of $\RR\cup\set{\infty}$}{$\seq{\infty,\ldots,\infty}$}\comment{afstandsskøn}\\
  \DeclareInit{\Id{forælder}}{\Id{Knuderække} \Of \Id{KnudeId}}{$\seq{\bot,\ldots,\bot}$}\\
  $\Id{forælder}[s]\Is s$\comment{løkke signaliserer roden}\\
  \Declare{$Q$}{\Id{KnudePK}} \comment{kø af ubearbejdede opdagede knuder}\\
  $d[s]\Is 0$;\quad  $Q$.\Id{tilføj}$(s)$\\
  \While $Q\neq\emptyset$ \Do\+\\
    $u\Is Q.\Id{fjernMindste}$\comment{der gælder $d[u] = \mu(u)$}\\[-3mm]
    %\Assert $d[u]=\mu(u)$\\[-2mm]
    \Foreach kant $e=(u,v)\in E$
      \Do\+\`\begin{minipage}[c]{37mm}\includegraphics{img/dijkstra2-new.eps}\end{minipage}\hspace*{2cm}\\[-2.5mm]
      \If $d[u]+c(e)<d[v]$ \Then\+\comment{slæk}\\
        $d[v]\Is d[u]+c(e)$\\
        $\Id{forælder}[v]\Is u$\comment{opdater træ}\\
        \If $v\in Q$ \Then $Q$.\Id{sænkNøgle}$(v)$\\[-2mm]
        \Else $Q$.\Id{tilføj}$(v)$\`\begin{minipage}[c]{17mm}
                               \includegraphics{img/dijkstra3-new.eps}\end{minipage}\-\-\-\\[-2mm]
  \Return $(d,\Id{forælder})$
\end{buchalgorithmpos}
%
%
\newcommand{\TDijkstra}{T_{\Id{Dijkstra}}}%  MD: die Funktion hat den Namen, schräggestellt.
\newcommand{\TdeleteMin}{T_{\Id{fjernMindste}}}%
\newcommand{\Tinsert}{T_{\Id{tilføj}}}%
\newcommand{\TdecreaseKey}{T_{\Id{sænkNøgle}}}%
%
Figur~\lref{alg:dijkstra} viser en detaljeret fremstilling af algoritmen.
Vi vil nu analysere algoritmens udførelsestid med udgangspunkt i tidsforbruget i det enkelte prioritetskø\-operationer.
Initialiseringen af rækkerne $d$ og \Id{forælder} samt prioritetskøen $Q=\set{s}$ koster tid $O(n)$.
Ved hver iteration af den ydre løkke kræver betingelsen $Q=\emptyset$ konstant tid, dvs. total tid  $O(n)$.
Hver knude $s$, som kan nås fra $s$, bliver fjernet fra prioritetskøen præcis én gang.
Bortset fra $s$ selv, bliver hver af disse køer også stillet i prioritetskøen præcis én gang.
Derfor forekommer der totalt højst $n$ kald af  \Id{fjernMindste} og \Id{tilføj} til $Q$.
Idet hver knude bearbejdes højst én gang, slækkes hver kant højst én gang, og derfor kan der forekomme højst $m$ kald til \Id{sænkMindste}.
Den samlede udførelsestid for Dijkstras algoritme er altså
\begin{equation*}\llabel{eq:dijkstra}
\TDijkstra=O(m\cdot\TdecreaseKey(n) + n\cdot(\TdeleteMin(n)+\Tinsert(n)))\,,
\end{equation*}
hvor $\TdeleteMin$, $\Tinsert$ og $\TdecreaseKey$ angiver udførelsestiderne for \Id{fjernMindste}, \Id{tilføj} hhv. \Id{sænkMindste}. 
Læg mærke til, at disse udførelsestider afhænger af størrelsen $|Q|=O(n)$ af køen.

\begin{exerc} 
  Givet $n, m$ med $n\le m \le n(n-1)/2$.
  Konstruer en graf $G$ med $n$ knuder og $m$ kanter og velvalgte kantvægte, sådan at
  $m - (n-1)$ af kantslækningerne i løbet af Dijkstras algoritme udløser et kald af \Id{sænkNøgle}.
\index{prioritetskø!sænk nøgle@\Id{sænkNøgle}}
\end{exerc}

I originalpublikationen fra 1959 foreslog Dijkstra følgende implementation af prioritetskøen:
Hold rede på antallet af opdagede, ubearbejdede knuder samt to rækker,
\index{prioritetskø!naiv},
hver indiceret med knudemængden som indeks -- en række $d$ til afstandsskønnene og en række, som for hver knude angiver, om den er enten uopdaget, opdaget men ubearbejdet eller bearbejdet.
Så koster \Id{tilføj} og \Id{formindskNøgle} tid $O(1)$. 
Operationen \Id{fjernMindste} koster $O(n)$, fordi hele rækken skal gennemgås for at finde en opdaget ubearbejdet knude med minimalt afstandsskøn.
Den samlede tid bliver altså 
\[T_{\Id{Dijkstra59}}=O(m+n^2)\,.\]

Siden Dijkstras artikel har man fundet på meget bedre prioritetskøer.
Hvis vi bruger en binær hob
\index{prioritetskø!binær hob} eller en fibononaccihob
\index{Prioritätswarteschlange!Fibonacci-Heap}
som beskrevet i afsnit~\ref{ch:pq:s:addressable}, får vi
\begin{align*}
T_{\Id{DijkstraBHeap}}&=O((m + n)\log n)\\
\intertext{hhv.}
T_{\Id{DijkstraFibonacci}}&=O(m + n\log n)\,.
\end{align*}
Ifølge den asympototiske analyse er implementationen med fibonaccihob bedre, medmindre grafen er tynd, med 
$m = O(n)$. 
I praksis giver fibonaccihobe ikke nødvendigvis den hurtigste implementation, både fordi analysen skuler ganske store konstante faktorer og der normalt forekommer langt færre \Id{sænkNøgle}-operationer, end værstefaldsanalysen forudser.
Vi skal underbygge disse eksperimentelle observationer med en teoretisk analyse i næste afsnit.

\section{Gennemsnitsanalyse af Dijkstras algoritme}
\llabel{s:average}%

\emph{Udeladt}

\section{Monotone prioritetskøer over heltal}
\llabel{s:monotone}

\emph{Udeladt}

%%%%%%%%%%%%%%%%%%%%%%%%%%%%%%%%%%%%%%%%%%%%%%%%%%%%%%%%%%%%%%%%%%%%%%
\section{Vilkårlige kantomkostninger (Bellman og Fords algoritme)}
\llabel{s:arbitrary edge costs}
\index{korteste veje!Bellman og Fords algoritme}%
\index{korteste veje!vilkårlige kantomkostninger}%
\index{algoritmekonstruktion!dynamisk programmering!Bellman og Fords algoritme}
\index{Bellman og Fords algoritma|sieheunter{korteste veje}}%

For acykliske grafer med ikke-negative kantomkostninger kunne vi nøjes med $m$ kantslækninger.
For vilkårlige grafer, dvs. grafer som kan indeholde kredse, findes der ikke noget tilsvarende resultat.
Men man kan hurtigt overbevise sig om, at $O(mn)$ kantslækninger er tilstrækkelige for at etablere korrektheskriteriet i lemma~\lref{lem:criterion}:
Algoritmen af  Bellman og Ford~\cite{Bel58,Ford56},
\index{Bellman, R.}
\index{Ford, L. R., Jr.}
som er vist i figur~\lref{alg:BellmanFord}, kører i $n-1$ runder.
I hver runde slækkes hver kant én gang.
Idet en simpel vej består af højst $n-1$ kanter, er hver korteste vej en delfølge af disse slækninger.
Ifølge lemma~\lref{l:simple:path} og lemma~\lref{lem:criterion} gælder der efter den sidste runde derfor $d[v] = \mu(v)$ for hver knude $v$ med $-\infty < \mu(v) < \infty$.
Desuden beskriver rækken $\Id{forælder}$ korteste veje fra $s$ til disse knuder.
For en knude $v$, som ikke kan nås fra $s$, gælder $d[v]=\infty$, som ønsket.

\begin{buchalgorithmpos}{t}{alg:BellmanFord}{Bellman og Fords algoritme for korteste veje i vilkårlige grafer}%
  \Funct{BellmanFord}{\Declare{s}{KnudeId}}{$\Id{Knuderække} \times\Id{Knuderække}$}\+\\
  \DeclareInit{$d$}{\Id{Knuderække} \Of $\RR\cup\set{-\infty,\infty}$}{$\seq{\infty,\ldots,\infty}$}\comment{Afstand fra startknuden}\\
  \DeclareInit{\Id{forælder}}{\Id{Knuderække} \Of \Id{KnudeId}}{$\seq{\bot,\ldots,\bot}$}\\
  $d[s]\Is 0$;\quad
  \Id{forælder}$[s]\Is s$\comment{Løkke angiver roden}\\
  \ForFromTo{i}{1}{n-1}\+\\ 
    \Foreach $e\in E$ \Do \Id{slæk}$(e)$\comment{Runde $i$}\-\\[0.2em]
  \Foreach $e=(u,v)\in E$ \Do\+\comment{Efterbehandling}\\
%    \Invariant $\Rorall v\in V\gilt d[v]=-\infty\rightarrow
%             \forall w\text{ reachable from }v\gilt d[w]=-\infty$\\
    \If $d[u]+c(e)<d[v]$ \Then infect$(v)$\-\\
  \Return $(d,\Id{forælder})$\-\\[2mm]
  \Procedure \Id{smit}$(v)$\+\\
  \If $d[v]>-\infty$ \Then\+\\
    $d[v]\Is-\infty$\\
\Foreach $(v,w)\in E$ \Do \Id{smit}$(w)$
\end{buchalgorithmpos}

% TODO: changed forall to foreach

På den anden side er det mindre klart, hvordan man identificerer de knuder $v$, som opfylder $\mu(v)=-\infty$.
Ifølge øvelsesopgave~\lref{ex:negative:cycles} er det netop de knuder, som kan nås fra $s$ med en vej, som indholder en kreds med negativ længde.
Følgende to egenskaber ved negative kredse er centrale:
\begin{lemma}
  Lad $C=\seq{v_0,\ldots,v_k}$ med $v_0=v_k$ være en kreds med negativ længde i grafen $G$, som kan nås fra $s$.
  Da eksisterer $i\in\{1,\ldots,k\}$, således at $d[v_{i-1}] + c((v_{i-1},v_i)) < d[v_i] $.
\end{lemma}
\begin{proof}
  Idet $C$ kan nås fra $s$, gælder $d[v_i] <+\infty$ for alle $i\in\{1,\ldots, k\}$.
  Antag modsætningsvist, at der for alle disse $i$ gælder $d[v_{i-1}]+c((v_{i-1},v_i)) \ge d[v_i]$.
  Når man lægger disse $k$ uligheder sammen, går $d[v_i]$-termerne ud mod hinanden, og vi får uligheden $\sum_{i=1}^kc((v_{i-1},v_i)) \ge 0$.
  Men så er længden af $C$ ikke negativ, i modstrid med antagelsen.
\end{proof}

\begin{lemma}
  Lad $v$ være en knude, som kan nås fra $s$.
  Da gælder $mu(v)=-\infty$ hvis og kun hvis der eksisterer en kant $e=(u,w)$, således at $d[u] + c(e) < d[w] < +\infty$, og $v$ kan nås fra $w$.
\end{lemma}
\begin{proof}
 Antag først $\mu(v)=-\infty$. 
 Da kan $v$ nås fra $s$ via en kreds $C=\seq{v_0,\ldots,v_k}$ med negativ længde.
 Ifølge lemmaet foroven indeholder $C$ en kant $e=(v_{i-1},v_i)$, som opfylder $d[v_{i-1}]+c((v_{i-1},v_i)) < d[v_i]$.
 Desuden kan $v$ nås fra $v_i$.
 Antag omvendt, at $e=(u,w)$ er en kant med $d[u] + c(e) < d[w] < +\infty$.
 Vi skal først indse, at 
  \begin{equation}\label{eq:muw}
    \mu(w) =-\infty\,.
  \end{equation}
  Antag modsætningsvis, at $\mu(w)>-\infty$.
  Idet $w$ klart kan nås fra $u$, må der også gælde $\mu(u)>-\infty$.
  Vi har altså $-\infty<\mu(u)<+\infty$, og kan bruge lemma~\lref{lem:criterion} til at indse, at $\mu(u)=d[u]$ og $\mu(w)=d[w]$. 
  Men uligheden $\mu(u) + c(e) < \mu(w)$ absurd, så \eqref{eq:muw} gælder.
  Med andre ord kann $w$ nås fra $s$ via en kreds af negativ længde.
  Via $w$ kan også $v$ nås på samme måde, så $\mu(v)=-\infty$.
\end{proof}

Disse overvejelser fører til følgende måde at identificere knuder $v$ med $\mu(v)=\infty$ ved at sætte $d[v]=-\infty$.
I efterbehandlingen prøver for hver kant $e=(u,v)$, om $d[u] + c(e) < d[v]$.
I så fald kalder vi $v$ for »smittet«, sætter $d[v]\Is -\infty$ og kalder den rekursive procedure $\Id{smit}(v)$, som gennemfører en slags bredde-først-søgning 
\index{graf!bredde-først-søgning}
for at sikre, at $d[w]=\infty$ for alle knuder $w$, som kan nås fra $v$.

\begin{exerc}
  Vis, at de knnuder, for hvilke kaldes $\Id{smit}(v)$ udføres en eller flere gange præcis er de knnuder, for hvilke  $\mu(v)=-\infty$, og at hele efterbehandlingen tager tid $O(m)$.
  \emph{Vink}: Genbrug argumenterne fra dybde-først-søgning.
\end{exerc}

%Lösung: Es ist klar, dass für jeden Knoten $v$ und jede Kante  $(u,v)$ der 
%Aufruf \Id{infect}$(v)$ höchstens einmal erfolgt,
%und dass der erste dieser Aufrufe Zeit $\Oh{\Id{outdegree}(v)}$ benötigt, die
%späteren jeweils Zeit $\Oh{1}$. Zusammen also Zeit  $\Oh{\Id{indegree}(v) + \Id{outdegree}(v)}$. 
%Aus Behauptung 2 folgt, dass \emph{nur} Knoten $v$ mit $\mu(v)= -\infty$ infiziert werden.
%(Achtung, subtil: Wenn $e=(u,v)$ getestet wird,
%könnte es sein, dass $v$ schon vorher infiziert worden ist.
%In diesem Fall gilt $d[u] + c(e) \not< d[v]=-\infty$ und es erfolgt kein Aufruf \Id{infect}$(v)$.)
%Wieso werden \emph{alle} Knoten $v$ mit $d[v]= -\infty$ infiziert?
%Sei nach den $n-1$ Runden $d[u] + c(e) < d[w] < +\infty$ und $v$ von $w$ aus erreichbar.
%Wähle einen (einfachen Weg) $p=\seq{w=v_0,v_1,\ldots,v_k=v}$.
%Kante $e=(u,w)$ wird auf jeden Fall getestet. Wenn dann $d[u] + c(e) \not< d[w]$ gilt, 
%kann das nur daran liegen, dass $w$ schon vorher infiziert und $d[w] \Is-\infty$ gesetzt wurde. 
%Also wird \Id{infect}$(w)$ auf jeden Fall irgendwann einmal aufgerufen.
%Annahme: \Id{infect}$(v)$ wird nicht aufgerufen. 
%Betrachte den letzten Knoten $v_i$ auf dem Weg, für den \Id{infect}$(v_i)$ irgendwann aufgerufen wird
%und betrachte den ersten solchen Aufruf. 
%In diesem Moment gilt noch $d[v_i]> -\infty$. 
%Während der Abarbeitung von \Id{infect}$(v_i)$ wird irgendwann Kante $(v_i,v_{i+1})$ betrachtet und es wird \Id{infect}$(v_{i+1})$ aufgerufen,
%im Widerspruch zur Wahl von $i$. Also ist die Annahme falsch, und \Id{infect}$(v)$ wird aufgerufen. 
%


\begin{exerc}
  Nogen foreslår en alternativ efterbehandling:
  Sæt $d[v]$ til $-\infty$ for alle knuder $v$ med $d[v]<+\infty$, for hvilke der ikke findes en vej af forgængerpegere $\Id{forælder}$ tilbage til startknuden  $s$.
  Vis et eksempel, hvor  denne metode kan overse en knude $v$ med $\mu(v)=-\infty$.
\end{exerc}

% Lösung: Abbildung 2 http://www.avglab.com/andrew/pub/neci-tr-96-029.ps
% Goldberg Cherkassky

\begin{exerc}[Arbitrage]
  \index{Arbitrage}
  \index{korteste veje!anvendelse}
  Betrage en mængde $C$ af valutaer med vekselkurs $r_{ij}$ mellem valuta $i$ og $j$, dvs. at man får $r_{ij}$ enheder af valuta $j$ for en enhed af valuta $i$. 
  En \emph{valutaarbitrage} opstår, når der findes en følge af elementære vekslinger (\emph{transaktioner}), som begynder med en enhed i en valuta og ender med mere end en enhed i samme valuta.
  \begin{enumerate}[(a)]
    \item Forklar, hvordan man kan afgøre, om en tabel af valutakurser tillader valutaarbitrage. 
      \emph{Vink}: $\log(xy)=\log x+\log y$.
    \item Udvid algoritmen til at beregne en følge af transaktioner, som maksimerer den gennemsnitlige gevinst \emph{per transaktion}!
  \end{enumerate}
\end{exerc}

I afsnit~\lref{s:further} beskriver vi yderligere forbedringer af Bellman og Fords algoritme, som gør den mere anvendlig i praksis.

%----------------------------------------------------------------------
\section{Korteste veje mellem alle par af knuder}
\llabel{s:potential}
\llabel{s:all-pairs}
\index{potentialefunktion|siehe{knudepotentiale}}%
\index{knudepotentiale|textbf}%
\index{korteste veje!alle par, negative omkostninger}

\emph{Udeladt}


%%%%%%%%%%%%%%%%%%%%%%%%%%%%%%%%%%%%%%%%%%%%%%%%%%%%%%%%%%%%%%%%%%%%%%
\section{Korteste-veje-opslag}
\llabel{s:query}
\index{kurzeste Wege@korteste veje!Anfrage}
\index{Algorithm Engineering}

\emph{Udeladt}


%%%%%%%%%%%%%%%%%%%%%%%%%%%%%%%%%%%%%%%%%%%%%%%%%%%%%%%%%%%%%%%%%%%%%%
\section{Implementationsaspekter}
\llabel{s:implementation}

\emph{Udeladt}
%%%%%%%%%%%%%%%%%%%%%%%%%%%%%%%%%%%%%%%%%%%%%%%%%%%%%%%%%%%%%%%%%%%%%%
\section{Historiske bemærknigner og videre resultater}

\llabel{s:further}

\emph{Udeladt}


