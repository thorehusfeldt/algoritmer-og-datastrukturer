\vspace*{-2.5cm}
\begin{minipage}{0.5\textwidth}
\begin{flushleft}
  \includegraphics[height=4cm]{img/cuneiform.eps}
\end{flushleft}
\end{minipage}
\begin{minipage}{0.5\textwidth}
\begin{flushright}
\includegraphics[height=4cm]{img/Quipu.eps} 
\end{flushright}
\end{minipage}
\vspace*{0.5cm}

\noindent
\emph{De måske ældste datastrukturer i verden var kileskriftstavlerne%
\footnote{%
Den 4600 år gamle tavle oppe til venstre indeholder en liste af gaver til ypperstepræstinden af Adab (se \texttt{commons.wikimedia.org/wiki/Image: Sumerian\_26th\_c\_Adab.jpg}).
},
som for mere en 5000 år siden blev brugt af vogterne i sumeriske templer.
\index{Sumer}
\index{kileskrift|textbf}
% TODO vogter?
Vogterne førte lister med varer, deres mængder, deres ejere og købere.
\index{liste}
Billedet til venstre viser sådan en tavle.
Dette er muligvis det første eksempel på skriftsprog.
Operationerne, som kunne udføres på disse lister, er de samme som dengang:
Indgange kan tilføjes, de gemmes til senere brug, man kan lede efter indgange og ændre dem, hele listen kan gennemgås for at skabe en sammenfatning, osv.
Det peruanske quipu [149] tjente i inkariget lignende formål;
\index{Peru}
\index{quipu|textbf}
hertil brugte man knuder med farvede snører, som i rækkefølje var anbragt på en hovedsnor.
\index{knude}
\index{snor}
Det er sandsynligvis lettere at gemme, pleje og anvende data på tavler end med knudesnore, men det er også ubekvemt at slæbe stentavler over bjergstier i Andesbjergene.
\index{tavle}
Åbenbart er det meningsfuldt at bruge forskellige repræsentationer af samme slags data alt efter behov.
}

\bigskip\noindent
Det abstrakte begreb følge, liste eller tabel er meget enkelt og har i første omgang ikke noget med dens repræsentation i en beregner at gøre.
\index{følge}
\index{liste}
\index{tabel}
Matematisk set er den eneste væsentlige egenskab ved følgen $\langle e_1,\ldots, e_n\rangle$, at en række af indgange er anbragt i en lineær ordning eller rækkefølge -- i modsætning graferne og træerne, som vi skal betragte i kapitel 7 og 8, og de uordnede spredetabeller i kapitel~4.
Grundlæggende kan man tilgå indgangene i en følge på to måder.

Den første mulighed består af at angive indgangens indeks.
\index{indeks}
Denne tilgang svarer til den sædvandlige adgang for en række,
\index{række}
\index{statisk række}
\index{række!statisk}
hvor man med udtrykket $s[i]$ får adgang til den $i$te indgang i følgen $s$.
Vores pseudokode (se afsnit~\ref{s:array}) stiller \emph{statiske rækker} til rådighed.
Vi taler om en \emph{statisk} datastruktur, når dens størrelse er givet på forhånd og ikke kan ændres ved tilføjelser og fjernelser.
Vi kalder datastrukturen \emph{begrænset},
\index{begrænset række|textbf}
\index{række!begrænset|textbf}
  når en øvre grænse på dens størreslse er givet på forhånd.
I afsnit 3.2 skal vi betragte \emph{dynamiske} eller \emph{ubegrænsede} rækker,
\index{ubegrænset række}
\index{række!ubegrænset}
som kan vokse og krympe, når indgange bliver tilføjet eller fjernet.
Regnetidsanalysen for ubegrænsede rækker er baseret på begrebet \emph{amortiseret analyse} af algoritmer.
\index{amortiseret analyse}
\index{algoritmeanalyse, amortiseret}

Den anden mulighed for at tilgå indgangene i en følge består i tage den aktuelle indgang som udgangspunkt.
For eksempel kunne man spørge om forgængeren til indgang $e$,
\index{forgænger|textbf}
efterfølgeren til indgang $e'$ 
\index{efterfølger|textbf}
eller delfølgen $\langle e,\ldots, e'\rangle$ af indgange mellen $e$ og $e'$.
Selvom den slags relativ adgang kan simuleres med rækker og indeksering, skal vi i afsnit~\ref{s:list} se, at fremstillingen af følger ved hjælp af lister er mere fleksibel.
Især gør denne fremstilling det lettere at indføje
\index{indføje i følge}
og fjerne
\index{fjerne fra følge}
vilfårlige afsnit i en følge.

Mange algoritmer nøjes med at benytte følger på en meget indskrænket måde.
Ofte bliver der kun læst og ændret i følgens begyndelse eller ende.
Følger, som benyttes på denne indskrænkede måde, hedder \emph{stakke}, \emph{køer} og \emph{dobbeltkøer}.
Disse datastrukturer betragtes i afsnit~\ref{s:stack}.
Afsnit~\ref{s:o} sammenfatter kapitlets resultater.

\section{Hægtede lister}
\label{s:list}


\newcommand{\friListe}{\Id{friListe}}

I dette afsnit betragter vi fremstillingen af følger ved brug af hægtede lister.
\index{liste!hægtet}
\index{liste!dobbelthægtet|textbf}
Hægtede lister er opbygget af \emph{knuder};
\index{knude!liste-}
hver knude består af et element fra følgen og en eller flere pegere.
Man kan forestille sig en hægtet liste som en kæde, hvor der står et følgeindgang på hvert led.
Så snart vi har et kædeled i hånden, kan vi nå og aflæse andre led i kæden.
I en dobbelthægtet liste peger hver knude på sin forgænger og sin efterfølger.
\index{forgænger}
\index{efterfølger}
I en enkelthægtet liste peger hver knude på sin efterfølger.
Vi skal se, at man kan let kan foretage ændringer i  hægtede lister på mange måder:
man kan indføje eller slette knuder eller dellister, og man kan hægte lister efter hinanden.
\index{liste!konkatenering}
Ulempen ved hægtede lister er, at de ikke understøtter vilkærlig adgang (dvs. operateroren $[\cdot]$).
Vi betragtrer dobbelthægtede lister i afsnit 3.1.1 og enkelthægtede lister i afstnit 3.2.1.
Enkelthægtede lister behøver mindre plads og er en smule hurtigere.
Derfør bør man foretrække dem, når deres funktionalitet er tilstrækkelig.

\subsection{Dobbelthægtede lister}

Den grundlæggende byggesten for en hægtet liste er afbildet i figur \ref{alg:ditem}
En \emph{knude} \index{knude!liste-} gemmer en indgang med to pegere til henholdsvis efterfølger og forgænger.
En peger til en knude kaldes også et \emph{greb}
\index{greb|textbf}
på denne knude, som i afsnit~|ref{ch:intro:pseudocode}.
Det lyder ganske ligetil, men pegere er så kraftfulde, at der kan opstå stor forvirring, medmindre man udviser stor omhu ved deres brug.
Vilken betingelse (»invariant«) garanterer listestrukturens konsistens, dvs. dens linearitet?
\index{invariant!for listestruktur}
\index{liste!invariant}
For det første skal man kræve, at forgængeren til efterfølgeren til knuden $k$ er lig med $k$, og at efterfølgeren til forgængeren til $k$ ligeledes er lig med $k$.
Ved siden af den lokale linearitet er det desuden nødvendigt at sørge for, at der findes et indgangspunkt til listen, hvorfra alle indgange kan nås ved at følge efterfølgerpegere.

\begin{figure}
  \begin{tabbing}
    ~~~~\=\hspace{4cm}\=\kill
    $\Class \Handle = \PointerTo \Node$\\
    \\
    $\Class \Node \Of \Element$ \qquad\textcolor{callout}{\emph{// Knude i en dobbelthægtet liste}}\\
    \> $e\colon \Element$\\
    \> $\nnext\colon\Handle$\> \tikz[remember picture] \node (lnode) {};\\
    \> $\prev\colon\Handle$\\
    \> $\Invariant \nnext\rightarrow\prev = \prev\rightarrow\nnext = \this$
  \end{tabbing}
  \tikzset{ 
    lnode/.style={rectangle split, rectangle split parts=3, draw}, 
    >=stealth'}
  \begin{tikzpicture}[callout, remember picture, overlay, anchor = west, shift = (lnode)]
    \node (this) [lnode, minimum width = 1cm] at (2,0) { $e$ \nodepart{two} $\bullet$ \nodepart{three} $\bullet$};
    \node (prev) [dashed,lnode, minimum width = 1cm] at (0,0) {};
    \node (next) [dashed,lnode, minimum width = 1cm] at (4,0) {};
    \draw [->] (this.north|-this.two east) -- (next);
    \draw [->] (this.north|-this.three west) -- (prev.three east);
  \end{tikzpicture}
  \caption{\label{alg:ditem}En knude i en dobbelthægtet liste.}
\end{figure}

En følge med $n$ indgange fremstilles som en ring af $n+1$ knuder.
I ringen findes en særlig knudeattrap $h$,
\index{liste!knudeattrap|textbf}
som ikke har nogen indgang.
I $h$s efterfølger $h_1$ gemmes følgens første indgang, i $h_1$s efterfølger gemmes følgens anden indgang, osv.
I forgængeren til $h$ gemmes følgens sidste indgang, se fig.~3.2.
Den tomme følge fremstilles af en ring, som kun indeholder attrapknuden $h$.
Herved bliver $h$ sin egen efterfølger og forgænger.
I fig.~3.4 er det vist, hvordan følger kan implementeres med dobbelthægtede lister.
Et objekt af klassen $\List$ indeholder en enkelt knude $h$ som instansvariabel.
Klassens konstruktør initialiserer denne »hovedknude« $h$ på den måde, at den får den uneigentlichen ingang $\bot$ og peger på sig selv som forgænger og efterfølger.
Dette fremstiller den tomme listes begyndelsestilstand.

\begin{figure}
  \[
  \tikzset{ 
    lnode/.style={rectangle split, rectangle split parts=3, draw}, 
    >=stealth'}
  \begin{tikzpicture}[remember picture, shift = (lnode)]
    \node (1) [lnode, minimum width = 1cm] at (0,0) 
    { $e_1$ \nodepart{two} $\bullet$ \nodepart{three} $\bullet$};
    \node (dummy) [dashed,lnode, minimum width = 1cm] at (-2,0) 
    { $\bot$ \nodepart{two} $\bullet$ \nodepart{three} $\bullet$};
    \node (n) [lnode, minimum width = 1cm] at (4,0)
    { $e_n$ \nodepart{two} $\bullet$ \nodepart{three} $\bullet$};
    \draw [->] (1.north|-1.two east) -- +(1,0) node [anchor = west] {$\cdots$};
    \draw [->] (1.north|-1.three west) -- (dummy.three east);
    \draw [->] (dummy.north|-dummy.two east) -- (1);
    \draw [->, rounded corners] (dummy.north|-dummy.three west) -- ++(-1,0) -- ++(0,-.5) -- ++ (8,0) -- ++(0,.5) -- (n.three east);
    \draw [->, rounded corners] (n.north|-n.two east) -- ++(1,0) -- ++(0,1) -- ++(-8,0) -- ++(0,-1) -- (dummy.two west);
    \draw [->] (n.north|-n.three west) -- +(-1,0) node [anchor = east] {$\cdots$};
  \end{tikzpicture}
\]
  \caption{Fremstilling af følgen $\langle e_1,\ldots, e_n\rangle$ vej hjælp af en dobbelthægtet liste. 
  Fremstillingen består af $n+1$ knuder anbragt i en ring:
  En attrapknude $h$, som ikke indeholder nogen indgang, og en knude for hver af følgens $n$ indgange.
  Knuden med indgang $e_i$ er efterfølger til knuden med indgang $e_{i-1}$ og forgænger til  knuden med indgang $e_{i+1}$.
  Attrapknuden sidder mellem de to knuder med ingang $e_n$ og $e_1$.}
\end{figure}

\begin{figure}
\begin{tabbing}
  \textcolor{callout}{/\!/ Fjern $ \langle a,\ldots,b\rangle$ fra sin liste og indføj den efter $t$}\\
  \textcolor{callout}{
    /\!/ $\ldots, a',a,\ldots,b,b',\ldots + \ldots, t,t',\ldots\mapsto\ldots,
    a',b',\ldots + \ldots,t,a,\ldots,b,t',\ldots$
  }\\
  ~~~~\=\kill
  $\Procedure \splice(a,b,t \colon \Handle)$\\
  \>$\Assert$ \emph{$a$ og $b$ tilhører samme liste, $b$ står ikke før $a$, og $t\notin\langle a,\ldots b\rangle$}\\
  \\
  \>   \textcolor{callout}{// klip $\langle a,\ldots,b\rangle$ ud}\\
  \> $a' :=a\rightarrow\prev$\qquad\qquad\qquad\tikz[remember picture] \node (splice1) {};\\
\> $b' :=b\rightarrow\nnext$\\
\> $a'\rightarrow\nnext:=b'$\tikz[remember picture] \node (splice2) {};\\
  \> $  b'\rightarrow\prev:=a' $  \\
\>   \\
  \>  \textcolor{callout}{// indføj $\langle a,...,b\rangle$ efter $t$} \tikz[remember picture] \node (splice3) {}; \\
\>   $t' :=t\rightarrow\nnext$\\
  \> $  b\rightarrow\nnext:=t' $  \\
\> $   a\rightarrow\prev:=t$\tikz[remember picture] \node (splice4) {};\\
\> $ t\rightarrow\nnext:=a $\\
  \> $   t'\rightarrow\prev:=b$ \tikz[remember picture] \node (splice5)  {};
\end{tabbing}
 \tikzset{ 
    lnode/.style={minimum width =.4cm, inner sep = 1pt, rectangle split, rectangle split parts=3, draw, 
    node contents = { \nodepart{two} $\scriptstyle\bullet$ \nodepart{three} $\scriptstyle\bullet$}
    },
    >=stealth',
    illustration/.style = { remember picture, overlay, callout, scale = .5 }
    }
  \begin{tikzpicture}[illustration, shift = (splice1)]
    \node (a_) at (0,0) [label = above:$a'$, lnode];
    \node (a) at (2,0) [label = above:$a$, lnode];
    \node (b) at (6,0) [label = above:$b$, lnode];
    \node (b_) at (8,0) [label = above:$b'$, lnode];
    \draw [->] (a_.north|-a_.three west) -- +(-1,0) node [anchor = east] {$\cdots$};
    \draw [->] (a_.north|-a_.two east) -- (a.two west);
    \draw [->] (a.north|-a.three west) -- (a_.three east);
    \draw [->] (a.north|-a.two east) -- +(1,0) node [anchor = west] {$\cdots$};
    \draw [->] (b.north|-b.three west) -- +(-1,0) node [anchor = east] {$\cdots$};
    \draw [->] (b.north|-b.two east) -- (b_.two west);
    \draw [->] (b_.north|-b_.three west) -- (b.three east);
    \draw [->] (b_.north|-b_.two east) -- +(1,0) node [anchor = west] {$\cdots$};
  \end{tikzpicture}
  \begin{tikzpicture}[illustration, shift = (splice1|-splice2)]
    \node (a_) at (0,0) [lnode];
    \node (a) at (2,0) [lnode];
    \node (b) at (6,0) [lnode];
    \node (b_) at (8,0) [lnode];
    \draw [->] (a_.north|-a_.three west) -- +(-1,0) node [anchor = east] {$\cdots$};
    \draw [->, rounded corners] (a_.north|-a_.two east) -- ++(1,1) -- ++(6,0) -- (b_);
    \draw [->] (a.north|-a.three west) -- (a_.three east);
    \draw [->] (a.north|-a.two east) -- +(1,0) node [anchor = west] {$\cdots$};
    \draw [->] (b.north|-b.three west) -- +(-1,0) node [anchor = east] {$\cdots$};
    \draw [->] (b.north|-b.two east) -- (b_.two west);
    \draw [->, rounded corners] (b_.north|-b_.three west) -- ++(-1,-.5) -- ++(-6,0) -- (a_);
    \draw [->] (b_.north|-b_.two east) -- +(1,0) node [anchor = west] {$\cdots$};
  \end{tikzpicture}
  \begin{tikzpicture}[illustration, shift = (splice1|-splice3)]
    \node (t) at (0,0) [lnode, label = above: $t$];
    \node (a)  at (2,0) [lnode, label = above: $a$];
    \node (b)  at (6,0) [lnode, label = above: $b$];
    \node (t_) at (8,0) [lnode, label = above: $t'$];
    \draw [->] (t.north|-t.three west) -- +(-1,0) node [anchor = east] {$\cdots$};
    \draw [->, rounded corners] (t.north|-t.two east) -- ++(1,1) -- ++(6,0) -- (t_);
%    \draw [->] (a.north|-a.three west) -- (t.three east);
    \draw [->] (a.north|-a.two east) -- +(1,0) node [anchor = west] {$\cdots$};
    \draw [->] (b.north|-b.three west) -- +(-1,0) node [anchor = east] {$\cdots$};
%    \draw [->] (b.north|-b.two east) -- (t_.two west);
    \draw [->, rounded corners] (t_.north|-t_.three west) -- ++(-1,-.5) -- ++(-6,0) -- (t);
    \draw [->] (t_.north|-t_.two east) -- +(1,0) node [anchor = west] {$\cdots$};
  \end{tikzpicture}
  \begin{tikzpicture}[illustration, shift = (splice1|-splice4)]
    \node (t)  at (0,0) [lnode];
    \node (a)  at (2,0) [lnode];
    \node (b)  at (6,0) [lnode];
    \node (t_) at (8,0) [lnode];
    \draw [->] (t.north|-t.three west) -- +(-1,0) node [anchor = east] {$\cdots$};
    \draw [->, rounded corners] (t.north|-t.two east) -- ++(1,1) -- ++(6,0) -- (t_);
    \draw [->] (a.north|-a.three west) -- (t.three east);
    \draw [->] (a.north|-a.two east) -- +(1,0) node [anchor = west] {$\cdots$};
    \draw [->] (b.north|-b.three west) -- +(-1,0) node [anchor = east] {$\cdots$};
    \draw [->] (b.north|-b.two east) -- (t_.two west);
    \draw [->, rounded corners] (t_.north|-t_.three west) -- ++(-1,-.5) -- ++(-6,0) -- (t);
    \draw [->] (t_.north|-t_.two east) -- +(1,0) node [anchor = west] {$\cdots$};
  \end{tikzpicture}
  \begin{tikzpicture}[illustration, shift = (splice1|-splice5)]
    \node (t)  at (0,0) [lnode];
    \node (a)  at (2,0) [lnode];
    \node (b)  at (6,0) [lnode];
    \node (t_) at (8,0) [lnode];
    \draw [->] (t.north|-t.three west) -- +(-1,0) node [anchor = east] {$\cdots$};
    \draw [->] (a.north|-a.three west) -- (t.three east);
    \draw [->] (a.north|-a.two east) -- +(1,0) node [anchor = west] {$\cdots$};
    \draw [->] (b.north|-b.three west) -- +(-1,0) node [anchor = east] {$\cdots$};
    \draw [->] (b.north|-b.two east) -- (t_.two west);
    \draw [->] (t_.north|-t_.two east) -- +(1,0) node [anchor = west] {$\cdots$};
  \end{tikzpicture}
\caption{\label{alg:splice}Operationen \Id{splejs} på lister.}
\end{figure}

Alle grundlæggende listeoperationer bliver gennemført med basisoperationen \emph{splejs}.
\index{splejs@\Id{splejs}|textbf}
\index{liste!splejs@\Id{splejs}|textbf}
Denne er vist i figur~\ref{alg:splice}.
Operationen \emph{splejs} klipper en delliste ud af listen og indføjer den igen bag en angivet målknude.
Dellisten er hertil angivet ved greb $a$ og $b$ på sin første og sidste knude.
For den korrekte udførelse af operationen er det væsentligt, at knuden med greb $b$ kan nås fra den med greb $a$ gennem en følge af efterfølgerpegere, uden at herved møde attrapknuden.
Målknuden $t$, som ligeledes er givet som et greb, kan sidde enten i den samme eller i en anden liste.
I første fald må $t$ selfølgeligt ikke være del af den delliste, som begyder ved $a$ og slutter ved $b$.

Operationen \emph{splejs} ændrer ikke på det totale antale knuder i systemet.
Vi går ud fra, at der findes en speciel liste $\friListe$, som stiller et forråd af ubenyttede knuder til rådighed.
\index{liste!lagerforvaltning}
Når nye indgange skal optages i en følge, tages de nødvendige knuder fra listen \emph{friListe}; når indgange fjernes, bliver det tilsvarende knuder givet tilbage til \emph{friListe}.
\index{liste|fjern@\Id{fjern}}
Funktionen $\Id{sikrFriListe}$ reserverer lagerplads til nye knuder, når dette er nødvendigt.
I opgave 3.3 spørges og i afsnit 3.6 diskuteres hvordan denne funktion kan implementeres.


\begin{figure}
  \small
  \begin{tabbing}
    ~~~~\=~~~~\=\kill
    $\Class \Id{Liste} \Of \Id{Element}$\\
    \> \textcolor{callout}{/\!\!/ Knude $h$ er forgænger til den første og efterfølger til den sidste knude.}\\
    \> $h=\begin{pmatrix}\bot\\\this\\\this\end{pmatrix}\colon\Id{Knude} $\\
    \\
    \> \comment{Funktioner for enkel adgang.}\\
    \> $\Function\Id{hoved}\colon\Handle\quad \return \adressof h$
    \qquad \comment{Position før alle indgange}\\
    \\
    \> $\Function\Id{tom}\colon\{0,1\} \quad\return h.next = \this$\\
    \> $\Function\Id{første}\colon\Handle \quad\Assert \neg\Id{tom}; \return h.\Id{næste} $\\
    \> $\Function\Id{sidste}\colon\Handle \quad\Assert \neg\Id{tom}; \return h.\Id{forrige} $\\
    % TODO removed semicolons after function head
    \\
    \>\comment{Flytning af enkelte indgange inden for samme følge.}\\
    % TODO added "enkelte", removed brackets
    \>\comment{$\langle \ldots, a,b,c,\ldots,a',c',\ldots\rangle \mapsto
   \langle \ldots, a,c,\ldots,a',b,c',\ldots\rangle$.}\\
    \>$\Procedure \Id{flytEfter}(b,a'\colon\Handle) \quad\Id{splejs}(b,b,a')$\\
    \>$\Procedure \Id{flytForrest}(b\colon\Handle) \quad\Id{flytEfter}(b,\Id{hoved})$\\
    \>$\Procedure \Id{flytBagerst}(b\colon\Handle) \quad\Id{flytEfter}(b,\Id{sidste})$\\
    \\
    \>\comment{Indsættelse og fjernelse af enkelte indgange.}\\
    % TODO added "enkelte"
    \>\comment{$\langle \ldots, a,b,c,\ldots\rangle \mapsto
   \langle \ldots, a,c,\ldots\ldots\rangle$.}\\
    \>$\Procedure \Id{fjern}(b\colon\Handle) \quad\Id{flytEfter}(b,\Id{friListe}.\Id{hoved})$\\
    \>$\Procedure \Id{fjernForrest} \quad\Id{fjern}(\Id{første})$\\
    \>$\Procedure \Id{fjernBagest} \quad\Id{fjern}(\Id{sidste})$\\
    \\
    \>\comment{$\langle \ldots, a,b,\ldots\rangle \mapsto
   \langle \ldots, a,e,b,\ldots\ldots\rangle$.}\\
    \>$\Function \Id{indsætEfter}(x\colon\Element, a\colon\Handle)\colon \Handle$ \\
    \>\>$\Id{sikrFriListe}$\hspace{2cm}\=\comment{Sikr at $\Id{friListe}$ er ikketom. Se opg. 3.3.}\\
    \>\>$a':=\Id{friListe}.\Id{første}$\>\comment{Skaf ny knude $a'$ til at indeholde $x$,}\\
    \>\>$\Id{flytEfter}(a',a)$\>\comment{indsæt den på rette plads,}\\
    \>\>$a'\rightarrow e:= x$\>\comment{og fyld den med det rette indhold.}\\
    \>\>$\return a'$\\
    \\
    \>$\Function\Id{indsætFør}(x\colon\Element; b\colon\Handle)\colon\Handle \quad \return\Id{indsætEfter}(e, b\rightarrow\Id{forrige})$\\
    % TODO wft er pred(b)
    \>$\Procedure\Id{tilføjForrest}(x\colon\Element) \quad \Id{indsætEfter}(x, \Id{hoved})$\\
    \>$\Procedure\Id{tilføjBagerst}(x\colon\Element) \quad \Id{indsætEfter}(x, \Id{sidste})$\\
    \\
    \>\comment{Behandling af hele lister.}\\
    \>\comment{$(\langle a, \ldots, b\rangle, \langle c,\ldots,d\rangle) \mapsto
		(\langle a, \ldots, b, c,\ldots,d\rangle, \langle \,\rangle)$}\\
    \>$\Procedure\Id{sammenføj}(L'\colon\Id{Liste})\quad 
    \Id{splejs}(L'.\Id{første}, L'.\Id{sidste},\Id{sidste})$\\
    \\
    \>\comment{$\langle a, \ldots, b\rangle \mapsto \langle \,\rangle$}\\
    \>$\Procedure\Id{tøm}\quad \Id{friListe}.\Id{sammenføj}(\this)$
  \end{tabbing}
  \caption{\label{alg:dlist}
  Konstanttidsoperationer på dobbelthægtede lister.}
\end{figure}
På grundlag af disse beslutninger kan vi nu implementere mange nyttige operationer på lister som funktioner bestående af en enkelt linje, som alle kræver konstant tid.
Idet \emph{splejs} er så magtfuld, kan vi sågar behandle dellister af vilkårlig længde i konstant tid.
I fig.~\ref{alg:dlist} finder man mange eksempler på den slags operationer.
For at afgøre, om en liste er tom, behøver man bare at undersøge, om $h$ er sin egen efterfølger.
\index{liste!tom@\Id{tom}}
Når en liste ikke er tom, finder man dens første og sidste indgang som efterfølger hhv. forgænger til $h$.
\index{liste!fzrste@\Id{første}}
\index{liste!sidste@\Id{sidste}}
For at flytte en knude $b$ til positionen efter en knude $a'$, klipper man bare den delliste, som begynder og ender ved $b$, ud og føjer den ind efter $a$.
\index{liste!flytning af enkelt knude|textbf}
Dette gøres med kaldet \emph{splejs}$(b,b', a)$.
En indgang kan flyttes til listens første eller sidste plads ved at den tilsvarende knude sættes bag ved hovedknuden $h$ eller bag ved den sidste indgang.
En indgang $b$ slettes fra en liste ved at flytte den tilsvarende knude til \emph{friListe}.
For at indsætte en ny indgang $e$, 
\index{liste!indsæt@\Id{indsæt}}
udtages en ny knude fra \emph{friListe}, indgangen skrives i knuden, og knuden anbringes på den pågældende plads.

\begin{exerc}[Alternativ listeimplementation]
  Diskuter en alternativ implementation af klassen $\List$, som ikke behøver attrapknuden $h$.
  I stedet gemmer man i listeobjektet en peger på den første knude i listen.
  Positionen før den første knuder indkodes som en nulpeger.
  Grænsefladen og de asymptotiske udførelsestider bør forblive uændrede.
  Nævn mindst en fordel og en ulempe ved denne implementation i sammenligning med den i teksten beskrevne implementation.
\end{exerc}

Attrapknuden kan også være nyttig for andre operationer. 
Betragt fx problemet at finde den næste forekomst af indgangen $x$ fra en givet knude \emph{fra}.
\index{liste!find@\Id{find}}
  Hvis $x$ ikke forekommer, skal attrapknudens greb \emph{hoved} returneres.
  Attrapknuden bruges her som \emph{vogter}.
\index{vogterknude|textbf}
  En vogterknude i en datastruktur er en knude, der sørger for, at en vis løkke terminerer.
  Ved søgning i en liste gemmer vi den søgte nøgle $x$ i attrapknuden.
  Herved opnår vi, at $x$ findes i listestrukturen, så søgningen garanteret terminerer.
  Søgningen ender enten i en egentlig knude eller i attrapknuden, alt efter om $x$ fandtes i den oprindelige liste eller ej.
  Dette trick gør, at man ikke i hvert skridt behøver at undersøge, om søgninen har nået enden af listen.
  Dette kan øge effektiviteten af søgningen betrageligt:

 \begin{quote}
  \begin{tabbing}
    ~~~~\=~~~~\=\kill
    $\Function\Id{findNæste}(x:\Element; \Id{fra}\colon\Handle)\colon \Handle$\\
    \>$h.e:= x\qquad\comment{Vogterknude}$\\
    %TODO = in orig, not :=
    \>$\while\Id{fra}\rightarrow e\neq x \ddo$\\
    \>\>$\Id{fra} :=\Id{fra}\rightarrow\Id{næste}$\\
    \>$h.e:=\bot$\\
    $\return\Id{fra}$
  \end{tabbing}
 \end{quote}
 
 %TODO fucking mange index-kald

  \begin{exerc}
    \index{liste!ombytning af delfølger|textbf}
    Implementer en procedure \emph{ombyt}, som i konstant tid ombytter to delfølger, dvs. transformerer følgerne
    $(\langle \ldots, a',a,\ldots, b,b',\ldots\rangle,
    \langle \ldots, c',c,\ldots, d,d',\ldots\rangle)$
    til følgerne
    $(\langle \ldots, a',c,\ldots, d,b',\ldots\rangle,
    \langle \ldots, c',a,\ldots, b,d',\ldots\rangle)$.
    Er \emph{splejs} et specialtilfælde af \emph{ombyt}?
  \end{exerc}


  \begin{exerc}[Lagerhåndtering.]
    \index{liste!lagerhåndtering|textbf}
    Angiv en implementation af funktionen $\Id{sikrFriListe}$, som kaldes fra $\Id{indsætEfter}$, som angivet i fig.~3.5.
    Funktionen skal sikre, at $\Id{friListe}$ ikke er tom, og i givet tilføje nye knuder.
    Fordi den kan have en negativ indvirkning på udførelsestiden at kalde den af programmeringssproget til rådighed stillede primitoperation $\allocate$ for hver knude separat, bør funktionen stille knuder til rådighed i større »portioner«.
    Udførelsestiden for $\Id{sikrFriListe}$ i værste fald bør være uafængig af portionernes størrelse.
    \emph{Vink:} 
    Benyt ved siden af $\Id{friListe}$ en lille række, som indeholder ubenyttede (og endnu ikke initialiserede) knuder.
  \end{exerc}

  \begin{exerc}
    \index{liste!rotation|textbf}
    Implementer en algoritme, som i konstant tid udfører en højrerotation
    $\langle a, \ldots, b,c\rangle \mapsto \langle c, a,\ldots, b\rangle$
    af en position i en følge.
    Giv en mere generel algoritme for rotationen
    $\langle a, \ldots, b,c,\ldots, d\rangle \mapsto \langle c,\ldots, d, a,\ldots, b\rangle$.
    Positionen af $b$ er givet som greb; udførelsestiden bør igen være konstant.
  \end{exerc}

  \begin{exerc}
    Funktionen $\Id{findNaeste}$ kører på grund af anvendelsen af en vogterknude hurtigere end en implementation, som tester ved hvert løkkegennemløb, om enden af listen er nået.
    Men hvor stor er gevinsten i beregningstid?
    Hvilken relativ forskel i køretid ville du forvente i følgende situationer? 
    (a) I en kort liste med 100 indgange gennemføres mange søgninger;
    (b) i en lang liste med fx $10\,000\,000$ indgange gennemføres en enkelt søgning.
    Hvorfor afhænger den relative hastighedsforskel af listelængden?
  \end{exerc}
  
  \subsection*{Vedligeholdelse af listelængden}
  \index{liste!størrelse@\Id{størrelse}|textbf}

  % TODO længde?

  I vores enkle listedatastruktur kan vi ikke afgøre længden af listen i konstant tid.
  Det kan man afhjælpe ved at bruge en instansvariabel $\size$, som akualiseres hver gang, antallet af listeelementer bliver ændret.
  Nu har alle operationer, som kan ændre flere lister, brug for at kende længderne af de berørte lister, i modsætning til synssættet med basisfuktionen $\splice$, som kunne nøjes med at kende til grebene til de berørte knuder.
  Betragt fx følgende kodestump, som udtager en knude $a$ fra en liste $L$ og flytter den til positionen $a'$ i en liste $L'$ og akualiserer de pågældende instansvariabler for længderne.

  \begin{tabbing}
~~~~\=\kill
    $\Procedure \moveAfter (a, a'\colon \Handle ; L,L'\colon \List)$\\
    \> $\splice(a,a,a'); \quad L.\size--; \quad L.\size++$
 \end{tabbing}
 
 Vedligeholdelsen af listelængden
 \index{liste!længde}
 kræver i sin tur ændringer i implementationen af andre listeoperationer.
 Når knuder flyttes fra en liste til en anden, skal man kende til de berørte lister.
 Et større problem opstår, når -- som ved operationen $\splice$ -- hele listeafsnit flyttes, fordi konstant udførelsestid nu ikke længere kan garanteres.
 Den næste opgave viser et muligt kompromis.

 \begin{exerc}
   Konstruer på grundlag af en dobbelthægtet liste en listedatatype, som tillader forskydning af dellister fra en liste til en anden i konstant tid, samt bestemmelse af længden af en liste, forudsat at listen ikke har deltaget i dellisteoperationer med andre lister siden den sidste adgang til $\size$.
   Hvis den slags dellisteoperationer blevet udført, skal værdien af $\size$ først beregnes påny, når den skal bruges.
 \end{exerc}
 
 \begin{exerc}
   \index{liste!sammenføjning} 
   Forklar, hvordan operationerne $\Id{fjern}$, $\Id{indsætEfter}$ og $\Id{sammenføj}$ skal modificeres, hvis man altid vil have den aktuelle listelængde til rådighed.
 \end{exerc}

\subsection{Enkelthægtede lister}
\label{s:slist}
\index{liste!enkelthægtet|textbf}

Programmeringen med dobbelthægtede lister bliver gjord meget nemmere af, at hver knude har både forlæns- og baglænsreferencer.
Enkelthægtede lister er de dobbelhægtedes slankere søskende.
Knudetypen i enkelthægtede kalder vi $\Id{EElement}$.
Typen $\Id{EElement}$ undgår forgængerpegeren og gemmer kun en peger på efterfølgeren.
Det gør, at  enkelthægtede lister er mere pladsbesparende og ofte også hurtigere end deres dobbelhægtede søskende.
Man må betale en vis pris for denne forbedring:
Mange operationer kan nu ikke længere udføres i konstant tid og kan ikke længere stilles til rådighed med samme generalitet.
For eksempel er det nødvendigt at kende en $\Id{EElement}$-knudes forgænger, for at fjerne den.

Vi følger den grundlæggende tilgangsmåde for implementationen af dobbelhægtede lister.
Vi danner ringe af knuder, hvor hver liste svarer til en ring.
Hvert objekt af typen $\Id{EListe}$ har en attrapknude $h$ af typen $\Id{EElement}$, som er forgænger til den første egentlige knude og efterfølger til den sidste egentlige knude.
Mange listeoperationer lader sig stadig gennemføre, hvis vi ændrer grænsefladen bare en smule.
For eksempel har $\Id{splejs}$-operationen nu behov for at kende \emph{forgængeren} til den første knude af den delliste, som skal flyttes:
\index{liste!splejs@\Id{splejs}} 

\begin{quote}
  \begin{tabbing}
    ~~~~\=\kill
    \comment{$(\langle \ldots,a',a,\ldots,b,b',\ldots\rangle,
    \langle\ldots, t,t',\ldots,\rangle) \mapsto
    (\langle \ldots,a',b',\ldots\rangle,
    \langle\ldots, t,a,\ldots,b,t',\ldots,\rangle)$}\\
    $\Procedure\Id{splejs}(a',b,t\colon \Id{EGreb})$\\
    \>$\begin{pmatrix}
      a'\rightarrow\Id{næste}\\
      t\rightarrow\Id{næste}\\
      b\rightarrow\Id{næste}
    \end{pmatrix} := 
    \begin{pmatrix}
      b\rightarrow\Id{næste}\\
      a'\rightarrow\Id{næste}\\
      t\rightarrow\Id{næste}
    \end{pmatrix} 
    $
  \end{tabbing}
\end{quote}

På en lignende måde skal operationen $\Id{findNæste}$ med argument $x$ ikke bruge et greb på den knude i listen, der indeholder $x$, men samme knudes \emph{forgænger}, så den fundne indgang stadig kan fjernes.
\index{liste!findNæste@\Id{findNæste}} 
Tilsvarende vil funktionen $\Id{findNæste}$ begynde sin søgning ikke ved den angivne knude, men først ved dennes efterfølger.
En nyttig tilføjelse til enkelthægtede lister er en peger på den sidste listeknude, hvilken gør det muligt at udføre operationen $\Id{indsætBagest}$ i konstant tid.
\index{liste!indsætBagest@\Id{indsætBagest}} 

\begin{exerc}
  %TODO shitload of index
  Implementer klassene $\Id{EGreb}$, $\Id{EElement}$,$\Id{EListe}$ på basis af enkelthægtede lister svarende til
  $\Id{Greb}$, $\Id{Element}$ og $\Id{Liste}$.
  Vis, at alle de følgende operationer kan implementeres, så de tager konstant tid.
  Hertil skal operationerne $\Id{hoved}$, $\Id{første}$, $\Id{sidste}$, $\Id{tom}$, $\Id{fjernForrest}$, $\Id{indsætForrest}$, $\Id{indsætForrest}$, $\Id{indsætEfter}$, $\Id{sammenføj}$ og $\Id{gørTom}$ have den samme grænseflade som for dobbelhægtede lister;
  hvorimod operationerne $\Id{flytEfter}$, $\Id{flytTilForrest}$, $\Id{flytTilBagest}$, $\Id{fjern}$, $\Id{fjernForrest}$ og $\Id{findNæste}$ har brug for ændringer i grænsefladen.
\end{exerc}

Vis skal i det videre forløb se en række eksempler på brugen af enkelthægtede lister, fx spredetabeller i afsnit 4.1 og flettesortering i afsnit 5.2.
Enkelthægtede lister kan også bruges til realiseringen af frie knuder i  lagerstyringsprogrammer -- selv for knuder i dobbelhægtede lister!

\section{Ubegrænsede rækker}
\label{s:array}
\index{række!ubegrænset|textbf}
\index{ubegrænset række|textbf}

Vi betragter nu en datastruktur for rækker, som ved siden af den indeksbaseret adgangsoperation $[\cdot]$ tillader operationerne $\Id{indsætBagest}$, $\Id{fjernBagest}$ og $\Id{størrelse}$ som følger:
\index{række!fjernBagest@\Id{fjernBagest}|textbf}
\index{række!indsætForrest@\Id{indsætForrest}|textbf}
\index{række!størrelse@\Id{størrelse}|textbf}
\index{række!adgang@adgang med $[{}\cdot{}]$|textbf}
\begin{align*}
  \langle e_1,\ldots, e_n\rangle.\Id{indsætBagest}(e) &=
\langle e_1,\ldots, e_n, e\rangle\,,\\
  \langle e_1,\ldots, e_n\rangle.\Id{fjernBagest}(e) &=
  \langle e_1,\ldots, e_{n-1}\rangle\,, \qquad (\text{for } n\geq 1)\,,\\
  \Id{størrelse}(\langle e_1,\ldots, e_n\rangle) &= n\,.
\end{align*}

Hvorfor en den slags ubegrænsede rækkestrukturer vigtige?
I mange situationer ved man ikke i forvejen, hvor stor en række skal blive.
Et typisk eksempel er følgende:
Man ønsker at implementere unixkommandoen $\Id{sort}$, som sorterer linjerne i et arkiv.
Man begynder med at læsningen arkivets enkelte linjer ind i en række av linjer, sortere rækker in hovedlageret og udlæser til sidst den sorterede række.
Med ubegrænsede rækker er dette meget enkelt, med begrænsede rækker skulle man derimod læse hele arkivet to gange:
En første gang bare for at bestemme antallet af linjer, og en anden gang for at læse linjerne ind i den nu parate række af den rette størrelse.

Vi skal nu undersøge, hvordan man implementerer ubegrænsede rækker.
Vi simulerer en ubegrænset række $u$ med $n$ indgange ved at dynamisk at vedligholde en begrænset række $b$ med $w$ pladser, hvor $w\geq n$.
De første $n$ pladser i $b$ bruges til at gemme indgangende fra $u$.
De sidste $w-n$ pladser i $b$ er ubenyttede.
Så længe $w>n$ gælder, kan operationen $\Id{indsætBagest}$ nøjes med at øge pejlvariablen $n$ og bruge den ubenyttede plads i $b$ til den nye indgang.
Når $w=n$ gælder, fører det næste kald af $\Id{indsætBagest}$ til, at der stilles en ny begrænset række $b'$ til rådighed, som er en konstant faktor (fx to gange) større end $b$.
For at genetablere invarianten, at $u$ er gemt i $b$, bliver $b$s indhold kopieret til de første $n$ pladser af $b'$, således at den gamle række $b$ kan frigøres.
Endelig flyttes pegeren på $b$ til at pege på $b'$.
Det er endnu nemmere at fjerne den sidste indgang (ved $\Id{fjernBagest}$), fordi der aldrig er fare for, at $b$ bliver for lille.
Det kan dog tænkes, at vi spilder alt for meget lagerplads ved at tillade, at $b$ er meget større end nødvendig.
Den overskydende lagerplads kan holdes lille, ved krympe $b$, når $n$ bliver for lille i forhold til $w$.
I fig.~3.6 vises pseudokoden for en klasse, som realiserer ubegrænsede rækker.
Den underliggende række vokser og krymper ved hjælp af proceduren $\Id{realloker}$.
\index{række!vokse}
\index{række!krympe}
\index{række!realloker@\Id{realloker}|textbf}
Vores implementation benytter konstanterne $\alpha$ og $\beta$ med $\beta= 2$ og $\alpha=4$.
Når den aktuelt anvendte, begrænsede række bliver for lille, erstattes den af en ny række, som er $\beta$ gange større;
når den udnyttede del af rækken bliver $\alpha$ gange mindre end dens størrelse, erstattes den med en ny række af størrelse $n/\beta$.
Forneden før vi rede for vores valg af værdierne for $\alpha$ og $\beta$.

\begin{figure}
  \begin{tabbing}
    ~~~~\=~~~~\=~~~~\=\kill
    $\Class \Id{URække} \Of \Id{Element}$\\
    \>$\Constant \beta = 2\colon \RR_{>0}$\\
    \>$\Constant \alpha = 4\colon \RR_{>0}$\\
    \>$w=1\colon\NN$\\
    \>$n=0\colon\NN$\\
    \>$\Invariant n\leq w\leq \alpha n\text{ eller }n=0\text{ og }w\leq\beta$\\
    \>$b\colon\Id{Række} [1..w] \Of \Id{Element}$\\
    \\
    \>$\mathbf{Operator} [i\colon\NN]\colon\Id{Element}$\\
    \>\>$\Assert 1\leq i\leq n$\\
    \>\>$\return b[i]$\\
    \\
    \>$\Function \Id{størrelse}\colon\NN\quad\return n$\\
    \\
    \>$\Procedure\Id{tilføjBagest}(e\colon\Id{Element})$\\
    \>\>$\iif n=w\tthen$\\
    \>\>\>$\Id{realloker}(\beta n)$\\
    \>\>$b[n+1]:=e$\\
    \>\>$n++$\\
    \\
    \>$\Procedure\Id{fjernBagest}$\\
    % whatthefuck!
    \>\>$\Assert n> 0$\\
    \>\>$n--$\\
    \>\>$\iif \alpha n\leq w\wedge n> 0\tthen$\\
    \>\>\>$\Id{realloker}(\beta n)$\\
    \\
    \>$\Procedure\Id{realloker}(w'\colon\NN)$\\
    \>\>$w:=w'$\\
    \>\>$b':= \allocate\Id{Række}[1..w']\Of\Id{Element}$\\
    \>\>$(b'[1],\ldots,b'[n]):=(b[1],\ldots,b[n])$\\
    \>\>$\dispose b$\\
    \>\>$b:=b'$
  \end{tabbing}
  \caption{\label{fig: UArray pseudocode}%
  Pseudokode for ubegrænsede rækker.}
\end{figure}

\subsection{Amortiseret analyse af ubegrænsede rækker: det globale perspektiv}
\index{algoritmeanalyse!amortiseret!ubegrænset række}

Vores implementation af ubegrænsede rækker følger princippet om at være »almindeligvis hurtig«.
\index{algoritmekonstruktion!almindeligvist@»almindeligvis hurtig«}.
Adgangen til en indgang ved brug af $[\cdot]$ er lige så hurtig som for en begrænset række.
Intuitivt betragtet vil $\Id{indsætBagest}$ og $\Id{fjernBagest}$ ligeledes være hurtige »i normalfaldet« -- datastrukturen skal jo bare aktualisere pejlevariablen $n$.
Alligevel vil visse indsættelser og fjernelser kræve tid $\Theta(n)$.
Vi skal nu vise, at disse dyre operationer er sjældne, og at hver følge af $m$ operationer begyndende fra en tom række kan udføres i tid $O(m)$.

\begin{lem}
  \label{lem: UArray amortised time}
  Betragt en ubegrænset række $u$, som er tom i begyndelsen.
  Hver følge $\sigma=\langle \sigma_1,\ldots,\sigma_m\rangle$ af $m$ $\Id{indsætBagest}$- og $\Id{fjernBagest}$-operationer på $u$ bliver udført i tid $O(m)$.
\end{lem}

Lemma~\ref{lem: UArray amortised time} er alt andet end trivielt.
En lille og uskyldigt udseende ændring i implementationen gør påstanden falsk.

\begin{exerc}
  Din afdelingsleder opfordrer dig til at ændre forvalget af $\alpha$ til $\alpha=2$.
  Hans argumentet er, at det er spild af plads at vente med at erstatte rækken med en mindre, når den er kvartfuld.
  Han foreslår derfor at krympe rækken, så snart $n\leq \frac{1}{2}w$.
  Overbevis ham om, at det er en dårlig ide.
  Angiv hertil in følge af $m$ $\Id{indsætBagest}$- og $\Id{fjernBagest}$-operationer, som skulle kræve tie $\Theta(m^2)$, hvis du fulgte hans råd.
\end{exerc}

Lemma~\ref{lem: UArray amortised time} er et udsagn om de »amortiserede« omkostninger ved en følge af $\Id{indsætBagest}$- og $\Id{fjernBagest}$-operationer.
Enkelte operationer kan være dyre, men hele følgen af $m$ operationer koster $O(m)$.
Hvis vi deler den samlede omkostning for alle operationer i følgen $\sigma$  med antallet af operationer, får vi en konstant.
Det udtrykker vi på følgende måde:
Den \emph{amortiserede omkostning} af én operation er konstant.
Betydningen, som vi her tildeler ordet »amortiseret« ligner den hverdagssproglige brug, men undgår en fælde, som ofte optræder i forbindelse med forestillinger om amortisering.
Måske kan læseren genkende følgende argument?
Bjarne siger: »Fra og med i dag cykler jeg på arbejde hver dag, så derfor har jeg råd til en luksuscykel.
I det lange løb er udgiften per køretur meget lille -- investeringen amortiserer sig altså.«
Hvad kommer til at ske i virkeligheden?
Bjarne køber den dyre cykel, det regner, og alle gode forsæt er glemt; den dyre cykel står i et hjørne, og ingenting er blevet amortiseret.
I modsætning hertil vil vi planlægge vores datastrukturer sådan, at alle store udgifter altid retfærdiggøres af opsparinger i fortiden i stedet for håbet om fremtidig nøjsomhed.
I vores eksempel om transport til og fra arbejde kunne det se sådan ud:
Astrids mål er egentlig at køre på arbejde i en luksuslimusine.
Hun køber den dog ikke på den første dag, men går derimod til fods og lægger hver dag et fast beløb til side.
Efter et stykke tid har hun nu råd til en cykel.
Hun fortsætter med at spare og har snart råd til en lille bil, og efter endnu længere tid den ønskede luksusbil.
Hver udgift er dækket af tidligere opsparing og er amortiseret i samme øjeblik, den sker.
Med dette amortiserede udgiftsbegreb kan vi formulere lemma~\ref{lem: UArray amortised time} endnu mere elegant.
Den elegante formulering tillader også en mere præcis sammenligning af forskellige datastrukturer.

\begin{cor}
  Ubegrænsede rækker understøtter operationen $[\cdot]$ i konstant tid i værste fald og operationerne $\Id{indsætBagest}$ og $\Id{fjernBagest}$ i amortiseret konstant tid.
\end{cor}

\begin{proof}[Bevis for lemma~\ref{lem: UArray amortised time}]
  For at etablere lemmaet benytter vi den såkaldte \emph{bankkontometode}.
Denne er ækvivalent til \emph{potentialemetoden}, som er nærmere forklaret forneden.
\index{algoritmeanalyse!amortiseret!bankkontometode|textbf}%
\index{algoritmeanalyse!amortiseret!potentialemetode}
\index{potentialemetode|sieheunter{algoritmeanalyse, amortiseret}} 
Til vores datastruktur tænker vi os en bankkonto, som til hver tid indeholder et vist beløb, som aldrig må være negativt.
For hver $\Id{tilføjBagest}$- og $\Id{indsætBagest}$-operation skal der indbetales vist (konstant) beløb på kontoen.
Vi kalder vores valuta for mønter.
\index{algoritmeanalyse!amortiseret!mønt|textbf}
Det er åbenbart, at indgangskopieringen i proceduren $\Id{realloker}$ er den eneste aktion i programmet i fig.~\ref{fig: UArray pseudocode}, som medfører mere end en konstant omkostning.
Nærmere betegnet kaldes proceduren altid med parametværdien $w'=2n$ og skal så kopiere eller flytte $n$ indgange.
Grundideen er nu, at lade opsparingen på bankkontoen stå for omkostningen ved denne dyre operation $\Id{realloker}$.
Lad os bestemme, at en enkelt mønt kan finansiere omkostningen ved at flytte en enkelt indgang fra $b$ til $b'$.
Det betyder, at vi kan nøjes med at hæve $n$ mønter fra kontoen for et kald af $\Id{realloker}$.
Vi vedtager at indbetale to mønter for hvert kald af $\Id{indsætBagest}$ og en mønt for hvert kald af $\Id{fjernBagest}$.
De står tilbage at vise, at disse indbetalinger rækker til at dække de udbetalinger, som medføres af kaldene til $\Id{realloker}$.

Det første kald til $\Id{realloker}$ sker, når rækken indeholder én indgang, og den næste indgang bliver tilføjet.
For den allerede eksisterende indgang blev der indbetale to mønter, og det er rigeligt til at dække den ene mønt, som $\Id{realloker}$-kaldet koster.
Efter et vilkårligt kald til $\Id{realloker}$ i det videre forløb indeholder rækken $w$ pladser, hvoraf $n=\frac{1}{2}w$ er benyttede og $\frac{1}{2}w$ står tomme.
Ved det næste kald af $\Id{realloker}$ gælder ente $n=w$ eller $4n\leq w$.
I det første tilfælde må der være tilkommet mindst $\frac{1}{2}w$ i mellemtiden; for hver af disse blev der indbetalt to mønter.
Derfor må der være mindst $w$ mønter på kontoen, hvilket er nok til at dække udbetalingen til næste kald af $\Id{realloker}$.
I det andet tilfælde blev der fjernet mindst $\frac{1}{2}w-\frac{1}{4}w=\frac{1}{4}w$ indange fra rækken; for hver af disse operationer blev der indbetalt en mønt.
Derfor er der mindst $\frac{1}{4}w$ mønter på kontoen, hvilket er nok til at dække udgiften for et kald af $\Id{realloker}$, som flytter $\frac{1}{4}w$ indgange og hertil hæver $\frac{1}{4}w$ mønter.
Hermed er lemma~\ref{lem: UArray amortised time} bevist.
\end{proof}
\begin{exerc}\label{exerc: alpha beta costs}
  Ændr beviset for lemma~\ref{lem: UArray amortised time} således, at den gælder for generelle værdier af $\alpha$ og $\beta$.
  Hertil kræves, at et kald af $\Id{indsætBagest}$ koster $\beta/(\beta-1)$ mønter, og et kald af $\Id{fjernBagest}$ koster $\beta/(\alpha-\beta)$ mønter.
  Når $n'$ opfylder ligningen $w=\beta n'$, ender hvert kald af $\Id{realloker}$ med netop $n'$ optagede pladser og $(\beta-1)n' = ((\beta-1)/\beta)w$ ledige pladser.
  Når næste kald af $\Id{realloker}$ sker, gælder enten $n=w$ eller $\alpha n\leq w$.
  Vis, at der er tilstrækkeligt mange mønter på kontoen i begge tilfælde.
\end{exerc}

Amortiseret analyser er et nyttigt redskap med mange anvendelser.
Derfor kan det betale sig at lære yderligere metoder til at bevise den slags udsagn.
Vi skal se nærmere på to variationer over beviset for lemma~\ref{lem: UArray amortised time}.

Hidtil har vi anslået to mønter for hvert $\Id{indsætBagest}$- og en mønt for hvert $\Id{fjernBagest}$-kald.
Alternativt kunne vi have regnet med tre mønter for $\Id{indsætBagest}$ og ingen mønter for $\Id{fjernBagest}$.
Afregningen er ligetil:
Af de tre mønter bruges de to til at finansiere en indgangs tilføjelse og den tredje til dens fjernelse, som altså er blevet betalt lang tid i forvejen.

\begin{exerc}[Fortsættelse af opg.~\ref{exerc: alpha beta costs}]
  Vis, at det er nok med en afgift på $\beta(\beta-1)+\beta)(\alpha-\beta)$ mønter for hver $\Id{indsætBagest}$-operation.
  Bestem (afhængigt af $\beta$) en værdi for $\alpha$, som sikrer $\beta/(\alpha-\beta)\leq 1/(\beta-1)$ hhv. $\beta/(\alpha-\beta)\leq \beta/(\beta-1)$.
\end{exerc}


\subsection{Amortiseret analyse af ubegrænsede rækker: det lokale perspektiv}

[TODO]

\subsection{Amortiseret analyse af binære tællere}

[TODO]


\section{Amortiseret analyse *}

[TODO]

\section{Stakke og køer}
\label{s:stack}

Ofte bruger følger brugt på en temmelig begrænset måde.
Lad os betynde med to eksemper fra tiden før datamater.
En sagsbehandler kan arbejde på følgende måde:
Hun har en \emph{stak} af ubehandlede sagsmapper på sit skrivebord.
\index{stak}
Nyt sagsmapper bliver lagt på toppen af stakken.
Når sagsbehandleren vil behandle en sag, vælger hun den mappe, der ligger øverst på stakken.
Denne »datastruktur« af sagsmapper er nem at bruge; selvom den kan ske, at visse mapper længere nede i stakken ender med at forblive ubehandlede i lang tid $\ldots$.
I terminologien for klassen $\Id{Liste}$ fra afs.~3.1 er en \emph{Stak} en følge, som kun bruger operationerne $\Id{tilføjBagest}$, $\Id{fjernBagest}$, $\Id{sidste}$ og $\Id{tom}$.
Når vi taler om stakke, forkorter vi gerne de første tre operationer til $\Id{stak}$,
\footnote{O.a.: I mange fremstillinger og programbiblioteker dækker funktionen $\Id{afstak}$ over en kombination af $\Id{sidste}$ og $\Id{fjernBagest}$, som både returnerer det sidste element (dvs. staktoppen) og fjerner det fra stakken.}
$\Id{afstak}$ og $\Id{top}$.
\index{stak!stak@\Id{stak}|textbf}
\index{stak!afstak@\Id{afstak}|textbf}
\index{stak!top@\Id{top}|textbf}
Andre betegnelser for stak er \emph{sifu-liste} eller \emph{sifu-kø} (»sidst ind, først ud«) og \emph{kælderlager}.

En anden opførsel kan man se, når folk står i kø hos bageren:
Kunder stiller sig bagest i køen og forlader en forrest, når de når frem til disken.
Sådan en følge kaldes \emph{kø}, \emph{fifu-liste} eller \emph{fifu-kø} (»først ind, først ud«).
\index{kø|textbf}
I terminologien for klassen $\Id{Liste}$ fra afs.~\ref{s:list} er en kø en følge, som kun bruger operationerne $\Id{tilføjBagest}$, $\Id{fjernForrest}$, $\Id{første}$ og $\Id{tom}$, hvoraf de første kan omdøbes til $\Id{kø}$ og $\Id{afkø}$%
\index{kø!kø@\Id{kø}|textbf}
\index{kø!afkø@\Id{afkø}|textbf}
\index{kø!første@\Id{første}|textbf}
\footnote{O.a.: I mange fremstillinger og programbiblioteker dækker funktionen $\Id{afkø}$ over en kombination af $\Id{første}$ og $\Id{fjernForrest}$, som både returnerer det første element i køen og fjerner det fra køen.}.

En mere almen \emph{dobbeltkø} 
\index{dobbeltkø|textbf}
stiller operationerne
$\Id{første}$,
$\Id{sidste}$,
$\Id{tilføjForrest}$, 
$\Id{tilføjBagest}$, 
$\Id{fjernForrest}$,
$\Id{fjernBagest}$,
og $\Id{tom}$ til rådighed.
Dobbeltkøen kan observeres hos bageren, når en ubehagelig person stiller sig forrest i køen eller den, eller den person, der står bagest i køen bliver træt af at vente og forlader butikken.
I fig~.\ref{fig:queues} er funktionaliteten af datastrukturerne stak, kø og dobbeltkø vist skematisk.


\begin{figure}
  \[
  \begin{tikzpicture}[scale = .5]
    \node (0) at (0,0) [draw, fill=myblue, rectangle, inner sep = 4pt] {};
    \node (0) at (1,0) [draw, fill=myblue, rectangle, inner sep = 4pt] {};
    \node (0) at (2,0)  {$\cdots$};
    \node (0) at (3,0) [draw, fill = myblue, rectangle, inner sep = 4pt] {};
    \draw (3.5, .5) -- (-.5, .5) -- (-.5,-.5) -- (3.5, -.5);
    \node at (1.5,1) [callout] {\emph{stak}};
    \draw [->, thick] (7,0)--(5,0) node[midway, above] {\small $\Id{stak\vphantom j}$};
    \draw [->, thick] (9,0)--(11,0) node[midway, above] {\small $\Id{afstak}$};
    %
    \begin{scope}[yshift = -3cm]
    \node (0) at (0,0) [draw, fill=myblue, rectangle, inner sep = 4pt] {};
    \node (0) at (1,0) [draw, fill=myblue, rectangle, inner sep = 4pt] {};
    \node (0) at (2,0)  {$\cdots$};
    \node (0) at (3,0) [draw, fill = myblue, rectangle, inner sep = 4pt] {};
    \draw (3.5, .5) -- (-.5, .5);
    \draw (-.5,-.5) -- (3.5, -.5);
    \node at (1.5,1) [callout] {\emph{kø}};
    \draw [->, thick] (7,0)--(5,0) node[midway, above] {\small $\Id{kø}$};
    \draw [->, thick] (-6,0)--(-8,0) node[midway, above] {\small $\Id{afkø}$};
    \end{scope}
    %
    \begin{scope}[yshift = -6cm]
    \node (0) at (0,0) [draw, fill=myblue, rectangle, inner sep = 4pt] {};
    \node (0) at (1,0) [draw, fill=myblue, rectangle, inner sep = 4pt] {};
    \node (0) at (2,0)  {$\cdots$};
    \node (0) at (3,0) [draw, fill = myblue, rectangle, inner sep = 4pt] {};
    \draw (3.5, .5) -- (-.5, .5);
    \draw (-.5,-.5) -- (3.5, -.5);
    \node at (1.5,1) [callout] {\emph{dobbeltkø}};
    \draw [->, thick] (7,0)--(5,0) node[midway, above] {\small $\Id{tilføjBagest\vphantom j}$};
    \draw [->, thick] (9,0)--(11,0) node[midway, above] {\small $\Id{fjernBagest}$};
    \draw [->, thick] (-4,0)--(-2,0) node[midway, above] {\small $\Id{tilføjForrest}$};
    \draw [->, thick] (-6,0)--(-8,0) node[midway, above] {\small $\Id{fjernForrest}$};
    \end{scope}
  \end{tikzpicture}
\]
  \caption{\label{fig:queues}Operationer på stakke, køer og dobbeltkøer.}
\end{figure}

\begin{exerc}[Hanois tårne]
  \index{stak}
  \index{Hanois tårne}
 \emph{
  I det store brahmatempel i den hellige indiske by Varanasi, under den kuppel, som markerer verdens midtpunkt, ligger der på en messingplade 64 skiver af det rene guld.
   Skiverne har forskellig størrelse, og hver skive har et hul i midten.
   Præster bærer disse skiver enkeltvis frem og tilbage mellem tre lange nåle i overenstemmelse med Brahmas urokkelige lov: ingen skive må anbringes på en mindre skive.
   Da jorden blev skabt, lå alle 64 skiver på samme nål; die dannede Brahmas tårn.
   For tiden er transporten af skiverne til den anden nål i fuld gang.
   Når endelig Brahmas tårn genopstår i sin helhed på den anden nål, så er verdens undergang kommet og alt bliver støv.} 
 \cite{Hof83}
   \footnote{I virkligheden blev denne historie fundet på i 1883 af den franske matematiker Édouard Lucas som matematisk gåde.}
   \index{Lucas, Édouard}

   Giv en formel beskrivning af problemet at flytte et vilkårligt antal $k$ af skiver fra en nål til en anden ved hjælp af en trejde nål.
   Skriv et program, som realiserer de tre tårn som stakke $\langle u_1,\ldots, u_l\rangle$ med indgange fra $\{1,\ldots,k\}$, hvor der altid skal gælde $u_1>\cdots>u_l$.
   Programmet skal udlæse en følge af stakoperationer, som transformerer tilstanden 
   $(\langle k,\ldots, 1\rangle, \langle\,\rangle, \langle\,\rangle)$
   til tilstanden
   $(\langle\,\rangle, \langle k,\ldots, 1\rangle, \langle\,\rangle)$.
   \emph{Vink:} Løsningen er lettest at formulere rekursivt.
\end{exerc}

\begin{exerc}[Kø af to stakke]
  \index{kø!to stakke}
  Forklar, hvordan man kan implementere en kø ved hjælp af to stakke, sådan at hver køoperation tager konstant amortiseret tid.
\end{exerc}

Hvorfor gør vi os overhovedet tanker om følgetyperne stak, kø og dobbeltkø, når nu $\Id{Liste}$-datastrukturen allerede stiller samtlige deres operationer og flere til rådighed i konstant tid?
Det er der mindst tre grunde til.
For det første er programmer mere læsbare og indeholder færre fejl, hvis man udtrykkeligt indskrænker sig til snævrere anvendelsesmønstre.
For den andet tillader smallere grænseflader en større fleksibilitet ved implemenationen.
På grund af deres enkelhed kan stakke og køer bruge specialiserede implmentationer, som er mere pladsbesparende end implentationen af den almindelige type $\Id{Liste}$.
Vi vil se nærmere på denne algoritmiske aspekt i resten af afsnittet.
Især vil vi stræve efter implementationer, som er baserede på rækker i stedet for lister.
For det tredje er lister uegnede til brug af fjernlageret, idet hver adgang til en listeknude kan udløse en i/u-operation.
Når man derimod fremstiller stakke og køer med rækker, føre deres sekventielle adgangsmønster til gentagen adgang til samme blok i skyggelageret, hvilket leder til en stor forbedring af effektiviteten.

Begrænsede stakke, hvis maksimale størrelse er kendt på forhånd, kan umiddelbart implementateres som begrænsede rækker.
\index{kø!begrænset}
For ubegrænsede stakke kan vi bruge ubegrænsede rækker.
\index{kø!ubegrænset}
Stakke kan også let fremstilles som enkelthægtede lister; staktoppen svarer til listens begyndelse.
Køer kan fremstilles som enkelthægtede lister med en peger på den sidste kunde i listen.
Dobbeltkøer kan derimod ikke fremstilles som enkelthægtede lister på nogen effektiv måde.


\begin{figure}
  \begin{tabbing}
    ~~~~\=~~~~\=\kill
    \Class \Id{BegrænsetKø}(\Declare{$n$}{$\NN$}) \Of\Id{Element}\+\\
    \Declare{$b$}{$\Array[0..n] \Of \Id{Element}$}\\
  \DeclareInit{$h$}{$\NN$}{$0$}\qquad\comment{Indeks på første indgang\hspace*{3.2cm}}\\
  \DeclareInit{$t$}{$\NN$}{$0$}\qquad\comment{Indeks på første ledige position\hspace*{0.5cm}\unitlength1cm\begin{picture}(2.5,0)\includegraphics[width=2.3cm]{img/cycle-queue.eps}\end{picture}}\\[2mm]
    \TFunct{tom}{$\{0,1\}$}; \Return $h=t$\\[2mm]
    \TFunct{første}{Element}; \Assert $\neg\Id{tom}$; $\Return b[h]$\\[2mm]
  \TFunct{størrelse}{$\NN$}; \Return $(t-h+n+1) \bmod (n+1)$\\[2mm]
    \Procedure \Id{tilføjBagest}(\Declare{$x$}{\Id{Element}})\+\\
    \Assert $\Id{størrelse}<n$\\
    $b[t] \Is x$\\
    $t \Is (t+1)\bmod (n+1)$\-\\[2mm]
    \Procedure \Id{fjernForrest} \Assert $\neg\Id{tom}$; $h \Is (h+1)\bmod (n+1)$%
\end{tabbing}
  \caption{
    \label{alg:fifo}
    Implementation af en begrænset kø med en række.
  }
\end{figure}

Videre betragter vi nu implementationen af en kø på basis af en begrænsede række, se fig.~\ref{alg:fifo}.
\index{kø!cyklisk række}
Vi opfatter her en række som cyklisk struktur, hvor den sidste indgang følger indgangen med indeks 0.
\index{række!cyklisk|textbf}
Med andre ord er de mulige rækkeindeks tallene $0$, $\ldots$, $n$, og vi opfatter indeksene modulo $n+1$.
Vi opretholder to indeks $f$ og $t$, som begrænser køens gyldige område; køen omfatter rækkepositionerne med indeks i $f..t-1$.
Indeksene $f$ og $t$ vandrer ved inføjelse og fjernelse af indgange rundt i ring.
Indgangenes cykliske opførsel får man ved at regne modulo rækkestørrelsen $n+1$.%
\footnote{På mange regnere kan man opnå en mærkbar hastighedsforbedring af indeksberegningerne ved at vælge rækkestørrelsen som en topotens og erstatte modulooperationen med bitoperationer.}
Til enhver tid forbliver en rækkeindgang ubenyttet; ellers bliver det vanskeligt at skelne en fuld kø (med $n$ indgange) fra en tom kø.
Implementationen kan let overføres til begrænsede dobbeltkøer.
Zykliske rækker stiller sågar vilkærlig adgang med indeksoperatoren $[\cdot]$ til rådighed:
\[
  \operatorname{\textbf{Operator}} [i\colon\NN] \colon
  \Element; \return b[i+j\bmod n]
\]
Med ugangspunkt i begrænsede køer og dobbeltkøer kan man med teknikkerne for ubegrænsede rækker fra afsnit 3.2 også opnå de tilsvarende ubegrænsede udgaver.

Nu har vi set de væsentlige teknikker, som man har brug for til implementationen af stakke, køer og dobbeltkøer.
Disse teknikker kan kombineres for at opnå løsninger, som er specielt velegnede til meget lange følger, og til beregninger, som benytter eksterne lagermedier.

\begin{exerc}[Lister af rækker]
  \index{liste!af rækker}
  TODO
\end{exerc}

\begin{exerc}[Stakke og køer i fjernlageret]
    \index{liste!fjernlager}
    TODO
\end{exerc}

\section{Sammenligning af lister og rækker}
\label{s:o}

I tabel~\ref{tab:operations} findes en oversigt over dette kapitels resultater.
Rækker har fordele ved indeksbaseret adgang, hvorimod hægtede lister er mere velegnede til følger, som skal ændres (ved tilføjelse og fjernelse) på en vilkårlig plads.
Begge tilgange kan realisere stak- og køoperationerne effektiv.
Dog skal det siges, at rækker mere effektivt udnytter det skjulte lager, mens lister kan garantere beregningstidsgrænser i værste fald.

\begin{table}
\begin{tabular}{llllll}
  \toprule
  Operation & $\Id{Liste}$ 
  & $\Id{EListe}$ 
  & $\Id{URække}$ 
  & $\Id{CRække}$ 
  & Forlaring af »$*$« \\ \midrule
  $[\cdot]$ & $n$ & $n$ & 1 & 1 \\
  $\Id{størrelse}$ & $1^*$ & $1^*$ & 1 & 1  & ikke med $\Id{splejs}$ for flere lister
  \\
  $\Id{første}$ & $1$ & $1$ & 1 & 1  \\
  $\Id{sidste}$ & $1$ & $1$ & 1 & 1  \\
  $\Id{indsæt}$ & $1$ & $1^*$ & $n$ & $n$  & kun for $\Id{indsætEfter}$\\
  $\Id{fjern}$ & $1$ & $1^*$ & $n$ & $n$  & kun for $\Id{fjernEfter}$\\
  $\Id{tilføjForrest}$ & $1$ & $1$ & $1^*$ & $1^*$  & amortiseret \\
  $\Id{tilføjBagest}$ & $1$ & $1$ & $n$ & $1^*$  & amortiseret \\
  $\Id{fjernForrest}$ & $1$ & $n$ & $1^*$ & $1^*$  & amortiseret \\
  $\Id{fjernBagest}$ & $1$ & $1$ & $n$ & $1^*$  & amortiseret \\
  $\Id{sammenføj}$ & $1$ & $1$ & $n$ & $n$  & \\
  $\Id{splejs}$ & $1$ & $1$ & $n$ & $n$  & \\
  $\Id{findNæste},\ldots$ & $n$ & $n$ & $n^*$ & $n^*$  & \text{cacheeffektiv}\\
  \bottomrule
\end{tabular}
  \caption{
    \label{tab:operations}
    Beregningstider for operationer på følger med $n$ indgange.
    Det er underforstået, at hver undgang er omgivet af »$O(\cdot)$«.
    $\Id{Liste}$ står for dobbelthægtet liste, $\Id{EListe}$ står for enkelthægtet liste,
    $\Id{URække}$ står for ubegrænset række, $\Id{CListe}$ står for cyklisk række.
  }
\end{table}

\index{følge!operationer i overblik}
Enkelthægtede lister kan konkurrere med dobbelthægtede lister på de fleste punkter, men ikke alle.
Den eneste fordel ved cykliske organiserede rækker i forhold til ubegrænsede ræakker er, at de tillader effektiv implementation af $\Id{indsætForrest}$ og $\Id{fjernForrest}$.

\index{følge!pladsforbrug}
Spørgsmål om datastrukturens pladseffektivitet er ligeledes ikketrivielle.
Hægtede lister er meget kompakte, når indgangene er betydeligt større end pegerne.
For indgangstyper med ringe pladsbehov er rækker normalt en mere kompakt løsning, fordi ingen ekstra plads skal bruges til pegere.
Dette gælder med sikkerhed altid, når rækkestørrelsen er kendt i forvejen, så vi kan bruge begrænsede rækker.
Ubegrænsede rækker udviser en afvejning mellem pladseffektivitet og ekstra tidsforbrug for kopiering af indgangene ved operationen $\Id{realloker}$. 


\section{Implementationsaspekter}

De fleste programmeringssprog still berænsede rækker til rådighed.
% TODO changed "jede vernünftige" 
Udover disse findes ubegrænsede rækker, lister, stakke, køer og dobbbeltkøer tilgængeligt i biblioteker, som findes til de gængse imperative programmeringssprog.
Alligevel kommer man ofte i en situation, hvor man selv skal implementere listelignende datastrukturer, fx når de behandlende objekter forekommer samtidigt i som indgange i flere sammenkædede lister.
I den slags implementationer udgør lagerforvaltningen ofte en betydelig udfordring.

\subsection{C++}
\index{C++!følge}

Klassen $\Id{vector}\langle\Id{Element}\rangle$ i STL (Standard Template Library) realiserer ubegrænsede rækker.
\index{C++!vector@\Id{vector}}
Alligevel tillader de færreste implementationer af STL at rækker krymper.
Klassens funktionalitet omfatter fastlæggelsen af rækkens maximale senere omfang ved det tidspunkt, den bliver skabt.
Normalt angiver man hertil på dette tidspunkt et skøn over følgens længde $n$.
På den måde kan man undgå mange udvidelsesoperationer.
Sommetider ved man også, hvornår rækkens vækst er afsluttet, og kan så fremtvinge egenskaben $w=n$.
Med disse raffinemeter findes der egenlig ingen grund til at bruge $C++$s rækker som rækker i stil med C.
En yderligere fordel ved datatypen $\Id{vector}$ er, at dens instanser automatisk tilintegøres, når de tilsvarende variables gyldighedsområde bliver forladt.
I løbet af fejlfindningsfasen kan man desuden skifte over til varianter, som ved hver rækkeadgang foretager en kontrol af indeksgrænser.

Der findes yderliger nogle aspekter at tage hensyn til, nå man ofte har at gøre med voksende om krympende rækker og har fokus på specielt gode beregningstider.
I løbet af forstørrelsen og formindskningen skal klassen $\Id{vector}$ flytte indgange fra en række til en anden, hvortil den benytter konstruktøren $\Id{Copy}$ fra klassen $\Id{Element}$.
I de fleste tilfælde vil det være meget tidseffektivt at kalde den maskinnære grundoperation $\Id{memcpy}$, som kopierer en nærmere betegnet blok af på hinanden følgende bytes.
\index{C++!memcpy@\Id{memcpy}}
En anden maskinnær optimeringsmulighed betstår i at forvirklige operationen $\Id{realloker}$ ved hjælp af C-standardfunktionen $\Id{realloc}$.
\index{C!realloc@\Id{realloc}}
Lagerforvaltningssystemet kan så muligvis helt undgå at kopiere data.

Snublestenen ved anvendelsen af ubgrænsede rækker er den omstændighed, at pegere på rækkeindgange bliver ugyldige, når en ny række stilles til rådighed. 
(Man taler om, at »referenceintegriteten« er ødelagt.)
\index{referenceintegritet}
Man må altså ubetinget sikre, at rækken ikke bliver skiftet ud, mens sådan en peger er i brug.
Når den slags udskftning ikke kan udelukkes, må man referere til indgange ved hjælp af indekser i stedet for pegere.

Både STL
\index{STL!list@\Id{list}} 
og LEDA 
\index{LEDA!list@\Id{list}}~\cite{LEDA-AS} 
stiller dobbbelthægtede lister til rådighed i form af klassen $\Id{list}\langle Element\rangle$, og enkelthægtede lister i form af klassen $\Id{slist}\langle Element\rangle$.

[TODO incomplete]

\subsection{Java}
\index{java!følge}

I pakken $\Id{util}$ i Java~6 findes klassen $\Id{ArrayList}$ for ubegrænsede rækker og klassen $\Id{LinkedList}$ for dobbelthægtede lister.
\index{java!hægtet liste}
\index{java!dobbeltkø}
\index{java!stak}
\index{java!vector@\Id{Vector}}
Der findes også en grænseflade $\Id{Deque}$ for dobbeltkøer, som implementeres både af klassen $\Id{ArrayDeque}$ og af klassen $\Id{LinkedList}$.
Stakgrænsefladen $\Id{Stack}$ implementeres i Java som udvidelse af klassen $\Id{Vector}$.

Mange javabøger forkynder stolt, at Java ikke indeholder pegere.
Læseren kan altså spørge sig, hvordan man så kan implementere hægtede lister.
Svaret er naturligvis, at referencer til objekter i al væsentlighed er pegere, og at man i hvert fald kan bruge dem som var de pegere.
I en vis forstand kan man sågar sige, at Java \emph{kun} har pegere, fordi den eneste adgang til objetker af ikke-elementære datatyper er via referencer -- objekter gemmes aldrig i forælderobjektet.

Eksplicit lagerforvaltning er i Java ikke ubetinget nødvendig,
\index{java!lagerforvaltning}
fordi køretidssystemets skraldhåndtering tager sig af alle objekter, som ikke længere referes til.

\subsection{Python}
\index{python!følge}

Python råder ikke over begrænsede rækker som del af sprogets grundlæggende datatyper. 
For numeriske typer har man dog adgang til begrænsede rækker ved hjælp af modulet $\Id{array}$; den konkrete repræsentation er dog maskinafhængig.
Ellers er den primære følgetype i Python typen $\Id{List}$, som er implementeret som en ubegrænset række.
Standardmodulet $\Id{collections}$ indeholder datatypen $\Id{deque}$ for dobbelkøer, som er implementeret som en dobbelthægtet liste for hurtig manipulation af begge ender af følgen, men uden at understøtte hverken hurtig sammenføjning eller operationer på følgens interne indgange som fx $\Id{splejs}$.

\section{Historiske anmærkninger og videre resultater}\label{s:further}

Alle resultaterne i dette kapitler kan betragtes som »folkore«
\index{folklore@»folkore« (resultat)}
i den forstand, at de er kendt meget længe, og ingen gør krav på at have opfundet dem.
Faktisk har vi set, at en hel række af de grundlæggende abstrakte ideer er ældre end computeren.
Amortisering er lige så gammel som selve algoritmeanalysen.
\emph{Bankkontometoden} og \emph{potentialemetoden} blev foreslået i betyndelsen af 1980erne af 
\index{Brown, M. R.}%
\index{Mehlhorn, K.}%
\index{Sleator, D.}%
\index{Tarjan, R. E.}%
\index{Huddlestone, S.}
R.\,E.\,Brown, S.\,Huddlestone,
K.\,Mehlhorn, D.\,D.\,Sleator og R.\,E.\,Tarjan
\cite{Brown-Tarjan,Huddlestone-Mehlhorn,
SleTar83,ST85}.
Oversigtsartiklen~\cite{Tarjan:Amortized-Complexity} gjorde betegnelsen \emph{amortiseret analyse} alment anvendt;
sætning~\ref{thm:universal} er fra~\cite{Me:amortisierte-Analyse}.

Der findes en rækkelignende datastruktur for følger, som tillader indiceret adgang i konstant tid samt indføjelse og fjernelse på vilkårlig plads i amortiseret tid
$O(\sqrt{n})$.
Dette opnår man med et relativt enkelt kneb.
Rækker realisers som en følge af delrækker, hvoraf hver indeholder et antal
$n'=\Theta(\sqrt{n})$ af indgange, borset fra den sidste, som muligvis indholder færre indgange.  
Hver delrække organiseres cyklisk som beskrevet i afsnit ~\ref{s:stack}.
Adgang $[i]$ til position $i$ forskydes defor med en værdi $h$.
Indgangen med indeks $i$ finder man på plads $i\bmod n'$ af delrække nummer $\floor{i/n'}$. 
En ny indgang tilføjes i tid $O(\sqrt{n})$ i den delrække, som bestemmes af indeks $i$.
Denne delrække får nu en overskydende indgang.
For at opretholde invarianten, at hver delrække har eksakt $n'$ indgange, tilføjes den overskydende indgang som første element til den næste delrække.
Denne proces af gentagen videresending af en overskydende indgang gentages nu
$n/n'= O(\sqrt{n})$ gange, indtil den sidste delrække er nået.
Fjernelser gøres på lignende måde.
Sommetider skal der skabes en ny delrække, eller størrelsen $n'$ skal ændres og alle indgange omfordeles.
De amortiserede omkostninger ved disse ekstraoperationer kan holdes små.
Med yderligere en modifikation kan desuden alle dobbeltkøoperationer gennemføres i konstant tid.
I~\cite{MorKat01}\index{Katajainen, J.} kan den interesserede læser finde nogle snedige implementationer af dobbeltkøer og et implementationsstudie.


