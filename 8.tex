\chapter{Repræsentation af grafer}
\renewcommand{\labelprefix}{ch:grepresent}
\llabel{}
\vspace*{-4.5cm}
\begin{flushright}
%\includegraphics[width=3.5cm]{\masterdir/grepresent/spinnennetz.ps}
\includegraphics[width=3.5cm]{img/spiderweb2.eps}
\end{flushright}

\noindent
\emph{Videnskabelige resultater er først og fremmest tilgængelige som artikler i tidsskrifter og konferencerapporter, men også i forskellige kilder på WWW.
\footnote{Billedet forover viser et edderkoppespind.}
Disse artikler er ikke selv-indeholdte, men bygger videre på og citerer tidligere publiceret arbejde med beslægtet indhold.
Hvis man i dag læser en artikel fra 1975, som indeholder et interessant delresultat, spørger man sig umiddelbart, hvilke fremskridt der er gjort siden dengang.
Især er man interesseret i nyere artikler, der citerer 1975-resultatet.
Projekter som »Google Scholar« stiller denne funktionalitet til rådighed ved at analysere literaturhenvisningerne i en artikel og bygge en database, der gør det muligt at finde de artikler, som citerer en given artikel.
\index{graf!citationsnetværk}. 
}
\medskip

\index{graf!som model|textbf}%
Situationen foroven kan enkelt modelleres som en rettet graf.
Grafen indeholder en knude for hver artikel og en kant for hver citation.
Kanten $(u,v)$ fra artikel $u$ til artikel $v$ betyder, at $v$ citeres af $u$.
I denne repræsentation gemmer hver knude (artikel) $u$  alle udgående kanter (til de i $u$ citerede artikler), men ikke de indgående kanter (fra de artikler, som citerer $u$).
Hvis vi også skulle gemme de indgående kanter sammen med hver knude, ville det være let at finde de citerende artikler.
En af de centrale opgaver ved realiseringen af Google Scholar er netop konstruktionen af de modsatrettede kanter.
Dette eksempel illustrerer, at omkostningen ved en elementær grafoperation, nemlig at finde de indgående kanter, er stærkt afhængig af grafens repræsentation.
Hvis de indgående kanter er gemte eksplicit, er operationen enkel, ellers er den ikke-triviel.

I dette kapitel gennemgår vi forskellige muligheder for digital repræsentation af grafer.
Vi vil fokusere på rettede grafer og går ud fra, at en urettet graf $G=(V,E)$ er repræsenteret som det tilsvarende dobbeltrettede graf
\index{graf!doppeltrettet} 
$G'=(V,\bigcup_{\set{u,v}\in E}\set{(u,v),(v,u)})$
(jf. afsnit~\ref{ch:intro:s:graphnot}). 
I figur~\lref{fig:adjarray} vises foroven en urettet graf den dens dobbeltrettede modpart. 
de fleste datastrukturer, som behandles i dette kapitel, tillader også repræsentationen af (parallelle) multikanter	
\index{multigraf}
\index{graf!multigraf}
\index{kante!parallel}
og løkker $(v,v)$.
Vi vil begynde med at skabe os et overblik over de operationer, som muligvis skal kunne understøttes.

\begin{itemize}
\item \emph{Adgang til informationskomponenter.} 
  Ofte vil man have adgang til information, som er knyttet til et given knude eller kant,
  \index{knude!tilknyttet information}
  \index{kant!tilknyttet information}
  fx vægten 
  \index{kant!vægt}
  af den kant eller afstanden til en særlig knude.
  I mange graf\-repræsentationer er knuder og kanter objekter, og vi kan derfor direkte gemme informationen i objektets instansvariable.
  Medmindre andet er sagt, sætter vi $V=\{1,\ldots, n\}$,
  \index{knude!nummerering} 
  så vi kan gemme den til en knude knyttede information i række.
  \index{knude!knuderække|textbf}
  \index{knuderække@\Id{knuderække}|textbf}
  Hvis der ikke er en anden mulighed, kan man gemme knude- og kantinformation i en hakketabel.
  \index{haking!anvedelse}
  I hvert fald kan adgangen til informationskomponenter implementeres på en sådan måde, at de tager konstant tid.
  I resten af bogen vil vi abstrahere bort fra de forskellige muligheder for denne adgang og bruger datatyperne \Id{knuderække} og \Id{kantrække}
  \index{kantrække@\Id{kantrække}|textbf}
  \index{kant!kantrække|textbf}
  for at vise, at det drejer sig om en rækkelignende datastruktur, hvis indeksområde er mængden af kanter hhv. knuder. 
\item \emph{Navigation.}\index{graf!navigation} 
  For en givet knude vil man ofte have adgang til de udgående kanter.
  Det viser sig, at denne operation er af central betydning for de fleste grafalgoritmer.
  Som vi har set i det indledende eksempel, er det sommetide også nødvendigt at have adgang til de indgående kanter til en knude.
\item \emph{Kantforespørgsler.}
  \index{Kantforespørgsel}
  For et givet par af knuder $(u,v)$ vil vi måske vide, om parret udgør en kant i grafen. 
  Dette kan man altid implmentere som en hakketabel,
  \index{hakning!anvendelse}
  men sommetider ønsker vi en endnu hurtigere proces.
  En mere speciel, men vigtig forespørgsel er givet kanten $(u,v)$ i $G$ at finde (repræsentationen af) den \emph{modsatrettede kant} 
  \index{graf!modsatrettet kant} $(v,u)\in E$,
  såfremt den findes.
  Denne operation kan implementeres ved at lade en peger forbinde hver kant med sin modsatrettede kant.
\item \emph{Konstruktion, omvandling og udgift.}
\index{graf!omvandling}%
\index{graf!konstruktion}%
\index{graf!udgift }%
\index{graf!indgift}%
    Sommetider er en graf givet i en anden repræsentation end den, der er bedst egnet til løsningen af det aktuelle algoritmiske problem.
Det er ikke noget større problem, fordi de fleste graf\-repræsentationer kan omvandles til hinanden i lineær tid.
\item \emph{Aktualisiering.}
  \index{graf!dynamisk} 
  Sommetider vil man tilføje eller fjerne knuder og kanter i en graf.
    Fx er beskrivelsen af nogle algoritmer enklere, hvis man tilføjer en knude, som kan nå alle andre knuder (se fx figur~\ref{ch:spath:a: all pair}). 
\end{itemize}

%---------------------------------------------------------------------
\section{Uordnede kantfølger}\index{graf!kantfølge}

Den måske enkleste fremtilling af en gra fer som uordnet følge af kanter.
Hver kant indeholder et par af knudeindeks og muligvis yderligere information som fx kantvægten.
Om disse knudepar skal repræsentere en rettet eller urettet graf er et fortolkningsspørgsmål.
Repræsentation som kantliste bliver ofte brugt til input og output.
\index{graf!input}
\index{graf!output}
I denne repræsentation er det let at tilføje knuder og kanter i konstant tid.
Mange andre operationer, især navigation i grafen, er derimod så langsomme, at repræsentationen stort set er ubrugelig.
Derfor arbejder kun få grafalgoritmer godt med kantlister;
de fleste har nemlig brug for hurtig adgang til en knudes nabokanter.
I disse tilfælde foretrækker man de ordende repræsentationer, som diskuteres i de følgende afsnit.
I kapitel~\ref{ch:mst:} skal vi møde to algoritmer for minimalt spændetræ;
den ene virker fint selv med kant\-liste\-repræsentationen, den anden har brug for en mere raffineret datastruktur.
 
%---------------------------------------------------------------------
\section{Naborækker -- statiske grafer}
\llabel{ss:adjarray}%
\index{graf!naborække|textbf}%
\index{naborække|sieheunter{graf}}%
\index{graf!statisk}

For at tillade hurtigere adgang til de udgående kanter for given knude kan man for hver knude gemme dens udgånde kanter i en række.
Når der ikke er knyttet yderlige information til kanterne, indeholder denne række blot målknudernes indeks.
Når grafen er \emph{statisk}, dvs. ikke forandres i løbet af tiden, kan man føje de små rækker sammen til én stor kantrække $E$.
En yderligere række $V$ med $n+1$ indgange gemmer begyndelsespositionerne for de små rækker, dvs. for hver knude $v$ indeholder $V[v]$ det første index i $E$, som indeholder en kant med startknude $v$.
I $V[n+1]$ står attrapværdien $m+1$.
For hver knude $v$ finder man de udgående kanter fra $v$ nemt som $E[V[v]]$, \ldots, $E[V[v+1]-1]$;
attrapværdien sikrer, at denne regel også gælder for $v=n$.
Et eksempel på sådan en naborække finder man i figur~\lref{fig:adjarray} (i midten til venstre).

Lagerbehovet for en graf med $n$ knuder og $m$ kanter er $n+m+\Theta(1)$ maskinord.
Det er endnu mere kompakt end de $2m$ maskinord, som bruges i kant\-liste\-repræsentationen.

Repræsentationen som naborække kan udvides til at gemme yderligere information:
Kantinformation kan man gemme enten i en yderligere række med $m$ indgange eller i selve kantrækken.
Hvis de indgående kanter også skal repræsenteres eksplicit, kan man gemme den omvendte graf i yderligere rækker $V'$ og $E'$.


\begin{figure}[t]
\begin{center}
%\leavevmode\includegraphics[width=0.9\textwidth]{\masterdir/grepresent/represent3.eps}
\leavevmode\includegraphics[scale=0.416]{img/represent3.eps}
\end{center}
\vspace*{-3cm}\hspace*{0.7\textwidth}$\displaystyle\left(\;\begin{array}{c@{\;\;\;}c@{\;\;\;}c@{\;\;\;}c}
  0&1&0&1\\
  1&0&1&1\\
  0&1&0&1\\
  1&1&1&0
\end{array}\;\right)$
\vspace*{2cm}
\caption{\llabel{fig:adjarray}
\emph{Foroven}: En urettet graf og dens tilsvarende dobbeltrettede graf.
\emph{Midte}: Repræsentation af samme graf som naborække og som nabolister. 
\emph{Forneden}: Repræsentation som sammenhægtede kantobjekter og som nabomatrix.}
\end{figure}

\begin{exerc}\index{graf!konvertering}
  Beskriv en algoritme som konverterer kant\-liste\-representationen til naborækker i lineær tid.
  Brug kun $O(1)$ ekstra plads.
  \emph{Vink}:
  Tænk på problemet som opgaven at sortere kanter efter startknude
  \index{startknude}
  og ændr algoritmen i figur~\ref{ch:sort:alg:KsortArray} til heltalssortering, så den kan bruges her.
\end{exerc}


%---------------------------------------------------------------------
\section{Naborækker -- dynamiske grafer}
\llabel{ss:adjlist}%
\index{graf!naboliste|textbf}%
\index{naboliste|sieheunter{graf}}%
\index{graf!dynamisk}%
\index{liste}

Kantrækker repræsenterer grafer på en kompakt og effektiv måde.
Deres hovedsaglige ulempe er, at tilføjelse og fjernelse af kanter er dyr.
Antag fx, at vi skal tilføje en ny kant $(u,v)$.
Selv hvis der stadig er plads i kantrækken $E$, skal en kant for hver af knuderne $u+1$ til $n$ flyttes til højre, hvilket kræver tid $O(n)$.

I kapitel~\ref{ch:sequence:} har vi set, hvordan man implementerer dynamiske følger.
Alle disse løsninger kan bruges til at konstruere en datastruktur for dynamiske grafer.
For hver knude $v$ fremtiller vi følgen $E_v$ af dens udgående kanter (eller dens indgående kanter, eller både og) som ubegrænset række \index{række!ubegrænset} eller (dobbelt eller enkelt) hægtet liste.
Fordele og ulemper ved de forskellige følge\-repræsentationer overføres selvfølgeligt. 
Ubegrænsede rækker bruger lagerhierarkiet bedre.
% TODO cache
Hægtede lister muliggør tilføjelse og fjernelse på vilkårlig plads i konstant tid.
De fleste grafer er »tynde«
\index{graf!tynd} 
i den forstand, at hver knude i gennemsnit kun har få ind- og udgående kanter. 
Hægtede lister til repræsentation af tynde grafer bør implementeres uden attrapknuden
\index{liste!attrapknude}
fra afsnit~\ref{ch:sequence:s:list},
fordi de nødvendige ekstra $n$ knuder udgør et betydligt pladsspilde.
I eksemplet i figur~\lref{fig:adjarray} (midten) bruges derfor cirkulært dobbelthægtede lister. \index{liste!cirkulær} 


\begin{exerc}
  Antag, at hver knude $u$s udgående kanter er gemt i en ubegrænset række $E_u$.
  En kant $e=(u,v)$ er givet ved sin position i $E_u$.
  Forklar, hvordan $e$ kan fjernes i amortiseret konstant tid.
  \emph{Vink}: De andre kanter relative orden behøver ikke at blive bevaret.
\end{exerc}


\begin{exerc}
  \llabel{ex:dagtest}
  \index{graf!kredsgenkendelse}
  \index{kø}
  Betragt igen algoritmen fra afsnit~\ref{ch:intro:s:graphnot}, som afgør, om en givet rettet graf er kredsfri.
  Forklar, hvordan den kan implementeres til at arbejde i lineær tid.
  Det kræves en passende repræsentation af dynamiske grafer samt en algoritme, som kan bruge denne repræsentation effektivt.
  \emph{Vink}:
  Brug en kø til at gemme knuder med udgrad $0$.
\end{exerc}


Doppeltrettede 
\index{graf!doppeltrettet}
\index{graf!hægtede kantobjekter}
grafer forekommer i mange sammenhænge. 
Urettede grafer kan repræsenteres som dobbeltrettede på en naturlig måde, og visse algortimer for rettede grafer har også brug for hurtig adgang til en knudes indgående kanter.
I denne situation ønsker man ofte at gemme den information, som hører til en urettet kant (eller til en rettet kant og dens omvending), kun én gang.
Ligeledes ønsker man nem adgang fra en kant til dens omvending.

Vi vil beskrive to løsninger, som opfylder disse krav.
Den første forsyner enkelt hver rettede kant med to yderligere pegere.
Den ene peger på den omvendte kant, den anden på den information, som er knyttet til kanten.

Den anden løsning nøjes med en enkelt listeknude for hver kant og henviser til denne knude i to nabolister.
For urettede grafer skulle altså listeknuden for den urettede kant $\{u,v\}$ forekomme både i listen $E_u$ og i listen $E_v$.
Hver knude $u$ gemmer en peger på en af sine hosliggende kanter og giver derigennem adgang til hele nabolisten.
Hvis nabolisten skal være dobbelthægtet (selvom en cirkulært hægtet liste ofte er tilstrækkelig), behøver sådan en listeknude fire pegere:
to pegere på $u$s naboliste $E_u$ og to pegere på $v$s naboliste $E_v$.
Fra den pågældende listeknude kan man gennemløbe alle dens hosliggende kanter.
Et eksempel er vist forneden i figur~\lref{fig:adjarray}.
Det er en smule omstændigt at finde frem til knuden i den anden ende af en kant.
Listeknuden for kanten bruger ikke nogen særlig ordning til at gemme endeknuderne.
Når vi altså i kantlisten støder på en hosliggende kant til $u$, skal vi inspicere begge knuder og vælge den, der er forskellig fra $u$.
Alt afhængig af, om dette er den første eller anden knude i objektet, skal næste skridt i gennemløbet tages ud fra det første eller det andet par af pegere.
Et elegant alternativ er at gemme værdien $u\oplus v$ i listeknuden for kanten $\{u,v\}$.
\cite{Naeher-Zlotowski}
\index{Naher@Näher, S.}
\index{Zlotowski, O.} 
Bitvis ekslusivt-eller af denne indgang med et endepunkt leverer så det andet.
Denne repræsentation sparer pladsen for et knudenavn.
Diese Darstellung spart den Platz für einen Knotennamen ein. 
Dog skal man nu træffe en beslutning of, i hvilken ordning listepegerne svarer til kantens endepunkter. 
For eksempel kan man vedtage, at listeknuden for kanten $\{u,v\}$ med $u<v$ bruger sig første par af pegere til $u$s naboliste og sit andet par til $v$s.

For rettede grafer, hvor hver kant $(u,v)$ skal gemmes som udkant til $u$ og som indkant til $v$, forekommer listeknuden for den rettede kant $(u,v)$ i listen af $u$s udkanter og i listen av $v$s indkanter.
Også her skal man for hver kant bruge fire pegere i dens listeknude, to for dens rolle som udkant og to for dens rolle som indkant.
Hver grafknude har en peger på en af sine udkanter, og en peger på en af sine indkanter.

%---------------------------------------------------------------------
\section{Nabomatriser}%
\index{graf!nabomatrix|textbf}%
\index{matrix}%
\index{nabomatrix|sieheunter{graf}}%
\index{graf!tæt}%
\index{algoritmedesign!algebraisk}

En graf med $n$ knuder kan repræsenteres som \emph{nabomatrix} $A$ af dimension 
$n\times n$.
Indgangen ${A}_{ij}$ er 1, hvis $(i,j)\in E$, og 0 ellers.
Indsættelse og fjernelse af en kant såvel som kantforespørgsler kan gennemføres i konstant tid.
For finde en hosliggende kant til en knude, kræves tid $\Theta(n)$.
For meget tætte grafer med $m=\Omega(n^2)$ bør det anses som effektivt. 
Lagerbehovet er $n^2$~bit.
For meget tætte grafer kan dette være bedre end de $n + m + O(1)$ maskinord, som man bruger på naborækker.
Men selv for disse grafer er besparelsen ringe, når kanterne indeholder yderligere information.

\begin{exerc}
  \index{graf!nabomatrix!urettet}
  Forklar, hvordan man kan repræsentere en urettet graf på $n$ knuder og uden løkker med kun $n(n-1)/2$~bit. 
  Kantforespørgsler skal gøres i tid $O(1)$.
\end{exerc}

Måske vigtigere end lageraspekterne ved brug af nabomatriser er den strukturelle sammenhæng mellem grafer og lineær algebra,
\index{lineær algebra},
som skabes af nabomatrisen $A$.
På den ene side kan nogle grafteoretiske problemer løses ved hjælp af metoder fra lineær algebra.
Hvis vi fx sætter ${C}={A}^k$, så er  ${C}_{ij}$ antallet af (ikke nødvendigvist simple) veje
\index{graf!antal veje} 
fra knude $i$ til knude $j$ med præcis $k$ kanter.

\begin{exerc}
  Forklar, hvordan man kan gemme en $n\times n$-matrix ${A}$ med $m$ indgange forskellig fra 0 i plads $O(m+n)$, således at matrix-vektor-produktet ${A}{x}$ kan udregnes i tid $O(m+n)$.
  Beskriv multiplikationsalgoritmen.
  Udvid repræsentationen, så den også omfatter produkter på formen ${x}^{\textsf{T}}{A}$ i tid $O(m+n)$.
\end{exerc}

På den anden side kan grafteoretiske koncepter være nyttige til at løse problemer fra lineær algebra.
Antag fx, at vi vil løse matrixligningen ${B}{x}={c}$, hvor $B$ er en symmetrisk matrix.
Begragt den tilsvarende nabomatrix ${A}$, for hvilken ${A}_{ij}=1$ hvis og kun hvis ${B}_{ij}\neq 0$.
Når en algoritme for sammenhængskomponenter finder ud af, at grafen, som beskrives af  $A$, har to sammenhængskomponenter, kan vi bruge denne opdeling til at omorganisere rækker og søjler i $B$ til en ensbetydende ligning:
$$
\left(\begin{array}{cc}
{B}_1 & {0} \\
{0}   & {B}_2
      \end{array}\right) 
\left(\begin{array}{c}
{x}_1\\
{x}_2
\end{array}\right)
=
\left(\begin{array}{c}
{c}_1\\
{c}_2
\end{array}\right)\,.
$$
Denne ligning kan løses ved at løse hver af de to ligninger ${B}_1{x}_1= {c}_1$
og ${B}_2{x}_2={c}_2$ for sig.
I praksis er situationen mere speget, fordi matriser sjældent har den egenskab, at deres tilsvarende urettede graf er usammenhængende.
Alligevel kan man bruge mere præcise grafteoretiske koncepter som fx snit
\index{graf!snit}
til at opdage strukturer i matrisen, som muliggør mere effektiv løsning af ligningssystemet.

\section{Implicitte repræsentationer}%
\index{graf!gitter-}

Mange anvendelser arbejder på grafer med speciel struktur.
Ofte kan man udnytte denne situation til at opnå enklere og mere effektive repræsentationer.
Vi giver to eksempler.

\emph{Gittergrafen} $G_{k\ell}$ med knudemængden
$ V= \{1,\ldots,k\}\times \{1,\ldots,\ell\}$
og kantmængden
$$E=\{\,((i,j),(i,j'))\in V^2\colon \abs{j-j'}=1\,\}\cup
   \{\,((i,j),(i',j))\in V^2\colon \abs{i-i'}=1\,\}$$
   er en graf, som er helt bestemt af de to parametre $k$ og $\ell$.
  Figur~\lref{fig:grid} viser $G_{3,4}$.
  Kantvægte bør man gemme i to todimensionelle rækker, en for de vertikale og en for de horisontale kanter.

En \emph{intervalgraf}
\index{graf!intervalgraf} 
er givet ved en endelig mængde af intervaller i $\RR$.
Til hvert interval findes en knude i grafen, og to knuder er naboer, hvis de tilsvarende intervaller overlapper.

\begin{figure}[h]
\sidecaption
\leavevmode\includegraphics[width=6cm]{img/grid.eps}
\caption{\llabel{fig:grid} Gittergrafen $G_{34}$ (\emph{venstre}) og en intervalgraf med fem knuder og seks kanter (\emph{højre}).}
\end{figure}

\begin{exerc}[Repræsentation af intervalgrafer]
  \llabel{ex:interval} 
\begin{description}
  \item[(a)] Vis, at der for hver mængde af $n$  intervaller i $\RR$ findes en mængde af intervaller, hvis endepunkter er heltal i $\{1,\ldots,2n\}$ og som definerer samme graf.
  \item[(b)] Beskriv en algoritme, som afgør, om en intervalgraf givet som $n$ intervaller er sammenhængende.
  \emph{Vink}: 
  Algoritmen bør sortere intervallernes endepunkter
  \index{sortering!anvendelse}
  \index{algoritmekonstruktion!sortering!anvendelse} 
  og derefter gennemgå dem i stigende rækkefølge.
  Oprethold herunder mængden af af påbegyndte, men ej afsluttede, intervalle.
  \item[(c*)] 
    Beskriv en repræsentation af intervalgrafer, som nøjes med $O(n)$ plads og understøtter effektiv navigation i følgende forstand:
    Givet et interval $I$ skal alle intervaller $I'$ rapporteres, som overlapper $I$;
    ved overlap forstås her at $I$ indeholder et af $I'$s endpunkter eller der gælder $I\subseteq I'$. 
    Hvordan kan man givet $I$ finde samtlige intervaller af begge slags?
\end{description}
\end{exerc}


%%%%%%%%%%%%%%%%%%%%%%%%%%%%%%%%%%%%%%%%%%%%%%%%%%%%%%%%%%%%%%%%%%%%%%
\section{Implementationsaspekter}\llabel{s:implementation}

\emph{Udeladt}

%%%%%%%%%%%%%%%%%%%%%%%%%%%%%%%%%%%%%%%%%%%%%%%%%%%%%%%%%%%%%%%%%%%%%%
\section{Historiske anmærkninger og yderligere resultater}
\llabel{s:further}

\emph{Udeladt}


