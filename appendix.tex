\newenvironment{mydescription}
{\begin{list}{}{\setlength\labelwidth{0pt}%
               \setlength\itemindent{-\leftmargin}%
               \addtolength{\itemindent}{5pt}
               \setlength\itemsep{8pt plus 1pt}%
               \let\makelabel\mydescriptionlabel}}
{\end{list}}
\newcommand*\mydescriptionlabel[1]{#1}

\newcommand{\lref}[1]{\eqref{#1}}
\section{Matematiske symboler}

\begin{mydescription}
\item[$\{e_1,\ldots,e_n\}$] mængden af elementer $e_1$, $\ldots$, $e_n$.

\item[$\{\,e\colon P(e)\,\}$] mængden af elementer, for hvilke $P$ gælder.

\item[$\langle e_1,\ldots,e_n\rangle $] følgen af indgange $e_1$, $\ldots$, $e_n$.

\item[$\{\,e\in S\colon P(e)\,\}$] delfølgen af indgange fra $S$, for hvilke $P$ gælder.

\item[$|x|$] absolutværdien af $x$ for et reelt tal $x$.

\item[$\lfloor x\rfloor$] det største heltal $n$ med $n\leq x$ for et reelt tal $x$ (gulvfunktionen, nedrunding).

\item[$\lceil x\rceil$] det minste heltal $n$ med $n\geq x$ for et reelt tal $x$ (loftsfunktionen, oprunding).


\item[\mbox{$[a,b]$}] $\{\, r\in \RR\colon a \leq x\leq b\}$.


\item[$i..j$] forkortelse for $\{i, \ldots, j\}$

\item[$A^B$] mængden af funtioner fra $A$ til $B$

\item[$A\times B$] mængden af ordnede par $(a,b)$ med $a\in A$ og $b\in B$.

\item[$\bot$] en udefineret værdi

\item[$\infty$] uendeligt

\item[$-\infty$] minus uendeligt

\item[$\forall x\colon P(x)$] alle $x$ opfylder udsagnet $P$ 

\item[$\exists x\colon P(x)$] der ekisterer et $x$, som opfylder udsagnet $P$ 

\item[$\NN$] mængden af ikke-negative heltal, $\NN= \{0,1,2,\ldots\}$.

\item[$\NN_+$] mængden af positive heltal, $\NN= \{1,2,\ldots\}$.

\item[$\ZZ$] mængden af hele tal, $\ZZ= \{\ldots, -2,-1,0,1,2,\ldots\}$.

\item[$\RR$] mængden af reelle tal.

\item[$\RR_{>0}$] mængden af positive reelle tal.
\item[$\QQ$] mængden af rationelle tal.
\item[|, \texttt{\&},  \texttt{«}, \texttt{»}, $\oplus$] bitvis eller, og, venstreskift, højreskift og eksklusivt eller.
\item[$\sum_{i=1}^n a_i = \sum_{1\leq i\leq n} a_i = \sum_{i\in\{1,\ldots,n\} a_i}$] $=a_1+\cdots+a_n$.
\item[$\sum_{s\in S} a_x$ hhv. $\sum_{P(x)}a_x$ eller $\sum_{x\text{ med } P(x)} a_x$] Summen over elementerne i en endelig mængde $S$, hhv. over alle elementer som upfylder prædikatet $P$.
\item[$\prod_{i=1}^n a_i = \prod_{1\leq i\leq n} a_i = \prod_{i\in\{1,\ldots,n\} a_i}$] $=a_1\cdot\cdots\cdot a_n$.
\item[$n!$] $\prod_{i=1}^ni$, kaldt »$n$ fakultet«. Der gælder $0!=1$.
\item[$H_n$] $\sum_{i=1}^n1/i$, kaldt »det $n$te harmoniske tal« (se ulighed (A.12)).
\item[$\log x$] logaritmen til $x$ med grundtal $2$, $\log_2x$, for $x>0$.
\item[$\log^* x$] for $x>0$, det mindste tal $k\geq 0$ for hvilket $\log(\log(\cdots (\log x)\cdots ))\leq 1$ ($k$-foldig gentagelse af logaritmen).
\item[$\ln x$] den naturlige logaritme til $x$ med grundtal $2,7182818284590\ldots$, for $x>0$.
\item [$\mu(s,t)$] længden af den korteste vej fra $s$ til $t$; $\mu(t):=\mu(s,t)$.
\item[$\ddiv$] kvotienten ved heltalsdivision; $m\ddiv n := \lceil m/n\rceil $ for $n>0$.
\item[$\bmod$] resten ved heltalsdivision; $m\bmod n := m - n(m\ddiv n)$ for $n>0$.
\item[$a\equiv b\pmod m$] for $m>0$: tallene $a$ og $b$ er kongruente modulo $m$, dvs. $a+im =b $ for et heltal $i$.
\item[$\prec$] en vilkårlig ordensrelation.
\item[$1$, $0$] de booleske værdier »sand« og »falsk«. 
\item[$\Sigma^*$] mængden $\{\,(a_1,\ldots,a_n) \colon n\in \NN, a_1,\ldots, a_n\in \Sigma\,\}$ af tegnfølger (eller »ord«) $a_1\cdots a_n=(a_1,\ldots,a_n)$ over det endelige alfabet $\Sigma$. 
\item[$|x|$] antallet $n$ af bogstaver i $x=(a_1,\ldots,a_n)\in \Sigma^*$ for ord $x$.

\end{mydescription}

\section{Matematiske begreber}\label{mathconcepts}

\newcommand{\Oh}[1]{\operatorname{O}(#1)}
\newcommand{\Om}[1]{\Omega(#1)}
\newcommand{\Th}[1]{\Theta(#1)}

\renewcommand*\mydescriptionlabel[1]{{\bf #1}:}
\begin{mydescription}

  \item[antisymmetrisk]
    \index{relation!antisymmetrisk|textbf}
    \index{antisymmetrisk relation|textbf}
    En relation $R$ kaldes \emph{antisymmetrisk}, hvis der for alle $a$ og $b$  gælder, at $aRb$ og $bRa$ medfører $a = b$. 

  \item[ækvivalensrelation]
    \index{relation!aekvivalens@ækvivalens-|textbf}
    \index{aekvivalensrelation@ækvivalensrelation|textbf}
    eine transitive, reflexive und
    symmetrische Relation. 

  \item[asymptotische Notation] 
    \begin{align*}
      \Oh{f(n)} &=  \{\,g(n)\colon\exists c>0\colon\exists n_0\in\NN_+\colon\forall n\geq n_0\colon g(n)\leq c\cdot f(n)\,\}.\\
      \Om{f(n)} &=  \{\,g(n)\colon\exists c>0\colon\exists n_0\in\NN_+\colon\forall n\geq n_0\colon g(n)\geq c\cdot f(n)\,\}.\\
      \Th{f(n)} &=  \Oh{f(n)}\cap\,\Om{f(n)}.\\
      o(f(n)) &=  \{\,g(n)\colon\forall c>0\colon\exists n_0\in\NN_+\colon\forall n\geq n_0\colon g(n)\leq c\cdot f(n)\,\}.\\
      \omega(f(n))  &=  \{\,g(n)\colon\forall c>0\colon\exists
    n_0\in\NN_+\colon\forall n\geq n_0\colon g(n)\geq c\cdot f(n)\,\}\,.\end{align*}
    Se også afsnit~\ref{ch:intro:s:o}.

  \item[konkav]\index{konkav funktion|textbf} 
    En funktion $f$ er konkav på intervallet $[a,b]$, hvis
    \[\forall x,y\in[a,b]\colon\forall t\in[0,1]\colon f(tx+(1-t)y)\geq tf(x)+(1-t)f(y).\]

  \item[konveks]
    \index{konveks funktion|textbf} 
    En funktion  $f$ er konveks på internvallet $[a,b]$, hvis
    \[\forall x,y\in[a,b]\colon\forall t\in[0,1]\colon f(tx+(1-t)y)\le tf(x)+(1-t)f(y).\]

  \item[legeme]
    \index{legeme (algebra)|textbf} 
    en mængde af tal (indeholdende nul og et), for hvilke der er defineret følgende operationer: 
    addition og (en hertil invers operation) subtraktion, multiplikation og (en hertil invers operation) division med elementer, som ikke er nul.
    Addition og multiplikation er kommutativ og associativ og har neutralelementer, som opfører sig som de reelle tal nul og et.
    De væsentligste eksempler på legemer er:
    $\RR$, de reelle tal;
    $\QQ$, de rationelle tal;
    $\ZZ_p$, heltallene modulo et primtal $p$.

  \item[leksikografisk ordning]
    \index{leksikografisk ordning|textbf} 
    den kanoniske måde at udvide en lineær ordning af en mængde til tupler, strenge og følger over samme mængde.
    Der gælder $\seq{a_1,\ldots,a_k} < \seq{b_1,\ldots,b_\ell}$
    hvis og kun hvis findes et $i\le\min\set{k,\ldots, \ell}$  med $\seq{a_1,\ldots,a_{i-1}} = \seq{b_1,\ldots,b_{i-1}}$ og $a_i < b_i$ eller 
    eller når $k<\ell$ og $\seq{a_1,\ldots,a_k} = \seq{b_1,\ldots,b_k}$.
    En alternativ, rekursiv definition er:
    $\seq{\,}<\seq{b_1,\ldots,b_\ell}$ for alle $\ell>0$; for $k>0$ og $\ell>0$ gælder
    $\seq{a_1,a_2,\ldots,a_k} < \seq{b_1,b_2,\ldots,b_\ell}$ hvis og kun hvis 
    $a_1 < b_1$ eller $a_1=b_1$ og $\seq{a_2,\ldots,a_k}<\seq{b_2,\ldots,b_\ell}$.

  \item[lineær ordning]\index{lineær ordning|textbf} 
    (også: total ordning)
    en refleksiv, transitiv, antisymmetrisk og total relation.
    Ofte benyttes symbolet $\le$ for en lineær ordning.
    For alle $a\leq b$ skrives da også $g\ge a$..
    Forkortelse (den \emph{strenge} version af en total ordning):
    Vi skriver $a<b$ når $a\le b$ og $a\ne b$; tilsvarende står $a>b$ for $a\ge b$ og $a\ne b$.
    I så fald er relationene $<$ transitiv, \emph{irrefleksiv} (idet $a<b$ medfører $a\ne b$) og \emph{total} i den forstand at der for hvert par $a,b$ gælder $a<b$ eller $a=b$ eller $a>b$.

  \item[lineær præordning]
    \index{lineær præordning|textbf}
    \index{lineær kvasiordning|textbf} (også: total præordning eller lineær/total kvasiordning)
    en refleksiv, transitiv og total relation.
    Også hertil benyttes ofte symbolerne $\le$ og $\ge$.

    Den strenge variant defineres i dette tilfælde på følgende måde:
    $a<b$ hvis der gælder $a\le b$ men ikke $a\ge b$.


  \item[median]\index{median|textbf}
    en indgang med rang $\ceil{\frac{1}{2}n}$ i en mængde med $n$ indgange.

  \item[multiplikativ invers]\index{invers|textbf}
    Når man multiplicerer et objekt $x$ med dets \emph{multiplikative invers} $x^{-1}$,
    opnås $x\cdot x^{-1}=1$, dvs. multiplikationens neutralelement. 
    Isæar gælder der i et \emph{legeme}, at hvert element bortset fra nul (additionens neutralelement) har en multiplikativ invers.

  \item[primtal]\index{primtal|textbf} 
    et heltal $n$ med $n \ge 2$ er et primtal, dersom der ikke findes heltal $a,b>1$ så $n=a\cdot b$.

  \item[rang]\index{rang|textbf} 
    Lad den lineære præordning $\le$ være defineret på en endelig mængde $S = \set{e_1,\ldots,e_n}$. 
    En injektiv afbildning $r\colon S \rightarrow\set{1,\ldots, n}$ er en \emph{rangfunktion} for $S$, hvis der gælder $r(e_i) < r(e_j)$  
    for alle $e_i$, $e_j$ med $e_i < e_j$.
    Når $\le$ er en lineær ordning på $S$, eksisterer der præcis en rangfunktion.

  \item[refleksiv]\index{refleksiv|textbf}\index{relation!refleksiv|textbf} 
    En relation $R \subseteq A\times A$ kaldes \emph{refleksiv}, hvis $\forall a\in A\colon (a,a)\in R$.

  \item[relation]\index{relation|textbf} 
    En (binær) relation $R$ er en mængde af ordnede par.
    Vi skriver ofte relationen med infiksnotation: 
    udsagnet $aRb$ betyder bare $(a,b)\in R$.
    (Alle relationer, som forekommer i denne bog, er binære.)  

    %\item[strict weak order] A relation that is like a total order except 
    %  the antisymmetry only needs to hold with respect to some equivalence relation $\equiv$ that
    %  is not necessarily the identity (see also
    %\url{http://www.sgi.com/tech/stl/LessThanComparable.html}).
  \item[symmetrisk relation]
    \index{symmetrisk relation|textbf}
    \index{relation!symmetrisk|textbf}
    Relationen $R$ er  \emph{symmetrisk}, hvis der for alle  $a$ og $b$ gælder, at $aR b$ medfører $bR a$.

  \item[total ordning]\index{total ordning|textbf} dss. lineær ordning.

  \item[total relation]
    \index{total relation|textbf} 
    \index{relation!total|textbf} 
    en relation $R \subseteq A\times A$ er total på $A$, dersom for alle $a,b\in A$ gælder mindst et af udsagnene $aR b$ og $bR a$.
    Når en relation $R$ er  total og transitiv, danner følgende definition en passende ækvivalensrelation $\sim_R$ på $A$: 
    $a \sim_R b$ hvis og kun hvis $aRb$ og $bRa$.

  \item[transitiv]
    \index{transitiv relation|textbf}
    \index{relation!transitiv|textbf} 
    relationen $R$ kaldes \emph{transitiv}, hvis der vor alle $a$, $b$, $c$ gælder, at $aR b$ og $bRc$ medfører $aR c$. 

    %\item[true] Abbreviation for the value $1$.

\end{mydescription}

\section{Sandsynlighedsregningens grundlag}

\section{Nogle nyttige formler og uligheder}

Vi begynder med at opremse nogle nyttige formler og grænser.
Beviser for nogle af dem finder man forneden.

\begin{itemize}
  \item Enkle grænser for fakultetsfunktionen:
    \begin{equation}\label{eq:factorial}
      \left(\frac{n}{e}\right)^n \le n! \le n^n \text{, eller skarpere: } e \left(\frac{n}{e}\right)^n \le n! \le (en)\left(\frac{n}{e}\right)^n.
    \end{equation}

  \item 
    Stirlings approksimationsformel for fakultetsfunktionen:
    \begin{equation}\label{eq:stirling}
      n! = \left(1+\Th{\frac{1}{n}}\right)\sqrt{2\pi n}\left(\frac{n}{e}\right)^n \,,\
    \end{equation}\index{Stirlings formel|textbf}
    eller mere nøjagtig:
    \begin{equation}\tag*{(A.9$'$)}
      \sqrt{2\pi n}\cdot\left(\frac{n}{e}\right)^n \cdot e^{\frac1{12n+1}}
      \; < \;  n!  \; < \; 
      \sqrt{2\pi n}\cdot\left(\frac{n}{e}\right)^n \cdot e^{\frac1{12n}}\mbox{ , für $n\ge1$.}
    \end{equation}

  \item En øvre grænse for binomialkoefficienten:
    \begin{equation}\label{eq:bincoeff}
      \binom{n}{k}\leq\left(\frac{n\cdot e}{k}\right)^k .
    \end{equation}\index{binomialkoefficient|textbf}

  \item
    Summen af de første $n$ positive heltal:
    \begin{equation}\label{eq:sumi}
      \sum_{i=1}^ni = \frac{n(n+1)}{2} .
    \end{equation}


  \item De harmoniske tal:
    \begin{equation}\label{eq:harmonic}
      \ln n\leq H_n=\sum_{i=1}^n \frac{1}{i}\leq \ln n+1 .
    \end{equation}\index{sum!harmonisk|textbf}


  \item 
    Geometrisk sum og geometrisk række:
    \begin{equation}\label{eq:geometric}
      \sum_{i=0}^{n-1}q^i=\frac{1-q^n}{1-q} \quad\text{für $q \not= 1$ und}\quad \sum_{i \ge 0} q^i =
      \frac{1}{1 - q} \quad \text{ für $|q| < 1$} .
    \end{equation}
    \index{sum!geometrisk|textbf}

    \begin{equation}\label{eq:ipowi}
      \sum_{i \ge 0} 2^{-i} = 2 \quad\text{og}\quad \sum_{i\geq 0}i\cdot 2^{-i}=
      \sum_{i\geq 1}i\cdot 2^{-i}=2 .
    \end{equation}

  \item Jensens ulighed:
    For hver konkave funktion $f$ og hver følge $(x_1,\ldots,x_n)$ af reelle tal i definitionsområrdet for $f$ gælder:
    \begin{equation}\label{eq:concave}
      \sum_{i=1}^n f(x_i)\leq n\cdot f\left(\frac{\sum_{i=1}^n x_i}{n}\right)\,,
    \end{equation}\index{ulighed!Jensens|textbf}%
    for hver konvekse funktion $f$ og hver følge $x_1,\ldots,x_n$ af reelle tal i definitionsområrdet af $f$ gælder:
    \begin{equation}\label{eq:convex}
      \sum_{i=1}^n f(x_i)\ \ge n\cdot f\left(\frac{\sum_{i=1}^n x_i}{n}\right)\,.
    \end{equation}
\end{itemize}

\subsection{Beviser}

For (\lref{eq:factorial})observerer vi først, at der gælder $n! = n(n-1)\;\cdots\; 1 \le n^n$. 
Fra analysen vides det, at eksponentialfunktionen kan skrives som 
$\exp(x)=\sum_{i\ge 0}x^i/i!$.
Derfor gælder $e^n \ge n^n/n!$, hvilket medføre den nedre grænse for $n!$. 

\smallskip
For de skarpere grænser bemærker vi, at der for alle $i\ge2$ gælder uligheden 
$\ln i \ge \int_{i-1}^{i} \ln x \, dx$, hvilket medfører
\[ \ln n! = \sum_{2 \le i \le n} \ln i \ge \int_1^{n} \ln x \, dx =
\Bigl[ x (\ln x - 1)
\Bigr]_{x = 1}^{x = n} = n (\ln n - 1) + 1  \,. \]
Heraf følger
\[ n! \ge e^{n (\ln n - 1)+1} = e(e^{\ln n - 1})^n = e\left(\frac{n}{e}\right)^n \,. \]
På samme måde følger af uligheden $\ln (i-1) \le \int_{i-1}^{i} \ln x \, dx$,
at der gælder $(n-1)! \le \int_1^{n} \ln x \, dx = e\left(\frac{n}{e}\right)^n$, hvilket medfører $n! \le (en)\left(\frac{n}{e}\right)^n$.

Uligheden (\lref{eq:bincoeff}) følger næsten direkte af \eqref{eq:factorial}. 
Vi har
\[  \binom{n}{k} = \frac{n(n-1)\cdots(n - k+1)}{k!} \le \frac{n^k}{(k/e)^k} = \left(\frac{n\cdot e}{k}\right)^k \,. \]
\smallskip


Ligning (\lref{eq:sumi}) fås ved udregning med et enkelt kneb:
\begin{align*}
1 +  \cdots + n &= \tfrac{1}{2}\bigl( (1 +  \cdots  + n) + (n +  \cdots +
 1)\bigr)\\
&= \tfrac{1}{2}\bigl( (1 + n) +  \cdots  + (n + 1)\bigr)\\
        &= \frac{n (n + 1)}{2} \,.\end{align*}
Det er ikke vanskelig at begrænse de tilsvarende summer af højere potenser.
For eksempel gælder
 $\int_{i-1}^i x^2\,
dx \le i^2 \le \int_i^{i+1} x^2\, dx$; heraf følger
\[ \sum_{1 \le i \le n} i^2 \le \int_1^{n+1} x^2\, dx = \Bigl[
\frac{x^3}{3}\Bigr]_{x = 1}^{x = n+1} =
\frac{(n+1)^3 - 1}{3} \]
og 
\[ \sum_{1 \le i \le n} i^2 \ge \int_0^{n} x^2\, dx = \Bigl[ \frac{x^3}{3}
\Bigr]_{x = 0}^{x = n} =
\frac{n^3}{3} \,.\]\smallskip

For (\lref{eq:harmonic}) benytter vi ligeledes begrænsningen baseret på integralet\index{sum!integralbegrænsning}. 
Der gælder
$\int_i^{i+1}( 1/x)\, dx\leq  1/i\leq \int_{i-1}^i (1/x)\, dx$, og derfor
\[ \ln n = \int_1^n \frac{1}{x} \, dx \le \sum_{1 \le i \le n} \frac{1}{i}
\le  1 + \int_1^{n-1} \frac{1}{x}\, dx \le 1 + \ln n \,.\]
\smallskip

Ligning (\lref{eq:geometric}) følger af
\[ (1 - q) \cdot \sum_{0 \le i \le n-1} q^i= \sum_{0 \le i \le n-1} q^i - \sum_{1 \le
i \le n } q^i  = 1 - q^n \,. \]
For $|q|<1$ kan vi lade $n$ gå mod uendelig, får at nå frem til  $\sum_{i \ge 0} q^i = 1/(1 - q)$. 
For $q = 1/2$ får vi $\sum_{i \ge 0}  2^{-i} = 2$.
Envidere gælder
\begin{align*}
\sum_{i \ge 1} i\cdot 2^{-i} &= \sum_{j \ge 1} 2^{-j} + \sum_{j \ge 2} 2^{-j} +
\sum_{j \ge 3} 2^{-j} + \ldots \\
  &= \left( 1 + \tfrac{1}{2} + \tfrac{1}{4} + \tfrac{1}{8} + \cdots\right) \cdot \sum_{i \ge 1} 2^{-i}\\
&= 2 \cdot 1 = 2 \,.
\end{align*}
For den første ligning skal man blot observere, at for hvert $i$ optræder termen $2^{-i}$ netop i de første $i$ af højresidens $j$-summer.
\smallskip

Ligning (\lref{eq:concave}) kan man vise ved induktion efter $n$.
For $n = 1$ er der intet at vise. 
Betragt altså $n \ge 2$.
Sæt $x^* = \sum_{1 \le i \le n} x_i/n$ og $\bar{x} = \sum_{1 \le i \le n - 1} x_i/(n-1)$. 
Da gælder $x^* =( (n-1)\bar{x} + x_n)/n$, og derfor
\begin{align*}
\sum_{1 \le i \le n} f(x_i) &= f(x_n) + \sum_{1 \le i \le n -1 } f(x_i) \\
& \le f(x_n) + (n - 1) \cdot f(\bar{x})
 = n \cdot \left(\frac{1}{n}\cdot f(x_n) + \frac{n-1}{n} \cdot f(\bar{x})
\right)\\
&\le n \cdot f(x^*) \,,
\end{align*}
hvor den første ulighed bruger induktionshypotesen og den anden ulighed bruger definitionen af begrebet »konkav«, med
$x = x_n$, $y = \bar{x}$ og $t = 1/n$. 
Udvidelsen til konvekse funktioner følger umiddelbart, idet $-f$ er konkav, når $f$ er konveks.

