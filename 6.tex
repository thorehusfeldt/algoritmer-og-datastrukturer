\chapter{Prioritetskøer}
\label{ch:pq}
\vspace*{-4.5cm}
\begin{flushright}
\includegraphics[width=6cm]{img/queue2.eps}
\end{flushright}
\vspace*{1cm}

\noindent
{\itshape
Virksomheden \emph{KnM} sælger skræddersyet tøj af højeste kvalitet.
Virksomheden er ansvarlig for markedsføring, opmåling, forsendelse, osv. men selve fremstillingen er overladt til uafhængige skrædderier.
Virksomheden beholder 20\% af overskuddet.
Da den blev grundlagt i det nittende århundrede, fandtes der fem kontraktskrædderier.
I dag behersker den 15\% af verdensmarkedet, på verdensplan arbejder tusindvis af skrædderier for \emph{KnM}.

Din opgaver er at tildele ordrer til kontraktskrædderierne.
Herved følges den enkle regel, at næste ordre går til det skrædderi, hvis samlede ordreværdi i indeværende år er mindst.
Grundlæggerne af \emph{KnM} benyttede en kridttavle for at holde rede på de løbende ordresummer -- i datalogisk terminologi havde de en lineær liste af værdier og brugte lineær tid til at finde det rigtig skrædderi.
Imidlertid er virksomheden blevet for stor til denne løsning.
Kan du give en »skalerbar« løsning, som kun kræver, at man kigger på et lille antal indgange for at finde ud af, hvem der skal få næste ordre?

Næste år ændres reglen.
For at gøre det attraktivt at levere til tiden, modtager skrædderiet nu den næste ordre, hvis dens \emph{udestående} bestillinger har den mindste totale værdi.
Det betyder, at så snart en ny bestilling leveres, skal dens værdi trækkes fra det pågældende skrædderis udestående ordreværdi.
Er din strategi for fordeling af bestillinger tilstrækkelig fleksibel til også at tage hensyn til den nye regel uden tab af effektivitet?
}

\footnotetext[1]{Billedet viser en kø foran Mao-mausoleet (V. Berger, se \texttt{http://commons.wikimedia.org/wiki/Image:Zhengyangmen01.jpg}).}
\setcounter{footnote}{1}

\bigskip\noindent
Den datatype,
% TODO orig uses datastruktur
som bedst løser problemet foroven, kaldes for \emph{prioritetskø}.
\index{prioritetskø|textbf}
Vi begynder med at angive datatypens specifikation.
Prioritetskøer forvalter en mængde $Q$ af \emph{indgange} med \emph{nøgler}
\index{nzgle@nøgle}
under følgende operationer:

\begin{tabular}{ll}
$Q.\Id{opbyg}(\set{e_1,\ldots,e_n})$ & $Q \Is \set{e_1,\ldots,e_n}$ \\
$Q.\Id{tilføj}(e)$ & $Q\Is Q\cup\set{e}$ \\
$Q.\Id{mindste}$ &  \Return $\min Q$
\index{prioritetskø!minimum|textbf}\index{prioritetskø!mindste@\Id{mindste}|textbf}  
 (indgang med mindste nøgle).\\
$Q.\Id{fjernMindste}$& $e\Is\min Q$;\quad 
   $Q\Is Q\setminus\set{e}$;\quad \Return $e$\index{prioritetskø!fjernMindste@\Id{fjernMindste}|textbf}.
\end{tabular}

Disse operationer rækker til den første del af vores eksempel.
Hvert år bygger vi en ny prioritetskø, som for hvert skrædderi har en indang med en nøgle, som sættes til 0 i begyndelsen.
For at uddele en ny ordre, udtager vi en indgang $e$ med minimal nøgle, adderer ordrens værdi til denne nøgle, og putter den modificere indgang tilbage i køen.
I afsnit \ref{s:heap} præsenterer vi en enkel og effektiv implementation af denne grundfunktionalitet.

[TODO udeladt afsnit om adresserbare prioritetskøer] 

[TODO udeladt afsnit om anvendelsesområder] 

\begin{exerc}[Naiv prioritetskø]
Vis, hvordan man implementerer en (størrelsesbegrænset) prioritetskø
\index{prioritetskø!naiv}
som række.
Når køen maksimale størrelse er $w$, bruges en række af størrelse $w$; de faktisk eksisterende indgange i prioritetskøen indtager rækkens første $n$ pladser.
Sammenling kørselstiderne for operationerne af to varianter af denne implementation:
den ene holder rækkens elementer i sorteret orden, den anden holder elementerne usorteret.
\end{exerc}

\begin{exerc}[Addresserbar prioritetskø som dobbelthægtet liste] 
Vis, hvordan man implementerer en  adresserbar prioritetskø
som dobbelthægtet liste.
Hver listeknude repræsenterer en indgang i prioritetskøen. 
Sammenling kørselstiderne for operationerne af to varianter af denne implementation:
den ene holder listens elementer i sorteret orden, den anden holder elementerne usorteret.
implementieren kann. 
\end{exerc}

%%%%%%%%%%%%%%%%%%%%%%%%%%%%%%%%%%%%%%%%%%%%%%%%%%%%%%%%%%%%%%%%%%%%%%
\section{Binærhobe}\label{s:heap}\index{prioritetskø!binærhob|textbf}%
\index{binærhob|sieheunter{prioritetskø}}

En hob (eller binærhob) er en enkel og effektiv implementation af en ikke-adresserbar prioritetskø \cite{Williams64}\index{Williams, J. W. J.}.
Ganske som en række kan den udvides til en ubegrænsede udgave
\index{prioritetskø!begrænset}\index{prioritetskø!ubegrænset}
(s. afs.~\ref{s:array}). 
Man kan sågar bruge hobe til at implementere adresserbare prioritetskøer, men vi skal lære endnu bedre muligheder at kende senere.

\begin{figure}
\begin{tikzpicture}[yscale =.4, xscale = .45]
\foreach \i in {1,...,13} \node (\i) at (\i,0) {\i};
  \node at (0,0) {$j$};

  \node at (0,-1) {$h[j]$};
  \node (A1) at (1,-1) [font = \ttfamily] {A};
  \node (C1) at (2,-1) [font = \ttfamily] {C};
  \node (G1) at (3,-1) [font = \ttfamily] {G};
  \node (R1) at (4,-1) [font = \ttfamily] {R};
  \node (D1) at (5,-1) [font = \ttfamily] {D};
  \node (P1) at (6,-1) [font = \ttfamily] {P};
  \node (H1) at (7,-1) [font = \ttfamily] {H};
  \node (W1) at (8,-1) [font = \ttfamily] {W};
  \node (J1) at (9,-1) [font = \ttfamily] {J};
  \node (S1) at (10,-1) [font = \ttfamily] {S};
  \node (Z1) at (11,-1) [font = \ttfamily] {Z};
  \node (Q1) at (12,-1) [font = \ttfamily] {Q};
  \node (new1) at (13,-1) [rectangle, font = \ttfamily, nearly transparent, blue,draw] {\phantom P};

  \node (A2) at (1,-2) [inner sep = 1pt, font = \ttfamily] {A};
  \node (C2) at (2,-4) [inner sep = 1pt, font = \ttfamily] {C};
  \node (G2) at (3,-4) [inner sep = 1pt, font = \ttfamily] {G};
  \node (R2) at (4,-6) [inner sep = 1pt, font = \ttfamily] {R};
  \node (D2) at (5,-6) [inner sep = 1pt, font = \ttfamily] {D};
  \node (P2) at (6,-6) [inner sep = 1pt, font = \ttfamily] {P};
  \node (H2) at (7,-6) [inner sep = 1pt, font = \ttfamily] {H};
  \node (W2) at (8,-9) [inner sep = 1pt, font = \ttfamily] {W};
  \node (J2) at (9,-9) [inner sep = 1pt, font = \ttfamily] {J};
  \node (S2) at (10,-9) [inner sep = 1pt, font = \ttfamily] {S};
  \node (Z2) at (11,-9) [inner sep = 1pt, font = \ttfamily] {Z};
  \node (Q2) at (12,-9) [inner sep = 1pt, font = \ttfamily] {Q};
  \node (new2) at (13,-9) [rectangle, font = \ttfamily, dotted ,draw] {\phantom P};


  \draw [myblue] (A1) -- (A2);
  \draw [myblue] (C1) -- (C2);
  \draw [myblue] (G1) -- (G2);
  \draw [myblue] (R1) -- (R2);
  \draw [myblue] (D1) -- (D2);
  \draw [myblue] (P1) -- (P2);
  \draw [myblue] (H1) -- (H2);
  \draw [myblue] (W1) -- (W2);
  \draw [myblue] (J1) -- (J2);
  \draw [myblue] (S1) -- (S2);
  \draw [myblue] (Z1) -- (Z2);
  \draw [myblue] (Q1) -- (Q2);
  \draw [myblue] (new1) -- (new2);
  
  \draw (A2) -- (C2); 
  \draw (A2) -- (G2); 
  \draw (C2) -- (R2); 
  \draw (C2) -- (D2); 
  \draw (G2) -- (P2); 
  \draw (G2) -- (H2); 
  \draw (R2) -- (W2); 
  \draw (R2) -- (J2); 
  \draw (D2) -- (S2); 
  \draw (D2) -- (Z2); 
  \draw (P2) -- (Q2); 
  \draw [dotted] (P2) -- (new2);  
\begin{scope}[xshift = 20cm]
\tikzstyle{treenode} = [circle, draw, inner sep = 1pt, font = \ttfamily];
  \node (A3) at ( 0,-2)   [treenode, label = right:1] {A};
  \node (C3) at (-2,-4)   [treenode, label = right:2] {C};
  \node (G3) at ( 2,-4)   [treenode, label = right:3] {G};
  \node (R3) at (-3.5,-6) [treenode, label = right:4] {R};
  \node (D3) at (-1.5,-6) [treenode, label = right:5] {D};
  \node (P3) at ( 1.5,-6) [treenode, label = right:6] {P};
  \node (H3) at ( 3.5,-6) [treenode, label = right:7] {H};
  \node (W3) at (-4,-9)   [treenode, label = below:8] {W};
  \node (J3) at (-3,-9)   [treenode, label = below:9] {J};
  \node (S3) at (-2,-9)   [treenode, label = below:10] {S};
  \node (Z3) at (-1,-9)   [treenode, label = below:11] {Z};
  \node (Q3) at ( 1,-9)   [treenode, label = below:12] {Q};
  \node (B3) at ( 2,-9) [treenode, label = below:13, dotted] {\phantom P};
  \draw (A3) -- (C3); 
  \draw (A3) -- (G3); 
  \draw (C3) -- (R3); 
  \draw (C3) -- (D3); 
  \draw (G3) -- (P3); 
  \draw (G3) -- (H3); 
  \draw (R3) -- (W3); 
  \draw (R3) -- (J3); 
  \draw (D3) -- (S3); 
  \draw (D3) -- (Z3); 
  \draw (P3) -- (Q3); 
  \draw [dotted] (P3) -- (B3); 
\end{scope}
\end{tikzpicture}
\caption{\label{fig:heapex-bijection}
En hob med $n=12$ indgange, som er gemt i en række med $w=13$ pladser.
Figuren fremstiller hoben både som række og som det implicitte binærtræ, her tegnet på to måder.
Knuderne svarer til indeksmængden $\{1,\ldots,n\}$, hvor $1$ står for roden.
Rodens børn har indeks $2$ og $3$.
Generelt har børnene til knude $j$ indeks $2j$ og $2j+1$ (forudsat at disse tal ligger i $\{1,\ldots, n\}$).
%TODO index error in original
Forgængeren til knude $j$ for $j\in\{2,\ldots, h\}$ har indeks $\floor{j/2}$.
Der gælder invariant i dette implicitte træ, at nøglen i en forgængerknude aldrig er større end nøglerne i nogen af dens børneknuder, dvs. at hele træet er hobordnet.
%TODO einem Kinderknoten -> seinem
}
\end{figure}


\begin{figure}
\begin{tikzpicture}[scale = .4]
\foreach \i in {1,...,13} \node (\i) at (\i,0) {\i};
  \node at (0,0) {$j$};

  \node at (0,-1) {$h[j]$};
  \node (A1) at (1,-1) [font = \ttfamily] {A};
  \node (C1) at (2,-1) [font = \ttfamily] {C};
  \node (G1) at (3,-1) [font = \ttfamily] {G};
  \node (R1) at (4,-1) [font = \ttfamily] {R};
  \node (D1) at (5,-1) [font = \ttfamily] {D};
  \node (P1) at (6,-1) [font = \ttfamily] {P};
  \node (H1) at (7,-1) [font = \ttfamily] {H};
  \node (W1) at (8,-1) [font = \ttfamily] {W};
  \node (J1) at (9,-1) [font = \ttfamily] {J};
  \node (S1) at (10,-1) [font = \ttfamily] {S};
  \node (Z1) at (11,-1) [font = \ttfamily] {Z};
  \node (Q1) at (12,-1) [font = \ttfamily] {Q};
  \node (new1) at (13,-1) [rectangle, font = \ttfamily, nearly transparent, blue,draw] {\phantom P};

  \node (A2) at (1,-2) [inner sep = 1pt, font = \ttfamily] {A};
  \node (C2) at (2,-4) [inner sep = 1pt, font = \ttfamily] {C};
  \node (G2) at (3,-4) [inner sep = 1pt, font = \ttfamily] {G};
  \node (R2) at (4,-6) [inner sep = 1pt, font = \ttfamily] {R};
  \node (D2) at (5,-6) [inner sep = 1pt, font = \ttfamily] {D};
  \node (P2) at (6,-6) [inner sep = 1pt, font = \ttfamily] {P};
  \node (H2) at (7,-6) [inner sep = 1pt, font = \ttfamily] {H};
  \node (W2) at (8,-9) [inner sep = 1pt, font = \ttfamily] {W};
  \node (J2) at (9,-9) [inner sep = 1pt, font = \ttfamily] {J};
  \node (S2) at (10,-9) [inner sep = 1pt, font = \ttfamily] {S};
  \node (Z2) at (11,-9) [inner sep = 1pt, font = \ttfamily] {Z};
  \node (Q2) at (12,-9) [inner sep = 1pt, font = \ttfamily] {Q};
  \node (new2) at (13,-9) [rectangle, font = \ttfamily, dotted, draw] {\phantom P};


  \draw [myblue] (A1) -- (A2);
  \draw [myblue] (C1) -- (C2);
  \draw [myblue] (G1) -- (G2);
  \draw [myblue] (R1) -- (R2);
  \draw [myblue] (D1) -- (D2);
  \draw [myblue] (P1) -- (P2);
  \draw [myblue] (H1) -- (H2);
  \draw [myblue] (W1) -- (W2);
  \draw [myblue] (J1) -- (J2);
  \draw [myblue] (S1) -- (S2);
  \draw [myblue] (Z1) -- (Z2);
  \draw [myblue] (Q1) -- (Q2);
  \draw [myblue] (new1) -- (new2);
  
  \draw (A2) -- (C2); 
  \draw [very thick] (A2) -- (G2); 
  \draw (C2) -- (R2); 
  \draw (C2) -- (D2); 
  \draw (G2) -- (H2); 
  \draw [very thick] (G2) -- (P2); 
  \draw (R2) -- (W2); 
  \draw (R2) -- (J2); 
  \draw (D2) -- (S2); 
  \draw (D2) -- (Z2); 
  \draw (P2) -- (Q2); 
  \draw [very thick, dotted] (P2) -- (new2); 

  \node (A3) at (1,-7) [inner sep = 1pt, font = \ttfamily] {A};
  \node (C3) at (2,-9) [inner sep = 1pt, font = \ttfamily] {C};
  \node (G3) at (3,-9) [inner sep = 1pt, font = \ttfamily] {G};
  \node (R3) at (4,-11) [inner sep = 1pt, font = \ttfamily] {R};
  \node (D3) at (5,-11) [inner sep = 1pt, font = \ttfamily] {D};
  \node (P3) at (6,-11) [inner sep = 1pt, font = \ttfamily] {P};
  \node (H3) at (7,-11) [inner sep = 1pt, font = \ttfamily] {H};
  \node (W3) at (8,-14) [inner sep = 1pt, font = \ttfamily] {W};
  \node (J3) at (9,-14) [inner sep = 1pt, font = \ttfamily] {J};
  \node (S3) at (10,-14) [inner sep = 1pt, font = \ttfamily] {S};
  \node (Z3) at (11,-14) [inner sep = 1pt, font = \ttfamily] {Z};
  \node (Q3) at (12,-14) [inner sep = 1pt, font = \ttfamily] {Q};
  \node (new3) at (13,-14) [inner sep = 1pt, font = \ttfamily] {P};

  \draw [myblue] (A2) -- (A3);
  \draw [myblue] (C2) -- (C3);
  \draw [myblue] (G2) -- (G3);
  \draw [myblue] (R2) -- (R3);
  \draw [myblue] (D2) -- (D3);
  \draw [myblue] (P2) -- (P3);
  \draw [myblue] (H2) -- (H3);
  \draw [myblue] (W2) -- (W3);
  \draw [myblue] (J2) -- (J3);
  \draw [myblue] (S2) -- (S3);
  \draw [myblue] (Z2) -- (Z3);
  \draw [myblue] (Q2) -- (Q3);
  \draw [myblue] (new2) -- (new3);
  
  \draw (A3) -- (C3); 
  \draw  (A3) -- (G3); 
  \draw (C3) -- (R3); 
  \draw (C3) -- (D3); 
  \draw (G3) -- (P3); 
  \draw  (G3) -- (H3); 
  \draw (R3) -- (W3); 
  \draw (R3) -- (J3); 
  \draw (D3) -- (S3); 
  \draw (D3) -- (Z3); 
  \draw (P3) -- (Q3); 
  \draw  (P3) -- (new3); 

  \node (ins) at (13,-11)[fill = white] {\Id{indsæt}($\mathtt B$)};
  \draw [callout,thick,->] (ins) -- (new2.center);

  \begin{scope}[xshift = 15cm]
    \foreach \i in {1,...,13} \node (\i) at (\i,0) {\i};
  \node at (0,0) {$j$};

  \node at (0,-1) {$h[j]$};
  \node (A1) at (1,-1) [font = \ttfamily] {A};
  \node (C1) at (2,-1) [font = \ttfamily] {C};
  \node (G1) at (3,-1) [font = \ttfamily] {G};
  \node (R1) at (4,-1) [font = \ttfamily] {R};
  \node (D1) at (5,-1) [font = \ttfamily] {D};
  \node (P1) at (6,-1) [font = \ttfamily] {P};
  \node (H1) at (7,-1) [font = \ttfamily] {H};
  \node (W1) at (8,-1) [font = \ttfamily] {W};
  \node (J1) at (9,-1) [font = \ttfamily] {J};
  \node (S1) at (10,-1) [font = \ttfamily] {S};
  \node (Z1) at (11,-1) [font = \ttfamily] {Z};
  \node (Q1) at (12,-1) [font = \ttfamily] {Q};
  \node (new1) at (13,-1) [rectangle, font = \ttfamily, nearly transparent, blue,draw] {\phantom P};

  \node (A2) at (1,-2) [inner sep = 1pt, font = \ttfamily] {A};
  \node (C2) at (2,-4) [inner sep = 1pt, font = \ttfamily] {C};
  \node (G2) at (3,-4) [inner sep = 1pt, font = \ttfamily] {G};
  \node (R2) at (4,-6) [inner sep = 1pt, font = \ttfamily] {R};
  \node (D2) at (5,-6) [inner sep = 1pt, font = \ttfamily] {D};
  \node (P2) at (6,-6) [inner sep = 1pt, font = \ttfamily] {P};
  \node (H2) at (7,-6) [inner sep = 1pt, font = \ttfamily] {H};
  \node (W2) at (8,-9) [inner sep = 1pt, font = \ttfamily] {W};
  \node (J2) at (9,-9) [inner sep = 1pt, font = \ttfamily] {J};
  \node (S2) at (10,-9) [inner sep = 1pt, font = \ttfamily] {S};
  \node (Z2) at (11,-9) [inner sep = 1pt, font = \ttfamily] {Z};
  \node (Q2) at (12,-9) [inner sep = 1pt, font = \ttfamily] {Q};
  \node (new2) at (13,-9) [rectangle, font = \ttfamily, dotted, draw] {\phantom P};

  \draw [myblue] (A1) -- (A2);
  \draw [myblue] (C1) -- (C2);
  \draw [myblue] (G1) -- (G2);
  \draw [myblue] (R1) -- (R2);
  \draw [myblue] (D1) -- (D2);
  \draw [myblue] (P1) -- (P2);
  \draw [myblue] (H1) -- (H2);
  \draw [myblue] (W1) -- (W2);
  \draw [myblue] (J1) -- (J2);
  \draw [myblue] (S1) -- (S2);
  \draw [myblue] (Z1) -- (Z2);
  \draw [myblue] (Q1) -- (Q2);
  \draw [myblue] (new1) -- (new2);
  
  \draw [very thick] (A2) -- (C2); 
  \draw (A2) -- (G2); 
  \draw (C2) -- (R2); 
  \draw [very thick] (C2) -- (D2); 
  \draw (G2) -- (H2); 
  \draw (G2) -- (P2); 
  \draw (R2) -- (W2); 
  \draw (R2) -- (J2); 
  \draw [very thick] (D2) -- (S2); 
  \draw (D2) -- (Z2); 
  \draw (P2) -- (Q2); 
    \draw [dotted] (P2) -- (new2); 

  \node (A3) at (1,-7) [inner sep = 1pt, font = \ttfamily] {C};
  \node (C3) at (2,-9) [inner sep = 1pt, font = \ttfamily] {D};
  \node (G3) at (3,-9) [inner sep = 1pt, font = \ttfamily] {G};
  \node (R3) at (4,-11) [inner sep = 1pt, font = \ttfamily] {R};
  \node (D3) at (5,-11) [inner sep = 1pt, font = \ttfamily] {Q};
  \node (P3) at (6,-11) [inner sep = 1pt, font = \ttfamily] {P};
  \node (H3) at (7,-11) [inner sep = 1pt, font = \ttfamily] {H};
  \node (W3) at (8,-14) [inner sep = 1pt, font = \ttfamily] {W};
  \node (J3) at (9,-14) [inner sep = 1pt, font = \ttfamily] {J};
  \node (S3) at (10,-14) [inner sep = 1pt, font = \ttfamily] {S};
  \node (Z3) at (11,-14) [inner sep = 1pt, font = \ttfamily] {Z};
  \node (Q3) at (12,-14)  [rectangle, font = \ttfamily, dotted, draw] {\phantom P};
  \node (new3) at (13,-14)  [rectangle, font = \ttfamily, dotted, draw] {\phantom P};
[

  \draw [myblue] (A2) -- (A3);
  \draw [myblue] (C2) -- (C3);
  \draw [myblue] (G2) -- (G3);
  \draw [myblue] (R2) -- (R3);
  \draw [myblue] (D2) -- (D3);
  \draw [myblue] (P2) -- (P3);
  \draw [myblue] (H2) -- (H3);
  \draw [myblue] (W2) -- (W3);
  \draw [myblue] (J2) -- (J3);
  \draw [myblue] (S2) -- (S3);
  \draw [myblue] (Z2) -- (Z3);
  \draw [myblue] (Q2) -- (Q3);
  \draw [myblue] (new2) -- (new3);
  
  \draw (A3) -- (C3); 
  \draw  (A3) -- (G3); 
  \draw (C3) -- (R3); 
  \draw (C3) -- (D3); 
  \draw (G3) -- (P3); 
  \draw  (G3) -- (H3); 
  \draw (R3) -- (W3); 
  \draw (R3) -- (J3); 
  \draw (D3) -- (S3); 
  \draw (D3) -- (Z3); 
  \draw [dotted] (P3) -- (Q3); 
  \draw [dotted] (P3) -- (new3); 

  \node (del) at (-2,-3) [fill=white]{\Id{fjernMindste}};
  \draw [callout,thick,->] (A2) -- +(-2,0);
  \draw [callout,thick,->] (Q2) to [bend right] (A2);
\end{scope}
\end{tikzpicture}
\caption{\label{fig:heapex}
Operationer på hoben fra fig.~\ref{fig:heapex-bijection}.
\emph{Til venstre} vises effekten af at tilføje den nye indgang \texttt{B}.
Vi skriver \texttt{B} i en ny knude med index 13, som bliver et blad.
De tykke kanter markerer vejen fra den nye bladknude til roden.
Indgang \texttt{B} vandrer ad denne vej op til den første knude, hvis forgænger ikke er støore end \texttt{B}.
For at skabe plads til \texttt{B}, bliver de andre indgange på vejen forskudt et niveau længere ned.
\emph{Til højre} vises effekten af at fjerne rodindgangen.
(Denne indeholder altid den mindste nøgle.)
De tykke kanter markerer den vej $p$ ned fra roden, som i hver knude følger den mindre nøgle.
Indgangen \texttt{Q}, fundet i bladet med størst indeks, flyttes først til roden, for derefter at vandre nedad langs $p$, til den finder en knude, hvis efterfølger (på $p$) ikke er mindre end {\tt Q} eller ikke eksisterer.
De andre indgange på vejen forskydes et niveau opad, for at gøre plads til \texttt{Q}.
}
\end{figure}

\begin{figure}
\begin{tabbing}
~~~~\=~~~~\=~~~~\kill
\Class \Id{BinærHobPK}$(\Declare{w}{\NN})$ \Of Element\+\\
   \Declare{$h$}{\Array$[1..w]$ \Of \Id{Element}}\comment{Hoben $h$ er}\\
   \DeclareInit{$n$}{$\NN$}{$0$}\comment{tom i begyndelsen.}\\
   \Invariant$\forall j\in 2..n\colon h[\floor{j/2}] \le h[j\,]$\comment{hobreglen}\\
   \Function $\Id{mindste}$ \Assert $n>0$ ; \Return $h[1]$\comment{$\Rightarrow$ roden indeholder den mindste nøgle}
\end{tabbing}
\caption{
  \label{alg:heap}
  En klasse for prioritetskøer på basis af binærhobe med fast størrelsesbegrænsning $w$.}
\end{figure}

\index{invariant!datastrukturinvariant}
\index{prioritetskø!binærhob!invariant}
\index{træ!implicit defineret}
For at gemme prioritetskøens $n$ indgange benytter vi de første $n$ positioner i rækken $h[1..w]$.
Vi siger at en række en \emph{hobordnet} eller bare »udgør en hob«, dersom den opfylder \emph{hobreglen}
\[h[\floor{\,j/2}] \le h[\,j\,]\quad\text{for alle }j\in\{2,\ldots,n\}\,.\]
Hvad skal hobreglen betyde?
For at forstå denne definition betragter vi en bijektion mellen tallene $\{1,\ldots, n\}$ og knuderne i et \emph{venstrefuldstændigt} binærtræ med $n$ knuder, som er vist i fig.~\ref{fig:heapex-bijection}.
\index{binærtræ!venstrefuldstændigt}
\index{binærtræ!fuldstændigt}
Hobreglen medfører, at nøglerne langs hver vej fra en knude til roden er monotont svagt aftagende;
især gælder at at indgangen ved roden, dvs. rækkeposition 1, indeholder den minimale nøgle.
Operationen $\Id{mindste}$, som skan finde sådan en indgang, behøver altså tid  $O(1)$.
Konstruktionen af en tom hob med plads til $n$ indgange behøver ligeledes konstant tid, fordi der blot skal stilles en række af størrelse $w$ til rådighed.
\footnote{O.a.: Vi går her ud fra vores antagelser om maskinmodellen. 
TODO.}
Fig.~\ref{alg:heap} angiver pseudokoden for disse enkle operationer.

At celle $h[1]$
\index{prioritetskø!minimum}
indeholder et mindsteelement svarer til situationen i en ordnet række, som er indekseret fra $1$.
% TODO  clarify indexing in orig
Hobreglen
\index{hobregel|textbf} 
er dog et meget svagere krav på rækkestrukturen end at være sorteret.
For eksempel findes der kun én sorteret udgave af mængden $\set{1,2,3}$, men både $\seq{1,2,3}$ og $\seq{1,3,2}$ er lovlige hob-repræsentationer af denne mængde. 

\begin{exerc}
Angiv samtlige hob\-repræsentationer af mængden $\set{1,2,3,4}$.
% 1  1  1  1
% 23 24 32 34
% 4  3  4  2
% Kommentar Martin: Der letzte Baum ist kein Heap. 
\end{exerc}

Vi skal nu se, at denne større fleksibilitet tillader meget effektive implementationer af \Id{tilføj} og \Id{fjernMindste}.
Vi skal beskrive disse procedurer så de er lette at forstå og deres korrekthed er let at vise.
I afsnittet~\ref{s:implementation} skal vi skitsere, hvordan en effektiv implementation kunne se ud. 
Instættelsesoperationen $\Id{tilføj}(e)$ placerer en ny indgang $e$ i første omgang sidst i hoben, dvs. den øger $n$ med 1 og sætter $e$ på position $h[n]$.
Derefter flyttes $e$ til en passende position på vejen fra bladet $h[n]$ til roden:
\begin{quote}
\begin{tabbing}
~~~~\=~~~~\=~~~~\kill
\Procedure \Id{tilføj}(\Declare{$e$}{\Id{Element}})\+\\
  \Assert $n < w$\\
  $n \Increment$; $h[n] \Is e$\\
  \Id{sigtOp}$(n)$
\end{tabbing}
\end{quote}
\index{prioritetskø!binærhob!tilføj@\Id{tilføj}|textbf}
\index{prioritetskø!binærhob!sigtOp@\Id{sigtOp}|textbf}%
Herved skubber operationen \Id{sigtOp}$(j)$ indgangen i knude $j$ i retning af roden,  indtil denne er nået eller den møder en forgænger, vhis nøgle ikke er større (s.~fig.~\ref{fig:heapex}).
Vi skal vise, at $h$ herefter er en hob igen.
Lad os sige at »kun $h[j]$  er måske for lille«, når $h$ enten selv er en hob eller kan gøres til en hob ved at erstatte nøglen $x$ i $h[j]$ med en passende nøgle $x'$ med $x'>x$.

\begin{quote}
\begin{tabbing}
~~~~\=~~~~\=\kill
\Procedure \Id{sigtOp}$(\Declare{j}{\NN})$\+\\
\Assert kun $h[j]$ er måske for lille\\
\If $j = 1\vee h[\floor{j/2}] \le h[j]$ \Then \Return\\
\Id{ombyt}$(h[j],h[\floor{j/2}])$\\
\Assert kun $h[\floor{j/2}]$ er måske for lille\\
  $\Id{sigtOp}(\floor{j/2})$\\
\Assert $h$ er en hob
\end{tabbing}
\end{quote}

\begin{exerc}
Vis: 
(a) Antagelserne i algoritmen etabler dens korrekthed.
Hertil skal man vise, at hvis kun $h[j]$ måske er for lille, så er enten $h$ en hob eller ombytningen $\Id{ombyt}(h[j],h[\floor{j/2}])$ medfører, at kun $h[\floor{j/2}]$ måske er for lille.
(b) \Id{sigtOp}$(n)$ har kørselstid $O(\log n)$.
(Dette medfører samme grænse for \Id{tilføj}.)  
\end{exerc}

Et kald af \Id{fjernMindste} returnerer indholdet i roden og erstatter det med inholdet i knude $n$.
Fordi $h[n]$ måske er større end $h[2]$ eller $h[3]$, kan hobreglen nu være brudt.
Den genoprettes ved et kald til proceduren \Id{sigtNed}$(1)$.
\index{prioritetskø!binærhob!fjernMindste@\Id{fjernMindste}|textbf}
\index{prioritetskø!binærhob!sigtNed@\Id{sigtNed}|textbf}

\begin{quote}
\begin{tabbing}
~~~~\=~~~~\=\kill
\Function \Declare{\Id{fjernMindste}}{\Id{Element}}\+\\
\Assert $n>0$\\
\DeclareInit{\Id{resultat}}{\Id{Element}}{$h[1]$}\\
$h[1]\Is h[n]$;\quad $n\Decrement$\\
\Id{sigtNed}$(1)$\\
\Return \Id{resultat}
\end{tabbing}
\end{quote}

Proceduren \Id{sigtNed}$(1)$ skubber den nye indgang i roden nedad, til hobreglen gælder igen.
For at være mere nøjagtig, betragt vejen $p$, som begynder ved roden og ved hver knude fortsætter ved det barnm som har den mindre nøgle (s. fig.~\ref{fig:heapex});
i tilfælde af nøglelighed er beslutningen ligegyldig. 
Man følger denne vej, til man når en knude, hvis børn (hvis de findes) indeholder nøgler, som er mindst lige så store som $h[1]$s nøgle.
Indgangen $h[1]$ sættes på denne plads; alle indgange på vejen $p$ vandrer et niveau opad.
På denne måde gælder hobreglen igen.
Strategien formuleres enklest rekursivt.
\index{algoritmekonstruktion!rekursion}
Vi benytter en lignende invariant som ved \Id{sigtOp}, og siger at »kun $h[j]$ er måske for stor«, hvis enten $h$ selv er en hob eller kan gøres til en hob ved at erstatte nøglen $x$ i  $h[j]$ med en passende nøgle $x'$ med  $x' < x$. 

\begin{quote}
\begin{tabbing}
~~~~\=~~~~\=~~~~\=~~~~\=\kill
\Procedure\Id{sigtNed}$(\Declare{j}{\NN})$\+\\
\Assert kun $h[j]$ er muligvis for stor \\
\If $2j \leq n$ \Then\+\comment{$j$ er ikke et blad}\\
  \If $2j + 1 > n \vee h[2j] \le h[2j + 1]$ \Then $m \Is 2j$ \Else $m\Is 2j + 1$\\
  \Assert nøglen i søskendeknuden til  $m$,\+\+\\
  såvidt den findes, er ikke større end nøglen i $m$\-\-\\
  \If $h[j] > h[m]$ \Then\+\comment{$h[j]$ er for stor}\\
    \Id{ombyt}$(h[j],h[m])$\\
    \Assert kun $h[m]$ er muligvis for stor\\
  \Id{sigtNed}$(m)$\-\-\\
\Assert $h$ er en hob
\end{tabbing}
\end{quote}


\begin{exerc}
Vis: 
(a) Angtagelserne i algoritmen sikrer dens korrekthed. 
Hertil skal man vise, at hvis kun $h[j]$ måske er for stor,
så er enten $h$ en hob, eller operationen \Id{ombyt}$(h[j],h[m])$ medfører, at kun $h[m]$ måske er for stor.
(b)
Prodeduren \Id{sigtNed} behøver ikke mere end $2\log n$ nøglesammenligninger. 
(c) Vis, at antallet af nøglesammenligninger kan reduceres til $\log n +O(\log\log n)$.
\emph{Vink}: 
Bestem først »vejen af mindre børn« $p$ fra roden til et blad.
Brug nu binærsøgning for at finde den rigtige plads til $h[1]$.
(Se afsnit~\ref{s:further} for andre varianter af \Id{sigtNed}.)
\end{exerc}

Vi betragter nu operationen \Id{opbyg} for hobe.
Det er åbenlyst, at man kan opbygge en hob af $n$ givne indgange i tid $O(n\log n)$ 
\index{prioritetskø!binærhob!opbygning}
ved at tilføje dem en efter en.
Interessant nok kan dette gøres bedre ved at opbygge hoben nedefra og op.
Med proceduren \Id{sigtNed} kan man etablere hobreglen i et undertræ af højde $k+1$, hvis den allerede gjaldt i begge undertræer af højde $k$.
I den følgende opgave behandles denne ide i detaljer.
\index{algoritmekonstruktion!rekursion}
\index{algoritmekonstruktion!del-og-hersk!hobopbygning}

\begin{exerc}[\emph{Opbyg} for binærhob] \label{heapify}
Betragt en vilkårlig række $h[1..n]$ af indgange. 
Disse skal permuteras således, at hobreglen gælder.
Hertil bruges følgende to procedurer:
\begin{tabbing}
~~~~\=~~~~\=\kill
\Procedure \Id{opbygHobBaglæns}\+\\
  \ForFromDownto{j}{\floor{n/2}}{1} \Id{sigtNed}$(j)$\-\\[2mm]
\Procedure \Id{opbygHobRekursivt}$(\Declare{j}{\NN})$\+\\
    \If $4j\leq n$ \Then\+\\
       \Id{opbygHobRekursivt}$(2j)$\\
       \Id{opbygHobRekursivt}$(2j+1)$\-\\
    \Id{sigtNed}$(j)$
\end{tabbing}

\begin{enumerate}[(a)]
\item Vis, at både \Id{opbygHobBaglæns} og \Id{opbygHobRekursivt}$(1)$ etablerer hobreglen.
\item Implementer begge algoritmer effektivt og sammenlign deres kørselstider for tilfældige tal  som nøgler og $n\in\{\,10^i\colon 2\leq i\leq 8\}$.
Det kommer meget an på, hvor effektiv implementationen af \Id{opbygHobRekursivt} bliver.
For små undertræer kan det være meningsfuld at »udrede« rekursionen for små undertræer, dvs. at erstatte den med direkte opdeling i forskellige tilfælde.
\item (*) For store $n$ udgøres forskellen mellem de to algoritmer hovedsageligt af de effekter, som lagerhierarkiet 
\index{fjernlager!hobopbygning}
medfører.
Analyser begge algoritmer med henblik på antallet af nødvendige i/u-operationer i fjernlagermodellen, som beskrevet sidst i afsnit~\ref{ch:intro:s:model}. 
Vis især, at \Id{opbygHobRekursivt} kun behøver $O(n/B)$ i/u-operationer, når blokstørrelsen er $b$ og det hurtige lager har størrelse $M=\Omega(B\log B)$ hat. 
\end{enumerate}
\end{exerc}

Følgende sætning sammenfatter vores resultater om hobe:

\begin{thm}\label{heap priority queue}
Betragt en ikke-adresserbar prioritetskø implementeret som hob.
Konstruktionen af en tom hob og funktionen \Id{mindste} tager konstant tid, operationerne \Id{fjernMindste} og \Id{tilføj} kan udføres i logaritmisk tid $O(\log n)$, og \Id{opbyg} tager lineær tid.
\end{thm}
\begin{proof} 
Det implicitte binærtræ, som svarer til en hob med $n$ indgange, har dybde $k = \floor{\log n}$.
Operationerne \Id{tilføj} og \Id{fjernMindste} gennemløber en del af vejen fra roden til et blad og har derform logaritmisk kørselstid.
Operationenen \Id{mindste} rapporterer indholdet af roden, hvilket tager konstant tid.
Konstruktionen af en tom hob kræver blot allokering af en række, hvilket tager konstant tid.
Uanset hvilken af de to procedurer for \Id{opbyg} man bruger:
Operationen \Id{sigtNed} kaldes højst en gang for de hver af de højst $2^l$ knuder på dybde $l$. 
Et sådant kald tager tid $O(k- l)$. 
Den totale kørselstid for  \Id{opbyg} er derfor asymptotisk begrænset af
\[
	 \sum_{l=0}^{k-1} 2^l (k - l) =
	 2^k \sum_{l=0}^{k-1} \frac{k - l}{2^{k - l}} \leq
	 2^k \sum_{j \ge 1} \frac{j}{2^j} =
	 O(n) \,.
 \]
%TODO orig uses more Landau symbols
Den sidste ulighed benytter formel (\ref{app:notation:eq:ipowi}).
\end{proof}

Hobe danner grundlaget for sorteringsmetoden \emph{hobsortering}\index{sortering!hobsortering@\Id{hobsortering}|textbf}. 
Ud fra $n$ indgange, som skal sorteres, dannes i første omgang en hob $h$ med proceduren \Id{opbyg}. 
Derefter følger $(n-1)$ kald af \Id{fjernMindste}.
Herved forkortes hoben med en indgang ad gangen.
Inden det $i$te kald af \Id{fjernMindste} står den $i$te-mindste indgang i roden $h[1]$.
Vi ombytter $h[1]$ og $h[n-i+1]$ og lader den nye rodindgang vandre nedad til sin rette plads i den resterende hob $h[1..n-i]$ med  \Id{sigtNed}.
Til sidst står indgangene i rækken i faldene orden.
Selvfølgeligt kan man også sortere i stigende orden ved at bruge en »størsteprioritetskø«, dvs. en datastruktur, som tillader tilføjelse af en indgang samt fjernelse af den \emph{største} indgang.
Hertil skal man blot erstatte »$\le$« med »$\ge$« hhv. »$<$« med »$>$« i samtlige procedurer.

Binærhobe implementerer ikke umiddelbart datatypen »adresserbar prioritetskø«, fordi indgange flyttes omkring i rækken ved tilføjelse og fjernelse, og man derfor ikke kan benytte indgangens position i rækken som greb.

\begin{exerc}[Adresserbar binærhob]\index{prioritetskø!binærhob!adresserbar} 
Udvid hoben så den implementer en adresserbar prioritetskø (men uden operationen (\Id{flet}). 
Hvor mange yderligere peger per indgang er nødvendige?
Der findes en løsning, som klarer sig med to ekstra pegere per indgang.
\end{exerc}

\begin{exerc}[Massetilføjelse]\index{prioritetskø!binærhob!massetilføjelse}
Konstruer en algoritme, som tillader tilføjelsen af $k$ nye indgange til en binærhob med $n$ indgange, og som har kørselstid $O(k+\log n)$.
\emph{Vink}: Gennemgå hoben nedefra og op, som ved hobopbygning.
\end{exerc}

\bigskip
[TODO resten af kapitlet udeladt]
