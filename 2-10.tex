\section{P og NP}
\llabel{s:P and NP}

\index{P@$\classP$|textbf}%
\index{NP@$\NP$|textbf}%
\index{effektivitet|siehe{udførelsestid}}%

Hvornår kalder vi en  algoritme for »effektiv«?
Findes der problemer, for hvilke der ikke findes effektive algoritmer?
Selvfølgeligt er det lidt af en smagssag, hvor man præcis vil trække grænsen mellem »effektive« og »ueffektive« algoritmer.
Men det har vist sig, at følgende adskillelse er nyttig:
En algoritme $\mathcal{A}$ bruger \emph{polynomiel tid} eller er en  \emph{polynomialtidsalgoritme},
\index{udførelsestid!polynomiel|textbf}
\index{algoritme!polynomialtids-|textbf}
hvis der findes et polynomium $p(n)$, således at udførelsestiden  af $\mathcal{A}$ på instanser af størrelse $n$ tilhører $O(p(n))$. 
Medmindre andet er sagt, vil vi måle instansstørrelsen i antal bit.
Vi siger at et problem \emph{kan løses i polynomiel tid}, hvis der findes en polynomialtidsalgoritme for det.
Nu sætter vi lighedstegn mellem »kan løses effektivt« og »kan løses i polynomiel tid«. 
En af fordelene ved denne definition er, at implementationsdetaljer normalt ikke spiller nogen rolle.
For eksempel er det underordnet, om en snedig datastruktur er i stand til at reducere udførelsestiden af en $O(n^3)$-algoritme med en faktor $n$.
Alle kapitler i denne bog, med undtagelse af kapitel~\ref{ch:optimization:}, drejer sig om effektive algoritmer.
  
Der findes mange problemer, for hvilke der findes algoritmer, uden at man kender en \emph{effektiv} algoritmer for dem.
Vi vil nævne seks eksempler:
\begin{itemize}
  \item Hamiltonkredsproblemet:
    Givet en urettet graf.
    Afgør, om den indeholder en hamiltonkreds.
    \index{kreds!hamilton-}
  \item Opfyldningsproblemet for boolske formler:
    \index{tildelingsproblem|textbf} 
    Givet en boolsk formel på \emph{konjunktiv normalform}. 
    Afgør, om der findes en opfyldende tildeling. --
    Her er en boolsk formel på konjunktiv normalform en konjunktion 
    $C_1 \wedge \cdots \wedge C_k$ af \emph{klausuler}. 
    En klausul er en disjunktion $(\ell_1 \vee \cdots \vee \ell_h)$ af \emph{literaler},
    og en literal er en boolsk variabel $x_i$ eller en negeret boolsk variabel $\neg x_i$.  
    Et eksempel på en klausul er $(x_1 \vee \neg x_3 \vee \neg x_9)$. 
    En tildeling knytter en boolsk værdi til hver variabel, den er \emph{opfyldende}, hvis den gør formlen sand.
  \item Klikeproblemet:
    \index{klike|sieheunter{graf}}
    \index{graf!klike|textbf}
    \index{graf!fuldstændig}
    Givet en urettet graf og et naturligt tal $k$.
    Afgør, om grafen indeholder en fuldstændig delgraf (også kaldt en \emph{klike}) med $k$ knuder.
    -- En graf kaldes \emph{fuldstændig}, hvis hvert par af knuder er forbundet med en kant;
    et eksempel er »5-kliken« $K_5$ i figur.~\lref{fig:graphexample}.
  \item Rygsæksproblemet:
    \index{rygsæk|textbf} 
    Givet $n$ par $(w_i,p_i)$  af naturliget tal for $i\in \{1,\ldots, n\}$ og to naturlige tal $M$ og $P$. 
    Afgør, om der findes en delmængde $I \subseteq \{1,\ldots, n\}$ af indeks, som opfylder 
    $\sum_{i \in I} w_i \le M$ og $\sum_{i \in I} p_i \ge P$.
    -- Man skal tænke på dette som $n$ givne objekter, hvor objekt $i$ har volumen $w_i$ og profit $p_i$, og som man prøver at pakke i en rygsæk med volumen $M$ for at opnå en samlet profit på mindst $P$.
  \item Handelsrejsendeproblemet:
    \index{handelsrejsendeproblemet|textbf}
    \index{tsp|siehe{handelsrejsendeproblemet}}
    Givet en kantvægtet urettet graf og det naturligt tal $C$.
    Afgør, om grafen indeholder en hamiltonkreds af længde højst $C$.
    (Se nærmere i afsnit~\ref{ch:mst:ss:tsp}.)
  \item Graffarvningsproblemet:
    \index{graf!farvning|textbf} 
    Givet en urettet graf og et naturligt tal $k>0$.
    Afgør, om der findes en farving af knuderne i grafen med $k$ farver, således hvert par af naboknuder har forskellige farver.
\end{itemize}
Bare fordi vi ikke kender nogle effektive algoritmer for noget af disse seks problemer, kan vi selvfølgeligt ikke udelukke, at den slags algoritmer alligevel findes.
Man ved det helt enkelt ikke.
Især har vi ingen matematiske beviser for, at der for disse problemer ikke kan findes en effektiv algoritme.
Generelt er det meget vanskeligt at bevise, at der for et givet problem ikke kan findes en algoritme inden for en bestemt tidsgrænse.
(Enkle eksempler på den slags nedre grænser skal vi dog lære at kende i kapitel~\ref{ch:sort:s:lower}.)
De fleste algoritmikere tror dog ikke, at der findes effektive algoritmer for de seks problemer.

Teorien for beregningskompleksitet,
\index{beregningskompleksitet}
\index{kompleksitetsteori}
bland dataloger ofte bare kaldt »kompleksitetsteorien«, giver en interessant måde at argumentere for disse problemers sværhed.
Teorien sammenfatter algoritmiske problemer i store grupper, kaldt »kompleksitetsklasser«, således at problemer i samme klasse er ækvivalente med hensyn til deres kompleksitet.
Især findes der en stor gruppe af ækvivalente problemer, som kaldes
\emph{$\NP$-fuldstændige}.
\index{NP-fuldstændig@$\NP$-fuldstændig|textbf} 
Her er »$\NP$« en forkortelse for »nondeterministisk polynomiel tid«. 
Det er unødvendigt for forståelsen af resten af afsnittet, om læseren kender begrebet »nondeterministisk polynomiel tid« eller ej. 
De seks problemer foroven er allesammen $\NP$-fuldstændige, ligesom mange andre naturlige problemer.
I resten af asnittet vil vi definere klassen $\NP$ og klassen af $\NP$-fuldstændige problemer formelt.
For en grundig fremstilling af teorien henviser vi til lærebøger i beregningsteori og kompleksitetsteori~\cite{Ausiello-et-al:book,Garey-Johnson:book,Sipser,Wegener:ComplexityTheory}.
\index{Garey, M. R.}
\index{Johnson, D. S.}
\index{Ausiello, G.}
\index{Sipser, M.}
\index{Wegener, I.}

Som sædvandligt i kompleksitetsteorien går vi ud fra, at instansen til problemet er kodet som streng over et fast, endeligt alfabet $\Sigma$ med $|\Sigma|\ge2$.
(Man kan tænke sig ASCII- eller Unicode-alfabeterne, eller binær\-repræsentationen af disse alfabeter. 
I det sidste tilfælde er $\Sigma=\{0,1\}$.)
Mængen af strenge (eller »ord«) med bogstaver fra $\Sigma$ betegnes $\Sigma^*$.
For $x=a_1\cdots a_n \in\Sigma^*$ bruger vi antallet $|x|$ af tegn i $x$ som størrelsesmål, med konventionen $n=|x|$.\footnote{
Undtagelsen er grafer.
I denne bog, ligesom i resten af literaturen, bruges $n$ for at betegne antallet af \emph{knuder} i grafen, se afsnit~\ref{s:graphnot}.
Størrelsen af en graf med $n$ knuder og $m$ kanter er derimod $\Theta(n+m)$ eller $\Theta(n^2)$, afhængigt af den valgte repræsentation (se afsnit~\ref{ch:grepresent}).
Graferne i denne bog er næsten udelukkende »simple« (der gælder $E\subseteq V\times V$, dvs. vores grafer har ingen multikanter i samme retning), så vi har $m\leq n^2$. 
Derfor er  grafers størrelse polynomiel i $n$, og den uheldige sammenblanding af $n$ og $n+m$ gør ingen forskel med hensyn til polynomialtidsreduktioner.}
%TODO add footnote to original
Et \emph{beslutningsproblem}
\index{beslutningsproblem} 
er en delmængde $L \subseteq \Sigma^*$.
Med $\chi_L\colon \Sigma^*\to\{0,1\}$ (læs: »chi«, udtalt med k) betegner vi den  \emph{karakteristiske funktion}
\index{karakteristisk funktion} af $L$, definieret som 
\begin{equation*}
  \chi_L(x)= 
  \begin{cases}1\,, &\text{hvis } x \in L\,,\\
    0\,,& \text{hvis } x \not\in L\,.    
  \end{cases}
\end{equation*}
Et beslutningsproblem kan \emph{løses i polynomiel tid}, hvis dens karakteristiske funktion kan beregnes i polynomiel tid.
Med $\classP$ betegnes klassen af beslutningsproblemer, som kan løses i polynomiel tid.
Et beslutningsproblemer $L$ tilhører klassen $\NP$, hvis der findes et prædikat $Q(x,y)$,
(dvs. en mængde $Q\subseteq(\Sigma^*)^2$) og en polynomium $p$, således at der gælder følgende:  
\begin{enumerate}[(1)]
\item For hvert $x \in \Sigma^*$ gælder $x \in L$, hvis og kun hvis der eksisterer $y \in \Sigma^*$ med $\abs{y} \le p(\abs{x})$ og således at $Q(x,y)$ gælder;
\item den karakteristike funktion for $Q$ kan beregnes i polynomiel tid. 
\end{enumerate}
Hvis  $x\in L$ og $y$ opfylder $Q(x,y)=\mathit{sand}$, kalder vi $y$ for et \emph{vidne} eller \emph{certifikat} for $x$ ($y$ bevidner altså den omstændighed, at $x\in L$).
For vores eksempelproblemer kan man let vise, at de tilhører $\NP$.
For hamiltonkredsproblemet er vidnet en hamiltonkreds i instansgrafen.
Et vidne for en boolsk funktion er en tildeling af de variable, som gør formlen sand.
At en instans rygsæksproblemet kan løses, bevidnes af en delmænge af objekterne, hvis samlede volumen får plads i rygsækken og som samlet udgør mindst den ønskede profit.

\begin{exerc}
  Vis, at klikeproblemet \index{graf!klike}, handelsrejsendeproblemet \index{handelsrejsendeproblem} og graffarvningsproblemet tilhører $\NP$.
\end{exerc}

Man formoder, at $\classP$ er en ægte delmængde af $\NP$.
Selvom der findes gode argumenter for denne formodning, som vi straks skal se, er den ikke bevist.
Et bevis for formodningen ville medføre, at der ikke findes effektive algoritmer for $\NP$-fuldstændige problemer. 

Beslutningsproblemet $L$  kan \emph{reduceres i polynomiel tid}
 (eller bare \emph{reduceres})
\index{reduktion|textbf}
\index{reducibel|textbf} 
til beslutningsproblemet $L'$, 
hvis der findes en funtion $g$, som kan beregnes i polynomiel tid,
og for hvilken der gælder for alle $x \in \Sigma^*$, at $x \in L$ hvis og kun hvis $g(x) \in L'$.
Hvis $L$ kan reduceres til $L'$, og $L'$ tilhører $\classP$, indses nemt at $L$ også tilhører $\classP$.
(Bevis: Givet en algoritme for reduktionensfunktionen $g$ med polynomiel tidsgrænse $p(n)$ og en algoritme for $\chi_{L'}$ med polynomiel tidsgrænse $q(n)$.
Så har vi følgende algoritme for  $\chi_L$. 
På input $x$ beregner vi $g(x)$ i tid højst $ p(|x|)$ med den første algoritme, hvorefter vi bruger den anden algoritme for at afgøre om $g(x)\in L'$.
% TODO Turing machines suddenly appear in original
Afgørelsen tager tid $q(|g(x)|)$.
For at begrænse længden af strengen $g(x)$ er det nok an indse, at den første algoritme i tid $p(|x|)$ højst kan skabe en streng af længde $p(|x|)$; strengt taget er dette et udsagn om maskinmodellen, som behøver mindst en tidsenhed for at skabe et tegn i den nye streng og fx gemme den i en lagercelle for sig -- detaljer for forskellige repræsentationer af bl.a. tegnfølger findes i kapitel~\ref{s: sequence}.
Vi har altså, at $|g(x)| \le |x| + p(|x|)$, så den samlede udførelsestid er $p(|x|) + q(|x|+p(|x|))$,
hvilket er et polynomium i $|x|$.)
På lignende måde kan man indse, at reducibilitet er en transitiv relation.
Beslutningsproblemetet $L$  kaldes \emph{$\NP$-svært},
\index{NP-svxr@$\NP$-svær|textbf} 
hvis \emph{hvert} problem i $\NP$ kann reduceres til $L$ i polyniel tid.
Et problem kaldes \emph{$\NP$-fuldstændigt}, hvis det er  $\NP$-svært  og tilhører $\NP$.
Umiddelbart virker det meget vanskeligt at bevise, at der overhovedet findes $\NP$-fuldstændige problemer -- blandt andet skal man jo vise, at \emph{hvert} problem i $\NP$ kan reduceres til $L$. 
Men netop det gjordes i år 1971, da Cook og Levin viste (uafhænigt af hinanden), at opfyldningsproblemet for boolske formler er $\NP$-fuldstændigt~\cite{Cook71,Lev73}.
Siden har det været »let« at etablere $\NP$-fuldstændighed.
Antag, at vi vil vise, at $L$ er $\NP$-fuldstændigt.
Dertil skal vi vise to ting:
(1) $L \in \NP$, og (2) der findes et andet problem $L'$, hvis $\NP$-fuldstændighed allerede er etableret, og som kan reduceres til $L$.
Hvert nye $\NP$-fuldstændige problem gør det lettere, at vise andre problemers $\NP$-fuldstændighed. 
% TODO removed reference to Kann's survey
Vi ser nu nærmere på et eksempel på en reduktion.

\begin{lemma} 
  Opfyldningsproblemet for boolske formler kan reduceres i polynomiel tid til klikeproblemet.
\end{lemma}

\begin{proof} 
  Lad $F = C_1 \wedge \cdots \wedge C_k$ være en boolsk formel på konjunktiv normalform, med 
  $C_i = (\ell_{i1} \vee \ldots \vee \ell_{ih_i})$ og $\ell_{ij} = x_{ij}^{\beta_{ij}}$ for $i\in\{1,\ldots, n\}$ for $j\in\{1,\ldots,r\}$. 
  Her er $x_{ij}$ en boolsk variabel, og $\beta_{ij} \in \{0,1\}$.
  Toptegnet $0$ angiver en negeret variabel. 
  Betragt nu følgende graf $G$.
  Knuderne svarer til formlens literaler, dvs.  $V = \{\,r_{ij}\colon 1 \le i \le k \text{ og } 1 \le j \le h_i\}$.
  Et par af knuder $r_{ij}$ og $r_{i'j'}$ er naboer, hvis $i \not= i'$ og enten $x_{ij} \not= x_{i'j'}$ eller $\beta_{ij} = \beta_{i'j'}$.
  Med andre ord:
  Knuderne for to literaler er naboer, hvis og kun hvis literalerne hører til forskellige klausuler, og der findes en tildeling, som opfylder begge samtidigt.
  Vi påstår, at $F$ kan opfyldes, hvis og kun hvis $G$ har en klike af størrelse $k$.

  Antag først, at der findes en tildeling $\alpha$, som opfylder $F$.
  Denne tildeling opfylder mindst en literal per klausul, lad os sige literalen
  $\ell_{i,j(i)}$ i klausul $C_{i}$ for $i\in\{1,\ldots, k\}$.     
  Betragt nu den delgraf af $G$, som induceres af de $k$ mange knuder $r_{i,j(i)}$ for  $i\in\{1 ,\ldots, k\}$. 
  Vi påstår, at delgrafen er en klike.
  Antag modsætningsvist, at $r_{i,j(i)}$ og $r_{i',j(i')}$ ikke er naboer.          
  I så fald gælder $x_{i,j(i)} = x_{i',j(i')}$ og $\beta_{i,j(i)} \not= \beta_{i',j(i')}$.
  Men da er literalerne $\ell_{i,j(i)}$ og $\ell_{i',j(i')}$ hinandens komplement, og $\alpha$ kan ikke opfylde dem begge, i modstrid med konstruktionen.

  Antag omvendt, at $G$ har en klike $K$ bestående af $k$ knuder.
  Da kan vi konstruere en opfyldende tildeling $\alpha$ for $F$.
  Kliken $K$ indeholder en knude for hvert $i\in\{1,\ldots, k\}$, som vi kalder $r_{i,j(i)}$.
  Lad nu $\alpha$ være givet ved $\alpha(x_{i,j(i)}) = \beta_{i,j(i)}$, for de øvrige variable kan $\alpha$ vælges vilkårligt.
  Læg mærke til, at $\alpha$ er veldefineret, idet $x_{i,j(i)} = x_{i',j(i')}$ medfører $\beta_{i,j(i)} = \beta_{i',j(i')}$;
  ellers skulle knuderne $r_{i,j(i)}$ og $r_{i',j(i')}$ ikke være naboer. 
  Det er klart, at $\alpha$ opfylder $F$.
\end{proof}

\begin{exerc} 
  Vis, at hamiltonkredsproblemet kan reduceres til opfyldningsproblemet for boolske formler.
\end{exerc}

\begin{exerc} 
  Vis, at graffarvningsproblemet kan reduceres til opfyldningsproblemet for boolske formler.
\end{exerc}

De $\NP$-fuldstædige problemer »sidder i samme båd«.
Hvis nogen skulle være i stand til at løse \emph{et eneste} af dem i polynomiel tid, så ville $\NP=\classP$.
Idet rigtig mange mennesker uden success har prøvet at udvikle den slags algoritmer, bliver det mere og mere vanskeligt at forestille sig, at det nogensinde skulle lykkes.
% TODO skulle lykkedes?
De $\NP$-fuldstædige problemer dokumenterer altså sammenn deres egen vanskelighed.

Kan man bruge $\NP$-fuldstændighedsteorien også på optimeringsproblemer?
Optimeringsproblemer
\index{optimeringsproblem} 
(se kapitel~\ref{ch:optimization:}) 
kan let omvandles til beslutningsproblemer.
I stedet for at lede efter en optimal løsning, prøver vi bare at afgøre, om der \emph{findes} en gyldig løsning med målfunktionsværdien mindst $k$, hvor $k$ er del af input.
Den omvende konstruktion gælder ligeledes:
Hvis der findes en algoritme for at afgøre, om der findes en gyldig løsning med målfunktionsværdi mindst $k$, kan vi bruge kombinationen af eksponentiel og binær søgning
\index{søgning!eksponentiel}
\index{søgning!binær-} 
(se afsnit~\lref{s:binary search}) 
for at bestemme den optimale værdi af målfunktionen.

En algoritme for beslutningsproblemet svarer »ja« eller »nej«, afhængig af, om input tilhører det tilsvarende sprog eller ej.
Den giver dog ikke noget vidne.
Ofte kan man konstruere vidner ved at udføre beslutningsalgoritmen gentagne gange på lettere forandrede instanser.
Lad os fx antage, at grafen $G$ indeholder en klike af størrelse $k$, og at vi ønsker at finde sådan en klike.
Antag, at har en algorime, der afgør, om en givet graf indeholder en klike af en givet størrelse.
For $k=1$ kan en vilkårlig knude i $G$ bruges som svar på den eftersøgte klike. 
Eller vælger vi en vilkårlig knude $v$ i $G$ og spørger, og konstruerer grafen $G'$ som $G'=G\setminus v$ 
(notationen beskriver grafen $G$ uden knude $v$ og dens hosliggende kanter)
Vi spørger nu algoritmen, om $G'$ indeholder en klike af størrelse $k$.
I bekræftende fald, fortsætter vi søgningen efter en $k$-klike rekursivt i $G'$.
I modsat fald må $v$ indgå i samtlige $k$-kliker i $G$.
Da betragter vi mængden $V'$  af naboer til $v$ og fortsætter søgningen efter en $(k-1)$-klike $K'$ rekursivt i den af $V'$ inducerede delgraf. 
Mængden  $\set{v} \cup K'$ er en $k$-klike $G$. 
%TODO wrong in original, V' need not be a clique



%%%%%%%%%%%%%%%%%%%%%%%%%%%%%%%%%%%%%%%%%%%%%%%%%%%%%%%%%%%%%%%%%%%%%%
\section{Implementationsaspekter}
\llabel{s:implementation}

Vores pseudokode 
\index{pseudocode} 
lader sig nemt oversætte til programmer i imperative programmeringssprog.
For \CC, Java og Python går vil lidt mere i detaljer. 
Programmeringssproget Eiffel\index{Eiffel}~\cite{Meyer97}
\index{Meyer, B.}
gør det muligt at arbejde med antagelser, invarianter, før- og efterbetingelser.

Vores specialværdier $\bot$, $-\infty$ og $\infty$ stilles til rådighed for kommatal
\index{kommatal}
\index{IEEE-kommatal|siehe{kommatal}} 
af programmeringssproget.
For andre datatyper skal man simulere disse værdier.
For eksempel kan man bruge det mindste og største heltal for  $-\infty$ hhv. $\infty$.
\index{uendeligt ($\infty$)}
Udefinerede pegere repræsenteres ofte af nulpegeren {\bf null}. 
Sommetider bruger vi specialværdierne $\bot$, $-\infty$ og $\infty$ kun af bekvemmelighedsgrunde;
en robust implementation skulle i forvejen undgå deres brug.
Vi skal se nogle eksempler på dette i de senere kapitler.

Randomiserede algoritmer har brug for en tilfældighedskilde.
\index{tilfældighedskilde}
Her har man valget mellem en maskinel tilfældighedsgenerator, som producerer ægte tilfældige tal, og en algoritmisk generator, der genererer pseudotilfældige tal.

\subsection{\protect\CC}\index{C++@\CC}
Man kan opfatte vores pseudokode som en kompakt notation for en delmængde af \CC.
Lagerforvaltningsoperationerne $\tAllocate$ og $\tDispose$ ligner operationerne \Id{new} og \Id{delete} i \CC.    
Når en række bliver skabt i \CC, kaldes den underliggende types standardkonstruktoren for hver indgang i rækken; det tager altså $\Omega(n)$ tid at skabe en række af længde $n$, hvorimod det kun tager konstant tid at skabe en række af $n$ elementer af heltalstypen \Id{int}. 
I bogens fremstilling foretages der ingen initialisering. 
Vi går derimod ud fra, at rækker uden udtrykkelig initialisering er fyldt med vilkårligt indhold, ofte kaldt »\emph{affald}«.
I \CC\ kan man opnå den samme opførsel ved at benytte de mere grundlæggende C-funktioner  
\Id{malloc} og \Id{free}.
Denne fremgangsmåde der dog ildeset og bør kun bruges i undtagelsestilfælde, når initialisering af rækken udgør en flaskehals i kørslen.
Hvis lagerforvaltningen for mange små objekter bliver kritisk for udførelsestiden, kan man tilpasse \CC\-standardbibliotekets klasse \Id{allocator} til respektive situation.

Vores parameterisering af klasser ved brug af $\Of$ er et specialtilfælde af \CC-sprogets mønstermekanisme.
De parametre, der ved objektdeklarationen tilføjes klassenavnet i parenteser svarer til \CC-konstruktorens parametre.

Antagelser implementeres som C-makroer i et inkluderet arkiv kaldt {\tt assert.h}.
Normalt udløser en brudt antagelse en køretidsfejl; fejlmeddelselsen indeholder antagelsen position i programteksten.
Ved at definere makroen \Id{NDEBUG} slår man den automatiske sikring af antagelser fra.

For mange af de datastrukturer og algoritmer, som diskuteres i denne bog, findes der fremragende implementationer i diverse standardbiblioteker.
Gode biblioteker er \emph{Standard Template Library} (STL)\index{STL}~\cite{Plau00},
BOOST\index{Boost}~\cite{boost} for \CC\
og LEDA\index{LEDA}~\cite{LEDAbook,LEDA-AS}.

\subsection{Java}
\index{Java}
I Java findes ingen udtrykkelig lagerforvaltning.
Derimod findes et separat lagerrensningsprogram, 
\index{lagerrensning|textbf}
\index{garbage collection|siehe{lagerrensning}}
som regelæssigt identificerer lagersegmenter, som ikke længere er i brug, og stiller dem til rådighed igen.
Dette forenkler programmeringen i Java betydeligt, men kan også have meget negative effekter på tidsforbruget.
Metoder, som afhjælper dette fænomen, ligger uden for bogens emner.
Java tillader parameteriserede klasser i form af generiske typer.
Antagelser implementeres som kommandoen \Id{assert}.

Pakken \Id{java}.\Id{util} og datastrukturbiblioteket JDSL~\cite{JDSL}
\index{JDSL}
indeholder fremragende implementationer af mange datastrukturer og algoritmer.

\subsection{Python}
\index{Python}

Pythons abstraktionsniveau er endnu højere en Javas; på nogle punkter ligger syntaksen meget tæt på vores pseudokode, på andre punkter er sproget næsten for langt fra von Neumann-arkitekturen til at en oversættelse bliver meningsfuld.
Som pædagogisk værktøj for at forstå koncepterne i denne bog, virker sproget ypperligt; vil man derimod skrive højtoptimerede programmer med fokus på effektivitet, gør man klogt i at vælge et andet sprog.

På samme måde som Java tillader Python heller ingen udtrykkelig lagerforvaltning, og oprydningsarbejdet sker i baggrundnen og uden for programmørens kontrol, med de samme fordele og ulemper.
Pythons typesystem er dynamisk, så parameterising af klasser med $\Of$ giver ikke meget mening.
Antagelser implementeres med \Id{assert}.

Mest bemærkelsesværdigt er det måske, at Python ikke stiller rækker til rådighed, men direkte bruger en liste-abstraktion, som vi først vil møde i kapitel~\ref{sec: sequence}.
Læseren aftale med sig selv, at fx udtrykket »$A$ = [0] * $N$« skaber en række af heltal af længde $N$, initialiseret til $0$, for at oversætte pseudokoden til fungerende python.
Men i virkeligheden er $A$ en pythonliste med langt mere magtfulde operationer.
Listeoperatiorne i moderne pythonfortolker er selvfølgeligt selv baseret på meget effektive implementationer af operationer på rækker, skrevet i C, men dette abstraktionsniveau er altså ikke eksponeret til programmøren.
Modulet \Id{arrays} stiller dog »rigtige« rækker af vsse grundtyper til rådighed.
Typen \Id{Set} i vores pseudokode er på de anden side del sproget, og udtrykkene \Id{None} og \Id{float('inf')} kan bruges som vores $\bot$ og $\infty$.

%%%%%%%%%%%%%%%%%%%%%%%%%%%%%%%%%%%%%%%%%%%%%%%%%%%%%%%%%%%%%%%%%%%%%%
\section{Historiske anmærkninger og videre resultater}
\llabel{s:further}

Definitionen af registermaskinen til algoritmeanalyse blev foreslået af Shep\-herd\-son og Sturgis~\cite{Shepherdson-Sturgis}.
Modellen begrænser størrelsen af de tal, der kan gemmes en i enkelt lagercelle, til logaritmisk mange bit.
Uden denne indskrænkning opstår der uønskede konsekvenser, fordi modellen bliver alt for kraftig; fx kollapser kompleksitetsklasserne $\classP$ og PSpace~\cite{Hartmanis-Simon}.
En meget detaljeret abstrakt maskinmodel er beskrevet af Knuth~\cite{Knu99}.

Invarianter som koncept i programsemantik blev introduceret af Floyd~\cite{Floyd:meaning},
\index{Floyd, R. W.} 
og yderligere systematiseret af Hoare~\cite{Hoare:Axioms,Hoare:Data}.
\index{Hoare, C. A. R.} 
Bogen~\cite{GKP94}
\index{Knuth, D.}
\index{Graham, R. L.}
\index{Patashnik, O.}
indeholder en meget omfattende gemmengang af arbejdet med summer
\index{sum} og rekursionsligninger
\index{rekursionsligning}
og mange andre emner i »tællende« kombinatorik, som kommer til anvendelse i algoritmeanalysen.

Bøger om oversætterkonstruktion 
\index{oversætter}
(fx \cite{compiler,Wilhelm-Maurer})
\index{Wilhelm, R.}
\index{Mauer, D.}
forklarer detaljerne ved oversættelsen af højniveauprogrammeringssprog
\index{programmeringssprog} 
til maskinkode.

\index{udførelsestid|sieheauch{algoritmeanalyse}}%
\index{algoritmeanalyse|sieheauch{udførelsestid}}%
\index{kompleksitet|sieheauch{udførelsestid}}%
\index{input|siehe{instans}}
